% Options for packages loaded elsewhere
% Options for packages loaded elsewhere
\PassOptionsToPackage{unicode}{hyperref}
\PassOptionsToPackage{hyphens}{url}
\PassOptionsToPackage{dvipsnames,svgnames,x11names}{xcolor}
%
\documentclass[
  10pt,
  letterpaper,
]{scrbook}
\usepackage{xcolor}
\usepackage[letterpaper,left=1.5in,right=1.5in,top=1.25in,bottom=1.25in]{geometry}
\usepackage{amsmath,amssymb}
\setcounter{secnumdepth}{5}
\usepackage{iftex}
\ifPDFTeX
  \usepackage[T1]{fontenc}
  \usepackage[utf8]{inputenc}
  \usepackage{textcomp} % provide euro and other symbols
\else % if luatex or xetex
  \usepackage{unicode-math} % this also loads fontspec
  \defaultfontfeatures{Scale=MatchLowercase}
  \defaultfontfeatures[\rmfamily]{Ligatures=TeX,Scale=1}
\fi
\usepackage[]{libertinus}
\ifPDFTeX\else
  % xetex/luatex font selection
\fi
% Use upquote if available, for straight quotes in verbatim environments
\IfFileExists{upquote.sty}{\usepackage{upquote}}{}
\IfFileExists{microtype.sty}{% use microtype if available
  \usepackage[]{microtype}
  \UseMicrotypeSet[protrusion]{basicmath} % disable protrusion for tt fonts
}{}
\makeatletter
\@ifundefined{KOMAClassName}{% if non-KOMA class
  \IfFileExists{parskip.sty}{%
    \usepackage{parskip}
  }{% else
    \setlength{\parindent}{0pt}
    \setlength{\parskip}{6pt plus 2pt minus 1pt}}
}{% if KOMA class
  \KOMAoptions{parskip=half}}
\makeatother
% Make \paragraph and \subparagraph free-standing
\makeatletter
\ifx\paragraph\undefined\else
  \let\oldparagraph\paragraph
  \renewcommand{\paragraph}{
    \@ifstar
      \xxxParagraphStar
      \xxxParagraphNoStar
  }
  \newcommand{\xxxParagraphStar}[1]{\oldparagraph*{#1}\mbox{}}
  \newcommand{\xxxParagraphNoStar}[1]{\oldparagraph{#1}\mbox{}}
\fi
\ifx\subparagraph\undefined\else
  \let\oldsubparagraph\subparagraph
  \renewcommand{\subparagraph}{
    \@ifstar
      \xxxSubParagraphStar
      \xxxSubParagraphNoStar
  }
  \newcommand{\xxxSubParagraphStar}[1]{\oldsubparagraph*{#1}\mbox{}}
  \newcommand{\xxxSubParagraphNoStar}[1]{\oldsubparagraph{#1}\mbox{}}
\fi
\makeatother

\usepackage{color}
\usepackage{fancyvrb}
\newcommand{\VerbBar}{|}
\newcommand{\VERB}{\Verb[commandchars=\\\{\}]}
\DefineVerbatimEnvironment{Highlighting}{Verbatim}{commandchars=\\\{\}}
% Add ',fontsize=\small' for more characters per line
\usepackage{framed}
\definecolor{shadecolor}{RGB}{241,243,245}
\newenvironment{Shaded}{\begin{snugshade}}{\end{snugshade}}
\newcommand{\AlertTok}[1]{\textcolor[rgb]{0.68,0.00,0.00}{#1}}
\newcommand{\AnnotationTok}[1]{\textcolor[rgb]{0.37,0.37,0.37}{#1}}
\newcommand{\AttributeTok}[1]{\textcolor[rgb]{0.40,0.45,0.13}{#1}}
\newcommand{\BaseNTok}[1]{\textcolor[rgb]{0.68,0.00,0.00}{#1}}
\newcommand{\BuiltInTok}[1]{\textcolor[rgb]{0.00,0.23,0.31}{#1}}
\newcommand{\CharTok}[1]{\textcolor[rgb]{0.13,0.47,0.30}{#1}}
\newcommand{\CommentTok}[1]{\textcolor[rgb]{0.37,0.37,0.37}{#1}}
\newcommand{\CommentVarTok}[1]{\textcolor[rgb]{0.37,0.37,0.37}{\textit{#1}}}
\newcommand{\ConstantTok}[1]{\textcolor[rgb]{0.56,0.35,0.01}{#1}}
\newcommand{\ControlFlowTok}[1]{\textcolor[rgb]{0.00,0.23,0.31}{\textbf{#1}}}
\newcommand{\DataTypeTok}[1]{\textcolor[rgb]{0.68,0.00,0.00}{#1}}
\newcommand{\DecValTok}[1]{\textcolor[rgb]{0.68,0.00,0.00}{#1}}
\newcommand{\DocumentationTok}[1]{\textcolor[rgb]{0.37,0.37,0.37}{\textit{#1}}}
\newcommand{\ErrorTok}[1]{\textcolor[rgb]{0.68,0.00,0.00}{#1}}
\newcommand{\ExtensionTok}[1]{\textcolor[rgb]{0.00,0.23,0.31}{#1}}
\newcommand{\FloatTok}[1]{\textcolor[rgb]{0.68,0.00,0.00}{#1}}
\newcommand{\FunctionTok}[1]{\textcolor[rgb]{0.28,0.35,0.67}{#1}}
\newcommand{\ImportTok}[1]{\textcolor[rgb]{0.00,0.46,0.62}{#1}}
\newcommand{\InformationTok}[1]{\textcolor[rgb]{0.37,0.37,0.37}{#1}}
\newcommand{\KeywordTok}[1]{\textcolor[rgb]{0.00,0.23,0.31}{\textbf{#1}}}
\newcommand{\NormalTok}[1]{\textcolor[rgb]{0.00,0.23,0.31}{#1}}
\newcommand{\OperatorTok}[1]{\textcolor[rgb]{0.37,0.37,0.37}{#1}}
\newcommand{\OtherTok}[1]{\textcolor[rgb]{0.00,0.23,0.31}{#1}}
\newcommand{\PreprocessorTok}[1]{\textcolor[rgb]{0.68,0.00,0.00}{#1}}
\newcommand{\RegionMarkerTok}[1]{\textcolor[rgb]{0.00,0.23,0.31}{#1}}
\newcommand{\SpecialCharTok}[1]{\textcolor[rgb]{0.37,0.37,0.37}{#1}}
\newcommand{\SpecialStringTok}[1]{\textcolor[rgb]{0.13,0.47,0.30}{#1}}
\newcommand{\StringTok}[1]{\textcolor[rgb]{0.13,0.47,0.30}{#1}}
\newcommand{\VariableTok}[1]{\textcolor[rgb]{0.07,0.07,0.07}{#1}}
\newcommand{\VerbatimStringTok}[1]{\textcolor[rgb]{0.13,0.47,0.30}{#1}}
\newcommand{\WarningTok}[1]{\textcolor[rgb]{0.37,0.37,0.37}{\textit{#1}}}

\usepackage{longtable,booktabs,array}
\usepackage{calc} % for calculating minipage widths
% Correct order of tables after \paragraph or \subparagraph
\usepackage{etoolbox}
\makeatletter
\patchcmd\longtable{\par}{\if@noskipsec\mbox{}\fi\par}{}{}
\makeatother
% Allow footnotes in longtable head/foot
\IfFileExists{footnotehyper.sty}{\usepackage{footnotehyper}}{\usepackage{footnote}}
\makesavenoteenv{longtable}
\usepackage{graphicx}
\makeatletter
\newsavebox\pandoc@box
\newcommand*\pandocbounded[1]{% scales image to fit in text height/width
  \sbox\pandoc@box{#1}%
  \Gscale@div\@tempa{\textheight}{\dimexpr\ht\pandoc@box+\dp\pandoc@box\relax}%
  \Gscale@div\@tempb{\linewidth}{\wd\pandoc@box}%
  \ifdim\@tempb\p@<\@tempa\p@\let\@tempa\@tempb\fi% select the smaller of both
  \ifdim\@tempa\p@<\p@\scalebox{\@tempa}{\usebox\pandoc@box}%
  \else\usebox{\pandoc@box}%
  \fi%
}
% Set default figure placement to htbp
\def\fps@figure{htbp}
\makeatother





\setlength{\emergencystretch}{3em} % prevent overfull lines

\providecommand{\tightlist}{%
  \setlength{\itemsep}{0pt}\setlength{\parskip}{0pt}}



 


% preamble.tex
% LaTeX preamble for GPU-Accelerated Mathematical Illustration

% Headers and footers
\usepackage{fancyhdr}
\pagestyle{fancy}



% Use JetBrains Mono for monospace/code instead of Libertinus Mono
\usepackage{fontspec}
\setmonofont{JetBrains Mono}[
  Scale=0.85,
  UprightFont = *-Regular,
  BoldFont = *-Bold,
  ItalicFont = *-Italic,
  BoldItalicFont = *-BoldItalic
]




% % Colored boxes for callouts
% % Note: 'most' loads all common libraries including skins, breakable, etc.
% \usepackage[most]{tcolorbox}

% % Define colors matching HTML callout styles
% \definecolor{noteblue}{HTML}{347dec}
% \definecolor{tipgreen}{HTML}{28a745}
% \definecolor{importantorange}{HTML}{fd7e14}
% \definecolor{warningyellow}{HTML}{ffc107}
% \definecolor{cautionred}{HTML}{dc3545}
\makeatletter
\@ifpackageloaded{tcolorbox}{}{\usepackage[skins,breakable]{tcolorbox}}
\@ifpackageloaded{fontawesome5}{}{\usepackage{fontawesome5}}
\definecolor{quarto-callout-color}{HTML}{909090}
\definecolor{quarto-callout-note-color}{HTML}{0758E5}
\definecolor{quarto-callout-important-color}{HTML}{CC1914}
\definecolor{quarto-callout-warning-color}{HTML}{EB9113}
\definecolor{quarto-callout-tip-color}{HTML}{00A047}
\definecolor{quarto-callout-caution-color}{HTML}{FC5300}
\definecolor{quarto-callout-color-frame}{HTML}{acacac}
\definecolor{quarto-callout-note-color-frame}{HTML}{4582ec}
\definecolor{quarto-callout-important-color-frame}{HTML}{d9534f}
\definecolor{quarto-callout-warning-color-frame}{HTML}{f0ad4e}
\definecolor{quarto-callout-tip-color-frame}{HTML}{02b875}
\definecolor{quarto-callout-caution-color-frame}{HTML}{fd7e14}
\makeatother
\makeatletter
\@ifpackageloaded{bookmark}{}{\usepackage{bookmark}}
\makeatother
\makeatletter
\@ifpackageloaded{caption}{}{\usepackage{caption}}
\AtBeginDocument{%
\ifdefined\contentsname
  \renewcommand*\contentsname{Table of contents}
\else
  \newcommand\contentsname{Table of contents}
\fi
\ifdefined\listfigurename
  \renewcommand*\listfigurename{List of Figures}
\else
  \newcommand\listfigurename{List of Figures}
\fi
\ifdefined\listtablename
  \renewcommand*\listtablename{List of Tables}
\else
  \newcommand\listtablename{List of Tables}
\fi
\ifdefined\figurename
  \renewcommand*\figurename{Figure}
\else
  \newcommand\figurename{Figure}
\fi
\ifdefined\tablename
  \renewcommand*\tablename{Table}
\else
  \newcommand\tablename{Table}
\fi
}
\@ifpackageloaded{float}{}{\usepackage{float}}
\floatstyle{ruled}
\@ifundefined{c@chapter}{\newfloat{codelisting}{h}{lop}}{\newfloat{codelisting}{h}{lop}[chapter]}
\floatname{codelisting}{Listing}
\newcommand*\listoflistings{\listof{codelisting}{List of Listings}}
\makeatother
\makeatletter
\makeatother
\makeatletter
\@ifpackageloaded{caption}{}{\usepackage{caption}}
\@ifpackageloaded{subcaption}{}{\usepackage{subcaption}}
\makeatother
\usepackage{bookmark}
\IfFileExists{xurl.sty}{\usepackage{xurl}}{} % add URL line breaks if available
\urlstyle{same}
\hypersetup{
  pdftitle={GPU-Accelerated Mathematical Illustration},
  pdfauthor={Steve Trettel},
  colorlinks=true,
  linkcolor={Maroon},
  filecolor={Maroon},
  citecolor={Blue},
  urlcolor={Blue},
  pdfcreator={LaTeX via pandoc}}


\title{GPU-Accelerated Mathematical Illustration}
\usepackage{etoolbox}
\makeatletter
\providecommand{\subtitle}[1]{% add subtitle to \maketitle
  \apptocmd{\@title}{\par {\large #1 \par}}{}{}
}
\makeatother
\subtitle{An Introduction to Shader Programming}
\author{Steve Trettel}
\date{December 2025}
\begin{document}
\frontmatter
\maketitle

\renewcommand*\contentsname{Table of contents}
{
\hypersetup{linkcolor=}
\setcounter{tocdepth}{1}
\tableofcontents
}

\mainmatter
\bookmarksetup{startatroot}

\chapter*{About}\label{about}
\addcontentsline{toc}{chapter}{About}

\markboth{About}{About}

This mini-course introduces shader programming as a tool for
mathematical illustration and exploration. Shaders are programs that run
in parallel on the GPU, making them exceptionally fast for visualization
tasks. We'll learn to write code that ``reads like mathematics'' using
Shadertoy, a beginner-friendly web-based platform that handles all the
low-level programming complexities.

\pandocbounded{\includegraphics[keepaspectratio]{./demos/day1/circle-curve/screenshot.png}}

We'll progress from 2D foundations (curves, tilings, fractals) to 3D
rendering via raymarching. Along the way, we will implement classic
examples like the Mandelbrot set, hyperbolic tessellations, and implicit
surface renderers. The final day will explore either advanced geometric
techniques (domain operations, 3D fractals) or temporal simulation
methods (PDEs, buffer-based dynamics), depending on the group's
interests.

\textbf{{[}Error{]}} No GLSL files found in demos/day1/circle-curve

No prior experience with shaders or GLSL is required---only a strong
foundation in undergraduate mathematics and willingness to work hard and
experiment with code through daily homework exercises. Here are some
examples of things we will make:

The code

\textbf{{[}Error{]}} No GLSL files found in demos/day1/half-space

and the demo

\pandocbounded{\includegraphics[keepaspectratio]{./demos/day1/half-space/screenshot.png}}

and a still pic for testing

\pandocbounded{\includegraphics[keepaspectratio]{images/mandelbrot.png}}

\bookmarksetup{startatroot}

\chapter*{Outline}\label{outline}
\addcontentsline{toc}{chapter}{Outline}

\markboth{Outline}{Outline}

\section*{Course Overview}\label{course-overview}
\addcontentsline{toc}{section}{Course Overview}

\markright{Course Overview}

This mini-course introduces shader programming as a tool for
mathematical illustration and exploration. Shaders are programs that run
in parallel on the GPU, making them exceptionally fast for visualization
tasks. We'll learn to write code that ``reads like mathematics'' using
Shadertoy, a beginner-friendly web-based platform that handles all the
low-level programming complexities.

\textbf{Format:} Five days, each with one hour of lecture and
approximately 1.5 hours of homework

\textbf{Prerequisites:} Strong foundation in undergraduate mathematics;
no prior experience with shaders or GLSL required

\textbf{Audience:} Graduate students, postdocs, and faculty in
mathematics

\begin{center}\rule{0.5\linewidth}{0.5pt}\end{center}

\section*{Day 1: Introduction to Shader
Programming}\label{day-1-introduction-to-shader-programming}
\addcontentsline{toc}{section}{Day 1: Introduction to Shader
Programming}

\markright{Day 1: Introduction to Shader Programming}

\subsection*{Learning Objectives}\label{learning-objectives}
\addcontentsline{toc}{subsection}{Learning Objectives}

\begin{itemize}
\tightlist
\item
  Understand the mathematical model of shader programming (function from
  pixels to colors)
\item
  Learn basic GLSL syntax and conventions
\item
  Master coordinate system setup and distance calculations
\item
  Create simple geometric shapes and implicit curves
\end{itemize}

\subsection*{In-Class Content}\label{in-class-content}
\addcontentsline{toc}{subsection}{In-Class Content}

\begin{itemize}
\tightlist
\item
  \textbf{Mathematical framing}: Shaders as parallel functions computing
  colors for all pixels simultaneously
\item
  \textbf{GLSL basics}: Syntax, vector types, built-in functions
\item
  \textbf{Coordinate systems}: Centering, normalizing, aspect ratio
  correction
\item
  \textbf{Conditional coloring}: Half-planes and regions defined by
  inequalities
\item
  \textbf{Distance fields}: Circles, filled and outlined
\item
  \textbf{Repetition}: Using \texttt{mod()} for grids and patterns
\item
  \textbf{Implicit curves}: Rendering curves defined by \(F(x,y) = 0\)
\end{itemize}

\subsection*{Homework}\label{homework}
\addcontentsline{toc}{subsection}{Homework}

\textbf{Required:} Parabola graphing calculator - Draw coordinate axes -
Plot \(y = ax^2 + bx + c\) with customizable coefficients - Make it
robust for various parameter values

\textbf{Optional \#1:} Animated curve family (vary parameters with time)

\textbf{Optional \#2:} Beautiful tiling pattern using \texttt{mod()}

\begin{center}\rule{0.5\linewidth}{0.5pt}\end{center}

\section*{Day 2: Complex Dynamics and Iterated
Inversions}\label{day-2-complex-dynamics-and-iterated-inversions}
\addcontentsline{toc}{section}{Day 2: Complex Dynamics and Iterated
Inversions}

\markright{Day 2: Complex Dynamics and Iterated Inversions}

\subsection*{Learning Objectives}\label{learning-objectives-1}
\addcontentsline{toc}{subsection}{Learning Objectives}

\begin{itemize}
\tightlist
\item
  Implement complex number arithmetic in GLSL
\item
  Render the Mandelbrot set through escape-time iteration
\item
  Master circle inversion as a conformal transformation
\item
  Use structs to organize geometric data
\item
  Generate the Apollonian gasket through iterated inversions
\end{itemize}

\subsection*{In-Class Content}\label{in-class-content-1}
\addcontentsline{toc}{subsection}{In-Class Content}

\begin{itemize}
\tightlist
\item
  \textbf{Complex arithmetic}: Addition, multiplication, division,
  conjugation
\item
  \textbf{Mandelbrot set}:

  \begin{itemize}
  \tightlist
  \item
    Iteration \(z_{n+1} = z_n^2 + c\) with \(z_0 = 0\)
  \item
    Escape-time algorithm
  \item
    Smooth coloring and palettes
  \end{itemize}
\item
  \textbf{Circle inversion}:

  \begin{itemize}
  \tightlist
  \item
    Mathematical definition and properties
  \item
    Conformal mapping (preserves angles, maps circles to circles/lines)
  \item
    Implementation and visualization
  \end{itemize}
\item
  \textbf{Structs in GLSL}: Organizing circle data (center, radius)
\item
  \textbf{Apollonian gasket}:

  \begin{itemize}
  \tightlist
  \item
    Three mutually tangent circles
  \item
    Iterated inversions generate fractal structure
  \item
    Coloring by escape time or basin of attraction
  \end{itemize}
\end{itemize}

\subsection*{Homework}\label{homework-1}
\addcontentsline{toc}{subsection}{Homework}

\textbf{Required:} Julia sets - Implement for fixed \(c\), varying
initial \(z_0\) - Explore parameter space (try different values of
\(c\)) - Optional: animate \(c\) to watch morphing

\textbf{Optional:} Schottky groups - Four or more disjoint circles -
Alternating inversion patterns - Create intricate nested structures

\begin{center}\rule{0.5\linewidth}{0.5pt}\end{center}

\section*{Day 3: Geometric Tilings in Euclidean and Hyperbolic
Space}\label{day-3-geometric-tilings-in-euclidean-and-hyperbolic-space}
\addcontentsline{toc}{section}{Day 3: Geometric Tilings in Euclidean and
Hyperbolic Space}

\markright{Day 3: Geometric Tilings in Euclidean and Hyperbolic Space}

\subsection*{Learning Objectives}\label{learning-objectives-2}
\addcontentsline{toc}{subsection}{Learning Objectives}

\begin{itemize}
\tightlist
\item
  Create Euclidean triangle tilings through reflection
\item
  Understand hyperbolic geometry models (upper half-plane, Poincaré
  disk)
\item
  Implement hyperbolic triangle tilings using circle inversion
\item
  Convert between different hyperbolic models
\end{itemize}

\subsection*{In-Class Content}\label{in-class-content-2}
\addcontentsline{toc}{subsection}{In-Class Content}

\begin{itemize}
\tightlist
\item
  \textbf{Euclidean triangle tiling}:

  \begin{itemize}
  \tightlist
  \item
    Fundamental domain (equilateral triangle)
  \item
    Reflection across edges
  \item
    Iterative folding algorithm
  \item
    Coloring by reflection count
  \end{itemize}
\item
  \textbf{Hyperbolic geometry introduction}:

  \begin{itemize}
  \tightlist
  \item
    Upper half-plane model: \(\mathbb{H}^2 = \{z : \text{Im}(z) > 0\}\)
  \item
    Hyperbolic metric: \(ds^2 = \frac{dx^2 + dy^2}{y^2}\)
  \item
    Geodesics: vertical lines and semicircles
  \item
    Hyperbolic distance formula
  \end{itemize}
\item
  \textbf{Poincaré disk model}:

  \begin{itemize}
  \tightlist
  \item
    Unit disk representation
  \item
    Cayley transform between models
  \end{itemize}
\item
  \textbf{Hyperbolic triangle tiling}:

  \begin{itemize}
  \tightlist
  \item
    \((2,3,\infty)\) triangle with nice edges
  \item
    Reflection across vertical geodesics (simple)
  \item
    Reflection across circular geodesics (circle inversion!)
  \item
    Folding algorithm
  \item
    Visualization in both models
  \end{itemize}
\end{itemize}

\subsection*{Homework}\label{homework-2}
\addcontentsline{toc}{subsection}{Homework}

\textbf{Required \#1:} Draw geodesics and hyperbolic disks - Visualize
geodesics in upper half-plane - Draw hyperbolic disks (constant
hyperbolic distance) - Observe metric distortion

\textbf{Required \#2:} Draw triangle edges and vertices - Compute
distance to geodesics - Render triangle boundaries explicitly - Mark
vertices

\textbf{Required \#3:} Model conversion and Möbius transformations -
Convert tiling to Poincaré disk - Apply Möbius transformations
(isometries) - Observe how tiling transforms

\textbf{Optional:} - Different triangle groups (e.g., \((2,3,7)\) for
Escher-like tilings) - Klein model (geodesics become straight lines) -
Decorated tiles (Escher-style patterns)

\begin{center}\rule{0.5\linewidth}{0.5pt}\end{center}

\section*{Day 4: Introduction to 3D
Rendering}\label{day-4-introduction-to-3d-rendering}
\addcontentsline{toc}{section}{Day 4: Introduction to 3D Rendering}

\markright{Day 4: Introduction to 3D Rendering}

\subsection*{Learning Objectives}\label{learning-objectives-3}
\addcontentsline{toc}{subsection}{Learning Objectives}

\begin{itemize}
\tightlist
\item
  Set up camera and generate rays from pixels
\item
  Implement analytical ray-object intersection
\item
  Learn the raymarching algorithm and signed distance functions
\item
  Apply basic lighting (diffuse shading)
\end{itemize}

\subsection*{In-Class Content}\label{in-class-content-3}
\addcontentsline{toc}{subsection}{In-Class Content}

\begin{itemize}
\tightlist
\item
  \textbf{Camera and ray setup}:

  \begin{itemize}
  \tightlist
  \item
    Pinhole camera model
  \item
    Ray generation from pixel coordinates
  \item
    Field of view control
  \end{itemize}
\item
  \textbf{Analytical intersections}:

  \begin{itemize}
  \tightlist
  \item
    Ray-sphere: solve quadratic equation
  \item
    Compute surface normals analytically
  \item
    Ray-torus: implicit equation and gradient
  \item
    Bisection method for root-finding
  \end{itemize}
\item
  \textbf{Lighting introduction}:

  \begin{itemize}
  \tightlist
  \item
    Surface normals
  \item
    Diffuse lighting: dot product with light direction
  \item
    Seeing 3D structure through shading
  \end{itemize}
\item
  \textbf{Motivation for raymarching}:

  \begin{itemize}
  \tightlist
  \item
    Analytical methods don't scale
  \item
    Complex surfaces need flexible approach
  \end{itemize}
\item
  \textbf{Signed Distance Functions (SDFs)}:

  \begin{itemize}
  \tightlist
  \item
    Definition and properties
  \item
    SDFs for primitives: sphere, box, plane, torus
  \item
    Distance as bound for safe marching
  \end{itemize}
\item
  \textbf{Raymarching algorithm}:

  \begin{itemize}
  \tightlist
  \item
    Sphere tracing: march by SDF value
  \item
    Stopping conditions
  \item
    Scene composition (minimum distance)
  \end{itemize}
\item
  \textbf{Normal estimation}:

  \begin{itemize}
  \tightlist
  \item
    Gradient via finite differences
  \item
    Estimating partial derivatives
  \item
    Same lighting applied to raymarched objects
  \end{itemize}
\item
  \textbf{Scene progression}:

  \begin{itemize}
  \tightlist
  \item
    Single sphere
  \item
    Two spheres
  \item
    Sphere and torus
  \end{itemize}
\end{itemize}

\subsection*{Homework}\label{homework-3}
\addcontentsline{toc}{subsection}{Homework}

\textbf{Required:} Algebraic variety rendering - Choose polynomial
implicit surface (degree 3 or 4) - Implement root-finding (bisection or
Newton's method) - Compute gradient for normals - Optional: bounding
sphere optimization

\textbf{Optional:} - Specular lighting (Phong model) - Rotation matrices
for object transformation - Complex multi-object scenes

\begin{center}\rule{0.5\linewidth}{0.5pt}\end{center}

\section*{Day 5: Choose Your
Adventure}\label{day-5-choose-your-adventure}
\addcontentsline{toc}{section}{Day 5: Choose Your Adventure}

\markright{Day 5: Choose Your Adventure}

The final day will be determined based on pacing, student interest, and
energy levels. Two complete lectures are prepared:

\subsection*{Option A: Advanced Raymarching
Techniques}\label{option-a-advanced-raymarching-techniques}
\addcontentsline{toc}{subsection}{Option A: Advanced Raymarching
Techniques}

\subsubsection*{Learning Objectives}\label{learning-objectives-4}
\addcontentsline{toc}{subsubsection}{Learning Objectives}

\begin{itemize}
\tightlist
\item
  Master domain operations for efficient complex scenes
\item
  Understand and apply boolean operations on SDFs
\item
  Create 3D fractals via iterated folding (Menger sponge)
\item
  Build sophisticated scenes from simple primitives
\end{itemize}

\subsubsection*{In-Class Content}\label{in-class-content-4}
\addcontentsline{toc}{subsubsection}{In-Class Content}

\begin{itemize}
\tightlist
\item
  \textbf{Domain operations}:

  \begin{itemize}
  \tightlist
  \item
    Repetition: \texttt{mod()} for infinite object grids
  \item
    Symmetry: \texttt{abs()} for mirror planes
  \item
    Polar repetition for radial patterns
  \item
    Zero computational cost for infinite complexity
  \end{itemize}
\item
  \textbf{Boolean operations on SDFs}:

  \begin{itemize}
  \tightlist
  \item
    Union: \texttt{min(d1,\ d2)}
  \item
    Intersection: \texttt{max(d1,\ d2)}
  \item
    Subtraction: \texttt{max(d1,\ -d2)}
  \item
    Smooth minimum: \texttt{smin()} for organic blending
  \end{itemize}
\item
  \textbf{Menger sponge}:

  \begin{itemize}
  \tightlist
  \item
    Box folding in 3D
  \item
    Axis-aligned operations
  \item
    Iterated subdivision
  \item
    Connection to 2D fractals
  \end{itemize}
\item
  \textbf{Scene building}:

  \begin{itemize}
  \tightlist
  \item
    Combining techniques
  \item
    Architectural structures
  \item
    Infinite repeated patterns
  \end{itemize}
\end{itemize}

\subsubsection*{Homework}\label{homework-4}
\addcontentsline{toc}{subsubsection}{Homework}

\textbf{Required:} Creative scene building - Build complex scene using
domain ops and booleans - Experiment with combinations - Focus on
mathematical or aesthetic interest

\textbf{Optional:} Sierpinski tetrahedron - Implement via 3D folding
(non-axis-aligned) - Connection to Day 2's triangle folding in higher
dimension

\begin{center}\rule{0.5\linewidth}{0.5pt}\end{center}

\subsection*{Option B: Buffers and Temporal
Dynamics}\label{option-b-buffers-and-temporal-dynamics}
\addcontentsline{toc}{subsection}{Option B: Buffers and Temporal
Dynamics}

\subsubsection*{Learning Objectives}\label{learning-objectives-5}
\addcontentsline{toc}{subsubsection}{Learning Objectives}

\begin{itemize}
\tightlist
\item
  Understand buffer-based computation in Shadertoy
\item
  Implement differential operators (Laplacian)
\item
  Solve partial differential equations on the GPU
\item
  Create dynamic, evolving mathematical systems
\end{itemize}

\subsubsection*{In-Class Content}\label{in-class-content-5}
\addcontentsline{toc}{subsubsection}{In-Class Content}

\begin{itemize}
\tightlist
\item
  \textbf{Introduction to buffers}:

  \begin{itemize}
  \tightlist
  \item
    Reading from previous frame
  \item
    Multi-pass rendering
  \item
    Simple example: conditional coloring based on buffer
  \end{itemize}
\item
  \textbf{Edge detection and the Laplacian}:

  \begin{itemize}
  \tightlist
  \item
    Discrete Laplacian stencil (5-point or 9-point)
  \item
    Sampling neighboring pixels
  \item
    Spatial derivatives on grids
  \end{itemize}
\item
  \textbf{The heat equation}:

  \begin{itemize}
  \tightlist
  \item
    Mathematical formulation: \(u_t = \alpha\nabla^2 u\)
  \item
    Applying Laplacian for diffusion
  \item
    Time-stepping:
    \texttt{u\_new\ =\ u\_old\ +\ dt\ *\ α\ *\ laplacian(u\_old)}
  \item
    Initial conditions (e.g., heat in a fractal region)
  \item
    Watching patterns blur and diffuse
  \end{itemize}
\item
  \textbf{Boundary conditions}:

  \begin{itemize}
  \tightlist
  \item
    Zero boundaries (edges set to 0)
  \item
    Avoiding wrap-around
  \end{itemize}
\item
  \textbf{Timestep stability}:

  \begin{itemize}
  \tightlist
  \item
    CFL condition (briefly mentioned)
  \item
    Providing stable \texttt{dt} value
  \end{itemize}
\end{itemize}

\subsubsection*{Homework}\label{homework-5}
\addcontentsline{toc}{subsubsection}{Homework}

\textbf{Required:} Interactive heat equation or reaction-diffusion -
Option 1: Heat source at mouse position, watch it diffuse - Option 2:
Gray-Scott reaction-diffusion (pattern formation)

\textbf{Optional:} Wave equation - Requires two buffers (current and
previous state) - Implement \(u_{tt} = c^2\nabla^2 u\) - Watch waves
propagate and reflect

\begin{center}\rule{0.5\linewidth}{0.5pt}\end{center}

\section*{Resources and Further
Exploration}\label{resources-and-further-exploration}
\addcontentsline{toc}{section}{Resources and Further Exploration}

\markright{Resources and Further Exploration}

\subsection*{Shadertoy}\label{shadertoy}
\addcontentsline{toc}{subsection}{Shadertoy}

\begin{itemize}
\tightlist
\item
  Main site: https://www.shadertoy.com
\item
  Community examples and tutorials
\item
  GLSL documentation
\end{itemize}

\subsection*{References}\label{references}
\addcontentsline{toc}{subsection}{References}

\begin{itemize}
\tightlist
\item
  Complex dynamics and fractals
\item
  Hyperbolic geometry and tilings
\item
  Signed distance functions (Inigo Quilez:
  https://iquilezles.org/articles/distfunctions/)
\item
  GPU computing for scientific visualization
\end{itemize}

\subsection*{Advanced Topics}\label{advanced-topics}
\addcontentsline{toc}{subsection}{Advanced Topics}

\begin{itemize}
\tightlist
\item
  Path tracing and global illumination
\item
  Non-Euclidean ray tracing
\item
  Real-time denoising
\item
  More complex PDEs and simulations
\end{itemize}

\begin{center}\rule{0.5\linewidth}{0.5pt}\end{center}

\section*{Assessment Philosophy}\label{assessment-philosophy}
\addcontentsline{toc}{section}{Assessment Philosophy}

\markright{Assessment Philosophy}

This is a workshop-style course focused on skill development. Success
means: - Completing required homework to keep pace - Experimenting with
optional problems based on interest - Developing intuition for when
shader programming is appropriate - Leaving with working code templates
for future projects

\textbf{Philosophy:} Getting something working and understanding it is
more valuable than perfect, polished results. The goal is to build
practical skills and mathematical intuition, not to create
production-quality graphics.

\begin{center}\rule{0.5\linewidth}{0.5pt}\end{center}

\section*{Schedule Summary}\label{schedule-summary}
\addcontentsline{toc}{section}{Schedule Summary}

\markright{Schedule Summary}

\begin{longtable}[]{@{}
  >{\raggedright\arraybackslash}p{(\linewidth - 4\tabcolsep) * \real{0.2000}}
  >{\raggedright\arraybackslash}p{(\linewidth - 4\tabcolsep) * \real{0.2800}}
  >{\raggedright\arraybackslash}p{(\linewidth - 4\tabcolsep) * \real{0.5200}}@{}}
\toprule\noalign{}
\begin{minipage}[b]{\linewidth}\raggedright
Day
\end{minipage} & \begin{minipage}[b]{\linewidth}\raggedright
Topic
\end{minipage} & \begin{minipage}[b]{\linewidth}\raggedright
Key Concepts
\end{minipage} \\
\midrule\noalign{}
\endhead
\bottomrule\noalign{}
\endlastfoot
1 & Shader Basics & Coordinates, distance fields, implicit curves \\
2 & Complex Dynamics & Mandelbrot, circle inversion, Apollonian
gasket \\
3 & Geometric Tilings & Euclidean and hyperbolic tilings, models \\
4 & 3D Rendering & Raymarching, SDFs, lighting \\
5 & Advanced (flexible) & Domain ops + fractals OR buffers + PDEs \\
\end{longtable}

Each day: 1 hour lecture + \textasciitilde1.5 hours homework Total: 5
lectures, 10-12 programming assignments (5 required, 5-7 optional)

\bookmarksetup{startatroot}

\chapter{Day 1: Introduction to Shader
Programming}\label{day-1-introduction-to-shader-programming-1}

\section{Overview}\label{overview}

By the end of today, you'll be able to create this:

\pandocbounded{\includegraphics[keepaspectratio]{./demos/day1/elliptic-family/screenshot.png}}

A family of elliptic curves \(y^2 = x^3 + ax + b\), drawn for several
values of \(a\) simultaneously, with \(b\) varying across the screen.
The curves shift in brightness to show the family structure, and you can
watch singularities appear and disappear along the discriminant locus.

This image is computed in real time, every pixel evaluated independently
on the GPU. To get here, we'll learn:

\begin{itemize}
\tightlist
\item
  What a shader is: a function from coordinates to colors, evaluated in
  parallel
\item
  How to set up a coordinate system for mathematical visualization
\item
  How to draw shapes using distance functions
\item
  How to render implicit curves \(F(x,y) = 0\) with uniform thickness
\item
  How to add interactivity with mouse input
\end{itemize}

Let's begin.

\section{What is a Shader?}\label{what-is-a-shader}

We want to draw images on a screen.

Mathematically, an image is a function from a region
\(S \subset \mathbb{R}^2\) to the space of visible colors
\(\mathcal{C}\). This color space is three-dimensional, spanned by the
responses of the three types of cone cells in our eyes. A convenient
basis, roughly aligned with these responses, is red, green, and blue.

To realize this on a computer, we discretize. A screen is a grid of
\emph{pixels}: \(X\) pixels wide, \(Y\) pixels tall. Each pixel is a
point in the integer lattice
\[\{0, 1, \ldots, X-1\} \times \{0, 1, \ldots, Y-1\}.\]

Colors are represented as RGB triples: red, green, and blue intensities,
each in \([0,1]\). The constraint to \([0,1]\) reflects physical
reality---a pixel has a maximum brightness it can display. (We can't
draw the sun.) So an image is a function
\[f\colon \{0,\ldots,X-1\} \times \{0,\ldots,Y-1\} \to [0,1]^3\]
\[(i,j) \mapsto (r,g,b).\]

In practice, we add a fourth component: \emph{alpha}, representing
transparency. This matters when compositing multiple layers (we won't
use it in this course, but the machinery expects it). So our shader
computes \[f\colon (i,j) \mapsto (r,g,b,1).\]

This is what a shader is. You write a function that takes pixel
coordinates and returns an RGBA color. The GPU evaluates your function
at every pixel to produce the image.

\subsection{Parallelism}\label{parallelism}

A 1920×1080 display has over two million pixels. How do we evaluate
\(f\) at all of them fast enough to animate at 60 frames per second?

The answer is parallelism. A GPU contains thousands of cores, and it
evaluates \(f\) at all pixels \emph{simultaneously}. There's no loop
over pixels in your code---you write \(f\), and the hardware handles the
rest.

The tradeoff: each pixel's computation must be \emph{independent}. Pixel
\((100, 200)\) cannot ask what color pixel \((100, 199)\) received.
Every pixel sees the same global inputs---coordinates, time, mouse
position---and must determine its color from those alone. Learning to
think within this constraint is what shader programming is about.

\begin{tcolorbox}[enhanced jigsaw, leftrule=.75mm, bottomrule=.15mm, breakable, left=2mm, opacityback=0, colback=white, arc=.35mm, colframe=quarto-callout-note-color-frame, rightrule=.15mm, toprule=.15mm]
\begin{minipage}[t]{5.5mm}
\textcolor{quarto-callout-note-color}{\faInfo}
\end{minipage}%
\begin{minipage}[t]{\textwidth - 5.5mm}

\vspace{-3mm}\textbf{Why ``shader''?}\vspace{3mm}

The name comes from 3D graphics, where these programs computed
\emph{shading}---how light interacts with surfaces. It stuck even though
we now use shaders for fractals, simulations, and mathematical
visualization.

\end{minipage}%
\end{tcolorbox}

\subsection{Why Shadertoy?}\label{why-shadertoy}

Shader programming normally requires substantial setup: OpenGL contexts,
buffer management, compilation, render loops.
\href{https://www.shadertoy.com}{Shadertoy} abstracts all of this---you
write one function, press play, and see results. We'll use it throughout
the course.

\section{First Shaders: Colors and
Syntax}\label{first-shaders-colors-and-syntax}

\subsection{The mainImage Function}\label{the-mainimage-function}

In Shadertoy, your shader is a function called \texttt{mainImage}:

\begin{Shaded}
\begin{Highlighting}[]
\DataTypeTok{void} \FunctionTok{mainImage}\OperatorTok{(}\DataTypeTok{out} \DataTypeTok{vec4}\NormalTok{ fragColor}\OperatorTok{,} \DataTypeTok{in} \DataTypeTok{vec2}\NormalTok{ fragCoord}\OperatorTok{)}
\OperatorTok{\{}
    \CommentTok{// your code here}
\OperatorTok{\}}
\end{Highlighting}
\end{Shaded}

This function is called once per pixel, every frame. The inputs and
outputs:

\begin{itemize}
\tightlist
\item
  \texttt{fragCoord} --- the pixel coordinates, passed \emph{in} to your
  function
\item
  \texttt{fragColor} --- the RGBA color, which you write \emph{out}
\end{itemize}

The \texttt{in} and \texttt{out} keywords are explicit about data flow:
\texttt{fragCoord} is read-only input, \texttt{fragColor} is where you
write your result. The function returns \texttt{void} because the output
goes through \texttt{fragColor}, not a return value.

\subsection{Hello World: A Solid Color}\label{hello-world-a-solid-color}

The simplest shader: make every pixel red.

\pandocbounded{\includegraphics[keepaspectratio]{./demos/day1/red/screenshot.png}}

The \texttt{vec4(1.0,\ 0.0,\ 0.0,\ 1.0)} constructs a 4-component
vector: red=1, green=0, blue=0, alpha=1. Every pixel receives the same
color, so the screen fills with red.

\subsection{GLSL Syntax Essentials}\label{glsl-syntax-essentials}

GLSL (OpenGL Shading Language) will feel familiar if you've seen C-like
syntax, but a few things are worth noting upfront.

\textbf{Semicolons} are required at the end of each statement.

\textbf{Floats must include a decimal point.} Write \texttt{1.0}, not
\texttt{1}. The integer \texttt{1} and the float \texttt{1.0} are
different types, and GLSL is strict about this.

\textbf{Vector types} are built in: \texttt{vec2}, \texttt{vec3},
\texttt{vec4} for 2, 3, and 4 component vectors. Construct them with:

\begin{Shaded}
\begin{Highlighting}[]
\DataTypeTok{vec2}\NormalTok{ p }\OperatorTok{=} \DataTypeTok{vec2}\OperatorTok{(}\FloatTok{3.0}\OperatorTok{,} \FloatTok{4.0}\OperatorTok{);}
\DataTypeTok{vec3}\NormalTok{ color }\OperatorTok{=} \DataTypeTok{vec3}\OperatorTok{(}\FloatTok{1.0}\OperatorTok{,} \FloatTok{0.5}\OperatorTok{,} \FloatTok{0.0}\OperatorTok{);}
\DataTypeTok{vec4}\NormalTok{ rgba }\OperatorTok{=} \DataTypeTok{vec4}\OperatorTok{(}\FloatTok{1.0}\OperatorTok{,} \FloatTok{0.0}\OperatorTok{,} \FloatTok{0.0}\OperatorTok{,} \FloatTok{1.0}\OperatorTok{);}
\end{Highlighting}
\end{Shaded}

\textbf{Arithmetic is component-wise.} Adding two vectors adds their
components:

\begin{Shaded}
\begin{Highlighting}[]
\DataTypeTok{vec2}\OperatorTok{(}\FloatTok{1.0}\OperatorTok{,} \FloatTok{2.0}\OperatorTok{)} \OperatorTok{+} \DataTypeTok{vec2}\OperatorTok{(}\FloatTok{3.0}\OperatorTok{,} \FloatTok{4.0}\OperatorTok{)}  \CommentTok{// = vec2(4.0, 6.0)}
\end{Highlighting}
\end{Shaded}

\textbf{Scalar-vector operations} apply the scalar to each component:

\begin{Shaded}
\begin{Highlighting}[]
\FloatTok{2.0} \OperatorTok{*} \DataTypeTok{vec2}\OperatorTok{(}\FloatTok{1.0}\OperatorTok{,} \FloatTok{3.0}\OperatorTok{)}  \CommentTok{// = vec2(2.0, 6.0)}
\end{Highlighting}
\end{Shaded}

\textbf{Accessing components} uses \texttt{.x}, \texttt{.y},
\texttt{.z}, \texttt{.w}:

\begin{Shaded}
\begin{Highlighting}[]
\DataTypeTok{vec2}\NormalTok{ p }\OperatorTok{=} \DataTypeTok{vec2}\OperatorTok{(}\FloatTok{3.0}\OperatorTok{,} \FloatTok{4.0}\OperatorTok{);}
\DataTypeTok{float}\NormalTok{ a }\OperatorTok{=}\NormalTok{ p}\OperatorTok{.}\FunctionTok{x}\OperatorTok{;}  \CommentTok{// 3.0}
\DataTypeTok{float}\NormalTok{ b }\OperatorTok{=}\NormalTok{ p}\OperatorTok{.}\FunctionTok{y}\OperatorTok{;}  \CommentTok{// 4.0}
\end{Highlighting}
\end{Shaded}

For colors, \texttt{.r}, \texttt{.g}, \texttt{.b}, \texttt{.a} are
synonyms---\texttt{color.r} is the same as \texttt{color.x}.

\textbf{Common math functions} work as expected: \texttt{sin},
\texttt{cos}, \texttt{abs}, \texttt{min}, \texttt{max}, \texttt{sqrt},
\texttt{pow}. These operate on floats, and apply component-wise to
vectors:

\begin{Shaded}
\begin{Highlighting}[]
\BuiltInTok{sin}\OperatorTok{(}\DataTypeTok{vec2}\OperatorTok{(}\FloatTok{0.0}\OperatorTok{,} \FloatTok{3.14159}\OperatorTok{))}  \CommentTok{// = vec2(0.0, \textasciitilde{}0.0)}
\end{Highlighting}
\end{Shaded}

\textbf{For loops} work as you'd expect:

\begin{Shaded}
\begin{Highlighting}[]
\KeywordTok{for} \OperatorTok{(}\DataTypeTok{int}\NormalTok{ i }\OperatorTok{=} \DecValTok{0}\OperatorTok{;}\NormalTok{ i }\OperatorTok{\textless{}} \DecValTok{5}\OperatorTok{;}\NormalTok{ i}\OperatorTok{++)} \OperatorTok{\{}
    \CommentTok{// body executes with i = 0, 1, 2, 3, 4}
\OperatorTok{\}}
\end{Highlighting}
\end{Shaded}

The loop variable is an \texttt{int}. Note that some older GPUs require
the loop bounds to be constants known at compile time---you can't always
loop up to a variable. We'll use loops extensively starting tomorrow.

\subsection{Uniforms: Global Inputs}\label{uniforms-global-inputs}

Shadertoy provides \emph{uniforms}---global values that are constant
across all pixels. Unlike \texttt{fragCoord}, which takes a different
value at each pixel, a uniform has the same value everywhere. They're
how external information (time, screen size, mouse position) gets into
your shader.

\begin{longtable}[]{@{}
  >{\raggedright\arraybackslash}p{(\linewidth - 4\tabcolsep) * \real{0.3214}}
  >{\raggedright\arraybackslash}p{(\linewidth - 4\tabcolsep) * \real{0.2143}}
  >{\raggedright\arraybackslash}p{(\linewidth - 4\tabcolsep) * \real{0.4643}}@{}}
\toprule\noalign{}
\begin{minipage}[b]{\linewidth}\raggedright
Uniform
\end{minipage} & \begin{minipage}[b]{\linewidth}\raggedright
Type
\end{minipage} & \begin{minipage}[b]{\linewidth}\raggedright
Description
\end{minipage} \\
\midrule\noalign{}
\endhead
\bottomrule\noalign{}
\endlastfoot
\texttt{iResolution} & \texttt{vec3} & Viewport size:
\texttt{(width,\ height,\ pixel\_aspect\_ratio)} \\
\texttt{iTime} & \texttt{float} & Seconds since the shader started \\
\texttt{iMouse} & \texttt{vec4} & Mouse position and click state \\
\end{longtable}

We'll use \texttt{iResolution} constantly (for coordinate transforms)
and \texttt{iTime} for animation.

\subsection{Animation: Using iTime}\label{animation-using-itime}

Let's make the red channel pulse:

\pandocbounded{\includegraphics[keepaspectratio]{./demos/day1/red-pulsing/screenshot.png}}

Since \texttt{sin(iTime)} oscillates between -1 and 1, the expression
\texttt{0.5\ +\ 0.5\ *\ sin(iTime)} oscillates between 0 and 1. The
screen pulses from black to red.

This is our first \emph{animated} shader---the output depends on time.

\section{Coordinate Systems}\label{coordinate-systems}

\subsection{Pixel Coordinates}\label{pixel-coordinates}

The input \texttt{fragCoord} gives the pixel coordinates of the current
pixel. The coordinate system:

\begin{itemize}
\tightlist
\item
  Origin at the \textbf{bottom-left} corner
\item
  \texttt{fragCoord.x} increases to the right
\item
  \texttt{fragCoord.y} increases upward
\item
  Ranges from \((0, 0)\) to \((X, Y)\) where \(X \times Y\) is the
  screen resolution
\end{itemize}

This is workable, but inconvenient for mathematics. We'd prefer
coordinates centered at the origin with a reasonable scale. Let's build
up a transformation step by step.

\subsection{\texorpdfstring{Step 1: Normalize to
\([0,1]^2\)}{Step 1: Normalize to {[}0,1{]}\^{}2}}\label{step-1-normalize-to-012}

Divide by the resolution to map pixel coordinates to the unit square:

\begin{Shaded}
\begin{Highlighting}[]
\DataTypeTok{vec2}\NormalTok{ uv }\OperatorTok{=}\NormalTok{ fragCoord }\OperatorTok{/}\NormalTok{ iResolution}\OperatorTok{.}\FunctionTok{xy}\OperatorTok{;}
\end{Highlighting}
\end{Shaded}

Now \texttt{uv} ranges from \((0,0)\) at bottom-left to \((1,1)\) at
top-right.

Since both coordinates are in \([0,1]\), we can visualize them directly
as color:

\begin{Shaded}
\begin{Highlighting}[]
\DataTypeTok{void} \FunctionTok{mainImage}\OperatorTok{(}\DataTypeTok{out} \DataTypeTok{vec4}\NormalTok{ fragColor}\OperatorTok{,} \DataTypeTok{in} \DataTypeTok{vec2}\NormalTok{ fragCoord}\OperatorTok{)}
\OperatorTok{\{}
    \DataTypeTok{vec2}\NormalTok{ uv }\OperatorTok{=}\NormalTok{ fragCoord }\OperatorTok{/}\NormalTok{ iResolution}\OperatorTok{.}\FunctionTok{xy}\OperatorTok{;}
\NormalTok{    fragColor }\OperatorTok{=} \DataTypeTok{vec4}\OperatorTok{(}\NormalTok{uv}\OperatorTok{.}\FunctionTok{x}\OperatorTok{,}\NormalTok{ uv}\OperatorTok{.}\FunctionTok{y}\OperatorTok{,} \FloatTok{0.0}\OperatorTok{,} \FloatTok{1.0}\OperatorTok{);}
\OperatorTok{\}}
\end{Highlighting}
\end{Shaded}

\pandocbounded{\includegraphics[keepaspectratio]{./demos/day1/coordinates/screenshot.png}}

Black at bottom-left (0,0), red at bottom-right (1,0), green at top-left
(0,1), yellow at top-right (1,1).

\subsection{Step 2: Center the Origin}\label{step-2-center-the-origin}

Subtract \((0.5, 0.5)\) to center the origin:

\begin{Shaded}
\begin{Highlighting}[]
\NormalTok{uv }\OperatorTok{=}\NormalTok{ uv }\OperatorTok{{-}} \DataTypeTok{vec2}\OperatorTok{(}\FloatTok{0.5}\OperatorTok{,} \FloatTok{0.5}\OperatorTok{);}
\end{Highlighting}
\end{Shaded}

Now \texttt{uv} ranges from \((-0.5, -0.5)\) to \((0.5, 0.5)\), with
\((0,0)\) at the screen center.

\subsection{Step 3: Aspect Ratio
Correction}\label{step-3-aspect-ratio-correction}

We've mapped a rectangle of pixels (\(X \times Y\)) to the square
\([-0.5, 0.5]^2\). This is an affine transformation, not a
similarity---it distorts shapes. A circle in our coordinates would
render as an ellipse on screen.

To fix this, we scale the \(x\)-coordinate by the aspect ratio:

\begin{Shaded}
\begin{Highlighting}[]
\NormalTok{uv}\OperatorTok{.}\FunctionTok{x} \OperatorTok{*=}\NormalTok{ iResolution}\OperatorTok{.}\FunctionTok{x} \OperatorTok{/}\NormalTok{ iResolution}\OperatorTok{.}\FunctionTok{y}\OperatorTok{;}
\end{Highlighting}
\end{Shaded}

Now a circle in our coordinates appears as a circle on screen. (When we
draw shapes later, try commenting out this line to see the distortion.)

\subsection{Step 4: Scale to a Useful
Range}\label{step-4-scale-to-a-useful-range}

Finally, scale to a convenient window:

\begin{Shaded}
\begin{Highlighting}[]
\DataTypeTok{vec2}\NormalTok{ p }\OperatorTok{=}\NormalTok{ uv }\OperatorTok{*} \FloatTok{4.0}\OperatorTok{;}
\end{Highlighting}
\end{Shaded}

With a scale factor of 4, our coordinates range roughly from \(-2\) to
\(2\)---a good default for visualizing mathematical objects.

\subsection{The Standard Boilerplate}\label{the-standard-boilerplate}

Putting it together, here's the coordinate setup we'll use throughout
the course:

\begin{Shaded}
\begin{Highlighting}[]
\DataTypeTok{vec2}\NormalTok{ uv }\OperatorTok{=}\NormalTok{ fragCoord }\OperatorTok{/}\NormalTok{ iResolution}\OperatorTok{.}\FunctionTok{xy}\OperatorTok{;}   \CommentTok{// normalize to [0,1]}
\NormalTok{uv }\OperatorTok{=}\NormalTok{ uv }\OperatorTok{{-}} \DataTypeTok{vec2}\OperatorTok{(}\FloatTok{0.5}\OperatorTok{,} \FloatTok{0.5}\OperatorTok{);}               \CommentTok{// center origin}
\NormalTok{uv}\OperatorTok{.}\FunctionTok{x} \OperatorTok{*=}\NormalTok{ iResolution}\OperatorTok{.}\FunctionTok{x} \OperatorTok{/}\NormalTok{ iResolution}\OperatorTok{.}\FunctionTok{y}\OperatorTok{;}  \CommentTok{// aspect correction}
\DataTypeTok{vec2}\NormalTok{ p }\OperatorTok{=}\NormalTok{ uv }\OperatorTok{*} \FloatTok{4.0}\OperatorTok{;}                      \CommentTok{// scale}
\end{Highlighting}
\end{Shaded}

From here on, \texttt{p} is our mathematical coordinate, centered at the
origin, aspect-corrected, with a reasonable range.

\section{Drawing with Distance}\label{drawing-with-distance}

So far we've colored every pixel the same, or colored based on position
as a gradient. Now we want to \emph{draw}: to render a shape on screen.

What does it mean to draw a shape? For a simple filled region, we need a
rule that tells us, for each pixel: are you inside the shape or not?
When inside, we do one thing (say, color yellow). When outside, we do
another (color blue). The boundary of the shape is where we switch.

\subsection{Half-Planes}\label{half-planes}

The simplest shape is a half-plane. Consider the rule: is the
\(y\)-coordinate greater than 0? This divides the plane into two
regions---above and below the \(x\)-axis.

\begin{Shaded}
\begin{Highlighting}[]
\DataTypeTok{float}\NormalTok{ L }\OperatorTok{=}\NormalTok{ p}\OperatorTok{.}\FunctionTok{y}\OperatorTok{;}

\DataTypeTok{vec3}\NormalTok{ color}\OperatorTok{;}
\KeywordTok{if} \OperatorTok{(}\NormalTok{L }\OperatorTok{\textless{}} \FloatTok{0.0}\OperatorTok{)} \OperatorTok{\{}
\NormalTok{    color }\OperatorTok{=} \DataTypeTok{vec3}\OperatorTok{(}\FloatTok{1.0}\OperatorTok{,} \FloatTok{0.0}\OperatorTok{,} \FloatTok{0.0}\OperatorTok{);}  \CommentTok{// red below}
\OperatorTok{\}} \KeywordTok{else} \OperatorTok{\{}
\NormalTok{    color }\OperatorTok{=} \DataTypeTok{vec3}\OperatorTok{(}\FloatTok{0.0}\OperatorTok{,} \FloatTok{0.0}\OperatorTok{,} \FloatTok{1.0}\OperatorTok{);}  \CommentTok{// blue above}
\OperatorTok{\}}

\NormalTok{fragColor }\OperatorTok{=} \DataTypeTok{vec4}\OperatorTok{(}\NormalTok{color}\OperatorTok{,} \FloatTok{1.0}\OperatorTok{);}
\end{Highlighting}
\end{Shaded}

\pandocbounded{\includegraphics[keepaspectratio]{./demos/day1/half-plane/screenshot.png}}

To color left versus right instead, use \texttt{p.x} in place of
\texttt{p.y}.

More generally, a line in the plane has the form \(ax + by + c = 0\).
This divides the plane into two half-planes: where \(ax + by + c < 0\)
and where \(ax + by + c > 0\).

\begin{Shaded}
\begin{Highlighting}[]
\DataTypeTok{float}\NormalTok{ a }\OperatorTok{=} \FloatTok{1.0}\OperatorTok{,}\NormalTok{ b }\OperatorTok{=} \FloatTok{1.0}\OperatorTok{,}\NormalTok{ c }\OperatorTok{=} \FloatTok{0.0}\OperatorTok{;}
\DataTypeTok{float}\NormalTok{ L }\OperatorTok{=}\NormalTok{ a }\OperatorTok{*}\NormalTok{ p}\OperatorTok{.}\FunctionTok{x} \OperatorTok{+}\NormalTok{ b }\OperatorTok{*}\NormalTok{ p}\OperatorTok{.}\FunctionTok{y} \OperatorTok{+}\NormalTok{ c}\OperatorTok{;}

\DataTypeTok{vec3}\NormalTok{ color}\OperatorTok{;}
\KeywordTok{if} \OperatorTok{(}\NormalTok{L }\OperatorTok{\textless{}} \FloatTok{0.0}\OperatorTok{)} \OperatorTok{\{}
\NormalTok{    color }\OperatorTok{=} \DataTypeTok{vec3}\OperatorTok{(}\FloatTok{1.0}\OperatorTok{,} \FloatTok{0.0}\OperatorTok{,} \FloatTok{0.0}\OperatorTok{);}  \CommentTok{// red}
\OperatorTok{\}} \KeywordTok{else} \OperatorTok{\{}
\NormalTok{    color }\OperatorTok{=} \DataTypeTok{vec3}\OperatorTok{(}\FloatTok{0.0}\OperatorTok{,} \FloatTok{0.0}\OperatorTok{,} \FloatTok{1.0}\OperatorTok{);}  \CommentTok{// blue}
\OperatorTok{\}}

\NormalTok{fragColor }\OperatorTok{=} \DataTypeTok{vec4}\OperatorTok{(}\NormalTok{color}\OperatorTok{,} \FloatTok{1.0}\OperatorTok{);}
\end{Highlighting}
\end{Shaded}

Recall that \((a, b)\) is the normal vector to the line, and \(c\) is an
offset. Since these are just variables, we can animate them to move the
line around:

\begin{Shaded}
\begin{Highlighting}[]
\DataTypeTok{float}\NormalTok{ a }\OperatorTok{=} \BuiltInTok{cos}\OperatorTok{(}\NormalTok{iTime}\OperatorTok{);}
\DataTypeTok{float}\NormalTok{ b }\OperatorTok{=} \BuiltInTok{sin}\OperatorTok{(}\NormalTok{iTime}\OperatorTok{);}
\DataTypeTok{float}\NormalTok{ c }\OperatorTok{=} \FloatTok{0.5} \OperatorTok{*} \BuiltInTok{sin}\OperatorTok{(}\NormalTok{iTime }\OperatorTok{*} \FloatTok{0.7}\OperatorTok{);}
\end{Highlighting}
\end{Shaded}

\pandocbounded{\includegraphics[keepaspectratio]{./demos/day1/half-plane-animated/screenshot.png}}

\subsection{Circles}\label{circles}

Now consider the function \(d(p) = |p|\), the distance from the origin.
Geometrically, the graph of this function is a cone---zero at the
origin, increasing linearly in all directions.

To draw a filled disk of radius \(r\), we could threshold on \(d < r\)
versus \(d \geq r\). But it's cleaner to define \(f(p) = |p| - r\). This
function is negative inside the circle (where \(d < r\)) and positive
outside (where \(d > r\)). The circle itself is the level set \(f = 0\).

\begin{Shaded}
\begin{Highlighting}[]
\DataTypeTok{float}\NormalTok{ d }\OperatorTok{=} \BuiltInTok{length}\OperatorTok{(}\NormalTok{p}\OperatorTok{);}
\DataTypeTok{float}\NormalTok{ r }\OperatorTok{=} \FloatTok{1.0}\OperatorTok{;}
\DataTypeTok{float}\NormalTok{ f }\OperatorTok{=}\NormalTok{ d }\OperatorTok{{-}}\NormalTok{ r}\OperatorTok{;}

\DataTypeTok{vec3}\NormalTok{ color}\OperatorTok{;}
\KeywordTok{if} \OperatorTok{(}\NormalTok{f }\OperatorTok{\textless{}} \FloatTok{0.0}\OperatorTok{)} \OperatorTok{\{}
\NormalTok{    color }\OperatorTok{=} \DataTypeTok{vec3}\OperatorTok{(}\FloatTok{1.0}\OperatorTok{,} \FloatTok{1.0}\OperatorTok{,} \FloatTok{0.0}\OperatorTok{);}  \CommentTok{// yellow inside}
\OperatorTok{\}} \KeywordTok{else} \OperatorTok{\{}
\NormalTok{    color }\OperatorTok{=} \DataTypeTok{vec3}\OperatorTok{(}\FloatTok{0.1}\OperatorTok{,} \FloatTok{0.1}\OperatorTok{,} \FloatTok{0.3}\OperatorTok{);}  \CommentTok{// dark blue outside}
\OperatorTok{\}}

\NormalTok{fragColor }\OperatorTok{=} \DataTypeTok{vec4}\OperatorTok{(}\NormalTok{color}\OperatorTok{,} \FloatTok{1.0}\OperatorTok{);}
\end{Highlighting}
\end{Shaded}

\pandocbounded{\includegraphics[keepaspectratio]{./demos/day1/circle/screenshot.png}}

Try commenting out the aspect ratio correction (\texttt{uv.x\ *=\ ...})
to see the distortion---the circle becomes an ellipse.

To center the circle at a point \(c\) instead of the origin, compute
distance from \(c\):

\begin{Shaded}
\begin{Highlighting}[]
\DataTypeTok{vec2}\NormalTok{ center }\OperatorTok{=} \DataTypeTok{vec2}\OperatorTok{(}\FloatTok{1.0}\OperatorTok{,} \FloatTok{0.5}\OperatorTok{);}
\DataTypeTok{float}\NormalTok{ d }\OperatorTok{=} \BuiltInTok{length}\OperatorTok{(}\NormalTok{p }\OperatorTok{{-}}\NormalTok{ center}\OperatorTok{);}
\end{Highlighting}
\end{Shaded}

Since \texttt{center} and \texttt{r} are variables, you can animate them
with \texttt{iTime} to create moving, pulsing circles.

\subsection{Drawing a Ring}\label{drawing-a-ring}

Our function \(f = d - r\) is negative inside the circle and positive
outside. To draw a filled disk, we colored based on the sign of \(f\).

But what if we want just the boundary---a ring of some thickness? We
want to color one way when \(f\) is small in absolute value (near the
circle), and a different way when \(|f|\) is large (far from the
circle).

So we look at \(|f| = |d - r|\) and ask: is this less than some
threshold \(\varepsilon\), or greater? Equivalently, is
\(|d - r| - \varepsilon\) negative or positive?

\begin{Shaded}
\begin{Highlighting}[]
\DataTypeTok{float}\NormalTok{ d }\OperatorTok{=} \BuiltInTok{length}\OperatorTok{(}\NormalTok{p}\OperatorTok{);}
\DataTypeTok{float}\NormalTok{ r }\OperatorTok{=} \FloatTok{1.0}\OperatorTok{;}
\DataTypeTok{float}\NormalTok{ eps }\OperatorTok{=} \FloatTok{0.1}\OperatorTok{;}
\DataTypeTok{float}\NormalTok{ f }\OperatorTok{=} \BuiltInTok{abs}\OperatorTok{(}\NormalTok{d }\OperatorTok{{-}}\NormalTok{ r}\OperatorTok{)} \OperatorTok{{-}}\NormalTok{ eps}\OperatorTok{;}

\DataTypeTok{vec3}\NormalTok{ color}\OperatorTok{;}
\KeywordTok{if} \OperatorTok{(}\NormalTok{f }\OperatorTok{\textless{}} \FloatTok{0.0}\OperatorTok{)} \OperatorTok{\{}
\NormalTok{    color }\OperatorTok{=} \DataTypeTok{vec3}\OperatorTok{(}\FloatTok{1.0}\OperatorTok{,} \FloatTok{1.0}\OperatorTok{,} \FloatTok{1.0}\OperatorTok{);}  \CommentTok{// white ring}
\OperatorTok{\}} \KeywordTok{else} \OperatorTok{\{}
\NormalTok{    color }\OperatorTok{=} \DataTypeTok{vec3}\OperatorTok{(}\FloatTok{0.1}\OperatorTok{,} \FloatTok{0.1}\OperatorTok{,} \FloatTok{0.3}\OperatorTok{);}  \CommentTok{// dark background}
\OperatorTok{\}}

\NormalTok{fragColor }\OperatorTok{=} \DataTypeTok{vec4}\OperatorTok{(}\NormalTok{color}\OperatorTok{,} \FloatTok{1.0}\OperatorTok{);}
\end{Highlighting}
\end{Shaded}

\pandocbounded{\includegraphics[keepaspectratio]{./demos/day1/circle-ring/screenshot.png}}

\section{Implicit Curves}\label{implicit-curves}

We've drawn circles using the distance function \(|p| - r\). But circles
are just one example of curves defined by an equation. Any equation
\(F(x,y) = 0\) defines a curve---the set of points satisfying that
equation. We can draw it the same way: threshold on \(|F|\).

\subsection{A First Example: The
Parabola}\label{a-first-example-the-parabola}

Consider \(F(x,y) = y - x^2\). The curve \(F = 0\) is the parabola
\(y = x^2\). Points where \(F < 0\) lie below the parabola; points where
\(F > 0\) lie above.

To draw the curve itself, we color pixels where \(|F|\) is small:

\begin{Shaded}
\begin{Highlighting}[]
\DataTypeTok{float}\NormalTok{ F }\OperatorTok{=}\NormalTok{ p}\OperatorTok{.}\FunctionTok{y} \OperatorTok{{-}}\NormalTok{ p}\OperatorTok{.}\FunctionTok{x} \OperatorTok{*}\NormalTok{ p}\OperatorTok{.}\FunctionTok{x}\OperatorTok{;}
\DataTypeTok{float}\NormalTok{ eps }\OperatorTok{=} \FloatTok{0.1}\OperatorTok{;}

\DataTypeTok{vec3}\NormalTok{ color}\OperatorTok{;}
\KeywordTok{if} \OperatorTok{(}\BuiltInTok{abs}\OperatorTok{(}\NormalTok{F}\OperatorTok{)} \OperatorTok{\textless{}}\NormalTok{ eps}\OperatorTok{)} \OperatorTok{\{}
\NormalTok{    color }\OperatorTok{=} \DataTypeTok{vec3}\OperatorTok{(}\FloatTok{1.0}\OperatorTok{,} \FloatTok{1.0}\OperatorTok{,} \FloatTok{0.0}\OperatorTok{);}  \CommentTok{// yellow curve}
\OperatorTok{\}} \KeywordTok{else} \OperatorTok{\{}
\NormalTok{    color }\OperatorTok{=} \DataTypeTok{vec3}\OperatorTok{(}\FloatTok{0.1}\OperatorTok{,} \FloatTok{0.1}\OperatorTok{,} \FloatTok{0.3}\OperatorTok{);}  \CommentTok{// dark background}
\OperatorTok{\}}

\NormalTok{fragColor }\OperatorTok{=} \DataTypeTok{vec4}\OperatorTok{(}\NormalTok{color}\OperatorTok{,} \FloatTok{1.0}\OperatorTok{);}
\end{Highlighting}
\end{Shaded}

\pandocbounded{\includegraphics[keepaspectratio]{./demos/day1/parabola/screenshot.png}}

\subsection{More Examples}\label{more-examples}

An ellipse: \(F(x,y) = \frac{x^2}{a^2} + \frac{y^2}{b^2} - 1\)

\begin{Shaded}
\begin{Highlighting}[]
\DataTypeTok{float}\NormalTok{ a }\OperatorTok{=} \FloatTok{2.0}\OperatorTok{,}\NormalTok{ b }\OperatorTok{=} \FloatTok{1.0}\OperatorTok{;}
\DataTypeTok{float}\NormalTok{ F }\OperatorTok{=} \OperatorTok{(}\NormalTok{p}\OperatorTok{.}\FunctionTok{x}\OperatorTok{*}\NormalTok{p}\OperatorTok{.}\FunctionTok{x}\OperatorTok{)/(}\NormalTok{a}\OperatorTok{*}\NormalTok{a}\OperatorTok{)} \OperatorTok{+} \OperatorTok{(}\NormalTok{p}\OperatorTok{.}\FunctionTok{y}\OperatorTok{*}\NormalTok{p}\OperatorTok{.}\FunctionTok{y}\OperatorTok{)/(}\NormalTok{b}\OperatorTok{*}\NormalTok{b}\OperatorTok{)} \OperatorTok{{-}} \FloatTok{1.0}\OperatorTok{;}
\end{Highlighting}
\end{Shaded}

A hyperbola: \(F(x,y) = \frac{x^2}{a^2} - \frac{y^2}{b^2} - 1\)

\begin{Shaded}
\begin{Highlighting}[]
\DataTypeTok{float}\NormalTok{ a }\OperatorTok{=} \FloatTok{1.0}\OperatorTok{,}\NormalTok{ b }\OperatorTok{=} \FloatTok{1.0}\OperatorTok{;}
\DataTypeTok{float}\NormalTok{ F }\OperatorTok{=} \OperatorTok{(}\NormalTok{p}\OperatorTok{.}\FunctionTok{x}\OperatorTok{*}\NormalTok{p}\OperatorTok{.}\FunctionTok{x}\OperatorTok{)/(}\NormalTok{a}\OperatorTok{*}\NormalTok{a}\OperatorTok{)} \OperatorTok{{-}} \OperatorTok{(}\NormalTok{p}\OperatorTok{.}\FunctionTok{y}\OperatorTok{*}\NormalTok{p}\OperatorTok{.}\FunctionTok{y}\OperatorTok{)/(}\NormalTok{b}\OperatorTok{*}\NormalTok{b}\OperatorTok{)} \OperatorTok{{-}} \FloatTok{1.0}\OperatorTok{;}
\end{Highlighting}
\end{Shaded}

The lemniscate of Bernoulli: \((x^2 + y^2)^2 = a^2(x^2 - y^2)\), or
\(F = (x^2+y^2)^2 - a^2(x^2 - y^2)\)

\begin{Shaded}
\begin{Highlighting}[]
\DataTypeTok{float}\NormalTok{ a }\OperatorTok{=} \FloatTok{1.5}\OperatorTok{;}
\DataTypeTok{float}\NormalTok{ r2 }\OperatorTok{=} \BuiltInTok{dot}\OperatorTok{(}\NormalTok{p}\OperatorTok{,}\NormalTok{ p}\OperatorTok{);}  \CommentTok{// x² + y²}
\DataTypeTok{float}\NormalTok{ F }\OperatorTok{=}\NormalTok{ r2 }\OperatorTok{*}\NormalTok{ r2 }\OperatorTok{{-}}\NormalTok{ a }\OperatorTok{*}\NormalTok{ a }\OperatorTok{*} \OperatorTok{(}\NormalTok{p}\OperatorTok{.}\FunctionTok{x} \OperatorTok{*}\NormalTok{ p}\OperatorTok{.}\FunctionTok{x} \OperatorTok{{-}}\NormalTok{ p}\OperatorTok{.}\FunctionTok{y} \OperatorTok{*}\NormalTok{ p}\OperatorTok{.}\FunctionTok{y}\OperatorTok{);}
\end{Highlighting}
\end{Shaded}

\subsection{The Thickness Problem}\label{the-thickness-problem}

Look carefully at the parabola. The rendered thickness isn't
uniform---it's thinner where the curve is steep, thicker where it's
flat. The problem gets worse with more complicated curves, especially
those with singularities. Here's the lemniscate:

\pandocbounded{\includegraphics[keepaspectratio]{./demos/day1/lemniscate-naive/screenshot.png}}

Notice how the thickness blows up near the origin, where the curve
crosses itself.

Why does this happen? The set \(|F| < \varepsilon\) contains all points
within \(\varepsilon\) of zero \emph{in the \(F\) direction}. But \(F\)
doesn't measure distance to the curve---it's just some function that
happens to be zero on the curve. Where \(|\nabla F|\) is large, \(F\)
changes rapidly, so the band \(|F| < \varepsilon\) is narrow. Where
\(|\nabla F|\) is small, \(F\) changes slowly, so the band is wide. At
the singular point, \(\nabla F = 0\), and the band becomes infinitely
wide.

\subsection{Why Circles Worked}\label{why-circles-worked}

For the circle, we used \(f(p) = |p| - r\). This is the \emph{signed
distance function}: it measures actual geometric distance to the curve.
The gradient of a distance function has magnitude 1 everywhere (it
points toward or away from the curve at unit rate). So
\(|f| < \varepsilon\) really does capture points within distance
\(\varepsilon\), giving uniform thickness.

This is a fact from differential geometry: \(|\nabla d| = 1\) for a
distance function \(d\). When we use an arbitrary implicit equation
\(F = 0\), we lose this property.

\subsection{Gradient Correction}\label{gradient-correction}

We can fix the non-uniform thickness by dividing by the gradient
magnitude. Instead of thresholding \(|F| < \varepsilon\), we threshold
\[\frac{|F|}{|\nabla F|} < \varepsilon.\]

This approximates the signed distance to the curve. The intuition:
\(|F|/|\nabla F|\) estimates how far you'd need to travel (in the
direction \(F\) changes fastest) to reach the curve.

For the lemniscate, we compute the gradient analytically:
\[\nabla F = \bigl(4x(x^2+y^2) - 2a^2 x,\; 4y(x^2+y^2) + 2a^2 y\bigr)\]

\begin{Shaded}
\begin{Highlighting}[]
\DataTypeTok{float}\NormalTok{ a }\OperatorTok{=} \FloatTok{1.5}\OperatorTok{;}
\DataTypeTok{float}\NormalTok{ r2 }\OperatorTok{=} \BuiltInTok{dot}\OperatorTok{(}\NormalTok{p}\OperatorTok{,}\NormalTok{ p}\OperatorTok{);}
\DataTypeTok{float}\NormalTok{ F }\OperatorTok{=}\NormalTok{ r2 }\OperatorTok{*}\NormalTok{ r2 }\OperatorTok{{-}}\NormalTok{ a }\OperatorTok{*}\NormalTok{ a }\OperatorTok{*} \OperatorTok{(}\NormalTok{p}\OperatorTok{.}\FunctionTok{x} \OperatorTok{*}\NormalTok{ p}\OperatorTok{.}\FunctionTok{x} \OperatorTok{{-}}\NormalTok{ p}\OperatorTok{.}\FunctionTok{y} \OperatorTok{*}\NormalTok{ p}\OperatorTok{.}\FunctionTok{y}\OperatorTok{);}

\DataTypeTok{vec2}\NormalTok{ grad }\OperatorTok{=} \DataTypeTok{vec2}\OperatorTok{(}
    \FloatTok{4.0} \OperatorTok{*}\NormalTok{ p}\OperatorTok{.}\FunctionTok{x} \OperatorTok{*}\NormalTok{ r2 }\OperatorTok{{-}} \FloatTok{2.0} \OperatorTok{*}\NormalTok{ a }\OperatorTok{*}\NormalTok{ a }\OperatorTok{*}\NormalTok{ p}\OperatorTok{.}\FunctionTok{x}\OperatorTok{,}
    \FloatTok{4.0} \OperatorTok{*}\NormalTok{ p}\OperatorTok{.}\FunctionTok{y} \OperatorTok{*}\NormalTok{ r2 }\OperatorTok{+} \FloatTok{2.0} \OperatorTok{*}\NormalTok{ a }\OperatorTok{*}\NormalTok{ a }\OperatorTok{*}\NormalTok{ p}\OperatorTok{.}\FunctionTok{y}
\OperatorTok{);}

\DataTypeTok{float}\NormalTok{ dist }\OperatorTok{=} \BuiltInTok{abs}\OperatorTok{(}\NormalTok{F}\OperatorTok{)} \OperatorTok{/} \BuiltInTok{max}\OperatorTok{(}\BuiltInTok{length}\OperatorTok{(}\NormalTok{grad}\OperatorTok{),} \FloatTok{0.01}\OperatorTok{);}  \CommentTok{// avoid division by zero}
\DataTypeTok{float}\NormalTok{ eps }\OperatorTok{=} \FloatTok{0.05}\OperatorTok{;}

\DataTypeTok{vec3}\NormalTok{ color}\OperatorTok{;}
\KeywordTok{if} \OperatorTok{(}\NormalTok{dist }\OperatorTok{\textless{}}\NormalTok{ eps}\OperatorTok{)} \OperatorTok{\{}
\NormalTok{    color }\OperatorTok{=} \DataTypeTok{vec3}\OperatorTok{(}\FloatTok{1.0}\OperatorTok{,} \FloatTok{1.0}\OperatorTok{,} \FloatTok{0.0}\OperatorTok{);}
\OperatorTok{\}} \KeywordTok{else} \OperatorTok{\{}
\NormalTok{    color }\OperatorTok{=} \DataTypeTok{vec3}\OperatorTok{(}\FloatTok{0.1}\OperatorTok{,} \FloatTok{0.1}\OperatorTok{,} \FloatTok{0.3}\OperatorTok{);}
\OperatorTok{\}}

\NormalTok{fragColor }\OperatorTok{=} \DataTypeTok{vec4}\OperatorTok{(}\NormalTok{color}\OperatorTok{,} \FloatTok{1.0}\OperatorTok{);}
\end{Highlighting}
\end{Shaded}

\pandocbounded{\includegraphics[keepaspectratio]{./demos/day1/lemniscate-gradient/screenshot.png}}

Compare with the naive version above to see the difference in thickness
uniformity.

\subsection{Animated Curve Families}\label{animated-curve-families}

The lemniscate is part of a one-parameter family called the Cassini
ovals, defined by the product of distances from two foci being constant:
\[(x^2 + y^2)^2 - 2c^2(x^2 - y^2) = a^4 - c^4\]

As the parameter \(a\) varies relative to the fixed focal distance
\(c\), the topology changes: two separate loops when \(a < c\), a
lemniscate when \(a = c\), a single oval when \(a > c\).

\pandocbounded{\includegraphics[keepaspectratio]{./demos/day1/lemniscate-animated/screenshot.png}}

\section{Interactivity and
Abstraction}\label{interactivity-and-abstraction}

So far our shaders respond to time (\texttt{iTime}) but not to user
input. Shadertoy provides \texttt{iMouse} for mouse interaction.

\subsection{The iMouse Uniform}\label{the-imouse-uniform}

\texttt{iMouse} is a \texttt{vec4}:

\begin{itemize}
\tightlist
\item
  \texttt{iMouse.xy} --- current mouse position (in pixels)
\item
  \texttt{iMouse.zw} --- position where the mouse was last clicked
\end{itemize}

For now we'll focus on \texttt{iMouse.xy}.

\subsection{Dragging a Circle}\label{dragging-a-circle}

Let's draw a circle centered at the mouse position. Since
\texttt{iMouse.xy} is in pixel coordinates, we need to normalize it the
same way we normalize \texttt{fragCoord}:

\begin{Shaded}
\begin{Highlighting}[]
\CommentTok{// Normalize fragment coordinate}
\DataTypeTok{vec2}\NormalTok{ uv }\OperatorTok{=}\NormalTok{ fragCoord }\OperatorTok{/}\NormalTok{ iResolution}\OperatorTok{.}\FunctionTok{xy}\OperatorTok{;}
\NormalTok{uv }\OperatorTok{=}\NormalTok{ uv }\OperatorTok{{-}} \DataTypeTok{vec2}\OperatorTok{(}\FloatTok{0.5}\OperatorTok{,} \FloatTok{0.5}\OperatorTok{);}
\NormalTok{uv}\OperatorTok{.}\FunctionTok{x} \OperatorTok{*=}\NormalTok{ iResolution}\OperatorTok{.}\FunctionTok{x} \OperatorTok{/}\NormalTok{ iResolution}\OperatorTok{.}\FunctionTok{y}\OperatorTok{;}
\DataTypeTok{vec2}\NormalTok{ p }\OperatorTok{=}\NormalTok{ uv }\OperatorTok{*} \FloatTok{4.0}\OperatorTok{;}

\CommentTok{// Normalize mouse coordinate the same way}
\DataTypeTok{vec2}\NormalTok{ mouse }\OperatorTok{=}\NormalTok{ iMouse}\OperatorTok{.}\FunctionTok{xy} \OperatorTok{/}\NormalTok{ iResolution}\OperatorTok{.}\FunctionTok{xy}\OperatorTok{;}
\NormalTok{mouse }\OperatorTok{=}\NormalTok{ mouse }\OperatorTok{{-}} \DataTypeTok{vec2}\OperatorTok{(}\FloatTok{0.5}\OperatorTok{,} \FloatTok{0.5}\OperatorTok{);}
\NormalTok{mouse}\OperatorTok{.}\FunctionTok{x} \OperatorTok{*=}\NormalTok{ iResolution}\OperatorTok{.}\FunctionTok{x} \OperatorTok{/}\NormalTok{ iResolution}\OperatorTok{.}\FunctionTok{y}\OperatorTok{;}
\NormalTok{mouse }\OperatorTok{=}\NormalTok{ mouse }\OperatorTok{*} \FloatTok{4.0}\OperatorTok{;}

\CommentTok{// Circle centered at mouse}
\DataTypeTok{float}\NormalTok{ d }\OperatorTok{=} \BuiltInTok{length}\OperatorTok{(}\NormalTok{p }\OperatorTok{{-}}\NormalTok{ mouse}\OperatorTok{);}
\DataTypeTok{float}\NormalTok{ r }\OperatorTok{=} \FloatTok{0.5}\OperatorTok{;}

\DataTypeTok{vec3}\NormalTok{ color}\OperatorTok{;}
\KeywordTok{if} \OperatorTok{(}\NormalTok{d }\OperatorTok{\textless{}}\NormalTok{ r}\OperatorTok{)} \OperatorTok{\{}
\NormalTok{    color }\OperatorTok{=} \DataTypeTok{vec3}\OperatorTok{(}\FloatTok{1.0}\OperatorTok{,} \FloatTok{0.9}\OperatorTok{,} \FloatTok{0.2}\OperatorTok{);}  \CommentTok{// yellow}
\OperatorTok{\}} \KeywordTok{else} \OperatorTok{\{}
\NormalTok{    color }\OperatorTok{=} \DataTypeTok{vec3}\OperatorTok{(}\FloatTok{0.1}\OperatorTok{,} \FloatTok{0.1}\OperatorTok{,} \FloatTok{0.3}\OperatorTok{);}
\OperatorTok{\}}

\NormalTok{fragColor }\OperatorTok{=} \DataTypeTok{vec4}\OperatorTok{(}\NormalTok{color}\OperatorTok{,} \FloatTok{1.0}\OperatorTok{);}
\end{Highlighting}
\end{Shaded}

\pandocbounded{\includegraphics[keepaspectratio]{./demos/day1/circle-mouse/screenshot.png}}

Click and drag to move the circle.

\subsection{Writing a Helper Function}\label{writing-a-helper-function}

We just wrote the same four lines of coordinate normalization twice.
This is a sign we should write a function.

A GLSL function declares its return type, then the function name, then
its parameters with their types:

\begin{Shaded}
\begin{Highlighting}[]
\DataTypeTok{vec2} \FunctionTok{normalize\_coord}\OperatorTok{(}\DataTypeTok{vec2}\NormalTok{ coord}\OperatorTok{)} \OperatorTok{\{}
    \DataTypeTok{vec2}\NormalTok{ uv }\OperatorTok{=}\NormalTok{ coord }\OperatorTok{/}\NormalTok{ iResolution}\OperatorTok{.}\FunctionTok{xy}\OperatorTok{;}
\NormalTok{    uv }\OperatorTok{=}\NormalTok{ uv }\OperatorTok{{-}} \DataTypeTok{vec2}\OperatorTok{(}\FloatTok{0.5}\OperatorTok{,} \FloatTok{0.5}\OperatorTok{);}
\NormalTok{    uv}\OperatorTok{.}\FunctionTok{x} \OperatorTok{*=}\NormalTok{ iResolution}\OperatorTok{.}\FunctionTok{x} \OperatorTok{/}\NormalTok{ iResolution}\OperatorTok{.}\FunctionTok{y}\OperatorTok{;}
    \KeywordTok{return}\NormalTok{ uv }\OperatorTok{*} \FloatTok{4.0}\OperatorTok{;}
\OperatorTok{\}}
\end{Highlighting}
\end{Shaded}

Functions must be defined before they're used, so they go above
\texttt{mainImage}. Here's the overall structure:

\begin{Shaded}
\begin{Highlighting}[]
\DataTypeTok{vec2} \FunctionTok{normalize\_coord}\OperatorTok{(}\DataTypeTok{vec2}\NormalTok{ coord}\OperatorTok{)} \OperatorTok{\{}
    \CommentTok{// normalization logic here}
\OperatorTok{\}}

\DataTypeTok{void} \FunctionTok{mainImage}\OperatorTok{(}\DataTypeTok{out} \DataTypeTok{vec4}\NormalTok{ fragColor}\OperatorTok{,} \DataTypeTok{in} \DataTypeTok{vec2}\NormalTok{ fragCoord}\OperatorTok{)} \OperatorTok{\{}
    \DataTypeTok{vec2}\NormalTok{ p }\OperatorTok{=} \FunctionTok{normalize\_coord}\OperatorTok{(}\NormalTok{fragCoord}\OperatorTok{);}
    \DataTypeTok{vec2}\NormalTok{ mouse }\OperatorTok{=} \FunctionTok{normalize\_coord}\OperatorTok{(}\NormalTok{iMouse}\OperatorTok{.}\FunctionTok{xy}\OperatorTok{);}
    
    \CommentTok{// code using p and mouse}
\OperatorTok{\}}
\end{Highlighting}
\end{Shaded}

Now our shader is cleaner, and we won't make mistakes copying the
normalization code.

\subsection{Combining iMouse and iTime: Sun and
Earth}\label{combining-imouse-and-itime-sun-and-earth}

Let's make a circle orbit around the mouse position:

\begin{Shaded}
\begin{Highlighting}[]
\DataTypeTok{vec2}\NormalTok{ p }\OperatorTok{=} \FunctionTok{normalize\_coord}\OperatorTok{(}\NormalTok{fragCoord}\OperatorTok{);}
\DataTypeTok{vec2}\NormalTok{ sun }\OperatorTok{=} \FunctionTok{normalize\_coord}\OperatorTok{(}\NormalTok{iMouse}\OperatorTok{.}\FunctionTok{xy}\OperatorTok{);}

\CommentTok{// Earth orbits the sun}
\DataTypeTok{float}\NormalTok{ orbit\_radius }\OperatorTok{=} \FloatTok{0.8}\OperatorTok{;}
\DataTypeTok{vec2}\NormalTok{ earth }\OperatorTok{=}\NormalTok{ sun }\OperatorTok{+}\NormalTok{ orbit\_radius }\OperatorTok{*} \DataTypeTok{vec2}\OperatorTok{(}\BuiltInTok{cos}\OperatorTok{(}\NormalTok{iTime}\OperatorTok{),} \BuiltInTok{sin}\OperatorTok{(}\NormalTok{iTime}\OperatorTok{));}

\CommentTok{// Draw sun (larger, yellow)}
\DataTypeTok{float}\NormalTok{ d\_sun }\OperatorTok{=} \BuiltInTok{length}\OperatorTok{(}\NormalTok{p }\OperatorTok{{-}}\NormalTok{ sun}\OperatorTok{);}
\CommentTok{// Draw earth (smaller, blue)}
\DataTypeTok{float}\NormalTok{ d\_earth }\OperatorTok{=} \BuiltInTok{length}\OperatorTok{(}\NormalTok{p }\OperatorTok{{-}}\NormalTok{ earth}\OperatorTok{);}

\DataTypeTok{vec3}\NormalTok{ color }\OperatorTok{=} \DataTypeTok{vec3}\OperatorTok{(}\FloatTok{0.02}\OperatorTok{,} \FloatTok{0.02}\OperatorTok{,} \FloatTok{0.05}\OperatorTok{);}  \CommentTok{// dark background}
\KeywordTok{if} \OperatorTok{(}\NormalTok{d\_sun }\OperatorTok{\textless{}} \FloatTok{0.3}\OperatorTok{)} \OperatorTok{\{}
\NormalTok{    color }\OperatorTok{=} \DataTypeTok{vec3}\OperatorTok{(}\FloatTok{1.0}\OperatorTok{,} \FloatTok{0.9}\OperatorTok{,} \FloatTok{0.2}\OperatorTok{);}  \CommentTok{// yellow sun}
\OperatorTok{\}}
\KeywordTok{if} \OperatorTok{(}\NormalTok{d\_earth }\OperatorTok{\textless{}} \FloatTok{0.15}\OperatorTok{)} \OperatorTok{\{}
\NormalTok{    color }\OperatorTok{=} \DataTypeTok{vec3}\OperatorTok{(}\FloatTok{0.2}\OperatorTok{,} \FloatTok{0.5}\OperatorTok{,} \FloatTok{1.0}\OperatorTok{);}  \CommentTok{// blue earth}
\OperatorTok{\}}

\NormalTok{fragColor }\OperatorTok{=} \DataTypeTok{vec4}\OperatorTok{(}\NormalTok{color}\OperatorTok{,} \FloatTok{1.0}\OperatorTok{);}
\end{Highlighting}
\end{Shaded}

\pandocbounded{\includegraphics[keepaspectratio]{./demos/day1/sun-earth/screenshot.png}}

Drag to move the sun; the earth follows in orbit. (Exercise: add a moon
orbiting the earth!)

\subsection{Mouse as Parameter}\label{mouse-as-parameter}

The mouse doesn't have to control position---it can control any
parameter. A useful pattern: map \texttt{iMouse.x} to a parameter range
and drag across the screen to explore a family of curves.

The folium of Descartes is the curve \(x^3 + y^3 = 3axy\). We can
explore its level sets by drawing \(x^3 + y^3 - 3axy = c\) for different
values of \(c\):

\begin{Shaded}
\begin{Highlighting}[]
\DataTypeTok{vec2}\NormalTok{ p }\OperatorTok{=} \FunctionTok{normalize\_coord}\OperatorTok{(}\NormalTok{fragCoord}\OperatorTok{);}

\CommentTok{// Fixed parameter a}
\DataTypeTok{float}\NormalTok{ a }\OperatorTok{=} \FloatTok{1.5}\OperatorTok{;}

\CommentTok{// Map mouse x to level set value c in [{-}2, 2]}
\DataTypeTok{float}\NormalTok{ c }\OperatorTok{=} \BuiltInTok{mix}\OperatorTok{({-}}\FloatTok{2.0}\OperatorTok{,} \FloatTok{2.0}\OperatorTok{,}\NormalTok{ iMouse}\OperatorTok{.}\FunctionTok{x} \OperatorTok{/}\NormalTok{ iResolution}\OperatorTok{.}\FunctionTok{x}\OperatorTok{);}

\CommentTok{// Folium of Descartes: x³ + y³ {-} 3axy = c}
\DataTypeTok{float}\NormalTok{ F }\OperatorTok{=}\NormalTok{ p}\OperatorTok{.}\FunctionTok{x}\OperatorTok{*}\NormalTok{p}\OperatorTok{.}\FunctionTok{x}\OperatorTok{*}\NormalTok{p}\OperatorTok{.}\FunctionTok{x} \OperatorTok{+}\NormalTok{ p}\OperatorTok{.}\FunctionTok{y}\OperatorTok{*}\NormalTok{p}\OperatorTok{.}\FunctionTok{y}\OperatorTok{*}\NormalTok{p}\OperatorTok{.}\FunctionTok{y} \OperatorTok{{-}} \FloatTok{3.0}\OperatorTok{*}\NormalTok{a}\OperatorTok{*}\NormalTok{p}\OperatorTok{.}\FunctionTok{x}\OperatorTok{*}\NormalTok{p}\OperatorTok{.}\FunctionTok{y} \OperatorTok{{-}}\NormalTok{ c}\OperatorTok{;}

\CommentTok{// Gradient: ∇F = (3x² {-} 3ay, 3y² {-} 3ax)}
\DataTypeTok{vec2}\NormalTok{ grad }\OperatorTok{=} \DataTypeTok{vec2}\OperatorTok{(}\FloatTok{3.0}\OperatorTok{*}\NormalTok{p}\OperatorTok{.}\FunctionTok{x}\OperatorTok{*}\NormalTok{p}\OperatorTok{.}\FunctionTok{x} \OperatorTok{{-}} \FloatTok{3.0}\OperatorTok{*}\NormalTok{a}\OperatorTok{*}\NormalTok{p}\OperatorTok{.}\FunctionTok{y}\OperatorTok{,} \FloatTok{3.0}\OperatorTok{*}\NormalTok{p}\OperatorTok{.}\FunctionTok{y}\OperatorTok{*}\NormalTok{p}\OperatorTok{.}\FunctionTok{y} \OperatorTok{{-}} \FloatTok{3.0}\OperatorTok{*}\NormalTok{a}\OperatorTok{*}\NormalTok{p}\OperatorTok{.}\FunctionTok{x}\OperatorTok{);}
\DataTypeTok{float}\NormalTok{ dist }\OperatorTok{=} \BuiltInTok{abs}\OperatorTok{(}\NormalTok{F}\OperatorTok{)} \OperatorTok{/} \BuiltInTok{max}\OperatorTok{(}\BuiltInTok{length}\OperatorTok{(}\NormalTok{grad}\OperatorTok{),} \FloatTok{0.01}\OperatorTok{);}

\DataTypeTok{vec3}\NormalTok{ color}\OperatorTok{;}
\KeywordTok{if} \OperatorTok{(}\NormalTok{dist }\OperatorTok{\textless{}} \FloatTok{0.05}\OperatorTok{)} \OperatorTok{\{}
\NormalTok{    color }\OperatorTok{=} \DataTypeTok{vec3}\OperatorTok{(}\FloatTok{1.0}\OperatorTok{,} \FloatTok{1.0}\OperatorTok{,} \FloatTok{0.0}\OperatorTok{);}
\OperatorTok{\}} \KeywordTok{else} \OperatorTok{\{}
\NormalTok{    color }\OperatorTok{=} \DataTypeTok{vec3}\OperatorTok{(}\FloatTok{0.1}\OperatorTok{,} \FloatTok{0.1}\OperatorTok{,} \FloatTok{0.3}\OperatorTok{);}
\OperatorTok{\}}

\NormalTok{fragColor }\OperatorTok{=} \DataTypeTok{vec4}\OperatorTok{(}\NormalTok{color}\OperatorTok{,} \FloatTok{1.0}\OperatorTok{);}
\end{Highlighting}
\end{Shaded}

\pandocbounded{\includegraphics[keepaspectratio]{./demos/day1/folium-mouse/screenshot.png}}

Drag left and right to sweep through the level sets and watch the curve
topology change.

\section{Exercises}\label{exercises}

Homework is organized into four types:

\textbf{Checkpoints} --- Short exercises to verify you understood the
lecture material. Required for anyone new to shader programming.

\textbf{Explorations} --- Open-ended problems that extend the lecture
topics. Pick the ones that interest you. If you can do several of these,
you're right on track with the course.

\textbf{Challenges} --- Problems that may require learning new concepts
beyond what was covered in lecture. Attempt these if you skipped the
checkpoints and found an exploration or two too easy.

\textbf{Project} --- An extended project for someone familiar with
shader basics, to make an artwork.

\begin{center}\rule{0.5\linewidth}{0.5pt}\end{center}

\subsection{Checkpoints}\label{checkpoints}

\textbf{C1. Solid Colors.} Modify the red screen shader to display: (a)
green, (b) cyan, (c) a color of your choice using all three RGB
channels.

\textbf{C2. Vertical Split.} Modify the half-plane shader to divide the
screen into left (red) and right (blue) instead of top and bottom.

\textbf{C3. Off-Center Circle.} Draw a filled circle of radius 0.5
centered at the point \((1, 1)\) instead of the origin.

\textbf{C4. Pulsing Circle.} Make a circle whose radius oscillates
between 0.5 and 1.5 over time using \texttt{iTime}.

\textbf{C5. Ring Thickness.} Draw a ring (circle outline) centered at
the origin. Experiment with different values of \texttt{eps} to
understand how it controls thickness.

\begin{center}\rule{0.5\linewidth}{0.5pt}\end{center}

\subsection{Explorations}\label{explorations}

\textbf{E1. Concentric Rings.} Draw several concentric rings (circles of
different radii, all centered at the origin). Can you color alternate
rings differently?

\textbf{E2. Moon Orbit.} Extend the sun-earth shader to add a moon that
orbits the earth. The moon should be smaller than the earth and orbit
faster.

\textbf{E3. Your Favorite Curve.} Pick an implicit curve from your
mathematical experience (or find one online) and render it. Some
suggestions: the cardioid \((x^2 + y^2 - ax)^2 = a^2(x^2 + y^2)\), the
astroid \(x^{2/3} + y^{2/3} = a^{2/3}\), or a rose curve in implicit
form. Apply gradient correction for uniform thickness.

\textbf{E4. Curve Explorer.} Take any one-parameter family of curves and
build a mouse-controlled explorer (like the folium example). Map
\texttt{iMouse.x} to the parameter and drag to explore the family.

\textbf{E5. Two Circles.} Draw two filled circles at different
positions. What happens when they overlap? Can you make one ``in front
of'' the other? Can you make the intersection a different color, like a
Venn diagram?

\begin{center}\rule{0.5\linewidth}{0.5pt}\end{center}

\subsection{Challenges}\label{challenges}

\textbf{H1. Parabola Graphing Calculator.} Build an interactive graphing
calculator for the parabola \(y = ax^2 + bx + c\). Requirements: - Draw
coordinate axes (the lines \(x = 0\) and \(y = 0\)) - Draw the parabola
using implicit curve techniques - Find the roots (where \(y = 0\)) and
draw small circles around them - Use mouse position to control two of
the coefficients (e.g., \(a\) and \(b\), with \(c\) fixed, or \(b\) and
\(c\) with \(a\) fixed)

As you drag the mouse, the parabola should reshape and the root
indicators should move (or appear/disappear as roots become real or
complex).

\textbf{H2. Elliptic Curve Explorer.} Elliptic curves in Weierstrass
form are \(y^2 = x^3 + ax + b\). Build a shader where the mouse position
controls \((a, b)\). Use gradient correction for uniform thickness. The
\emph{discriminant} \(\Delta = 4a^3 + 27b^2\) determines whether the
curve is smooth (\(\Delta \neq 0\)) or singular (\(\Delta = 0\)). Can
you display the current value of \(\Delta\) somehow, or change the
curve's color when it becomes singular?

\textbf{H3. Signed Distance Functions.} For a filled circle,
\(f(p) = |p| - r\) is the \emph{signed} distance function: negative
inside, positive outside, with \(|f|\) giving the actual distance to the
boundary. What is the signed distance function for a half-plane? For an
axis-aligned rectangle? Implement both and draw them with
uniform-thickness boundaries. Note: when you have the true signed
distance function, you don't need the gradient correction trick---that's
the payoff for computing the right thing from the start!

\textbf{H4. Smooth Blending.} When two circles overlap, we currently
just draw one on top of the other. Research \emph{smooth minimum}
functions (e.g., \texttt{smin}) that blend distance fields smoothly.
Draw two circles that ``melt together'' where they meet.

\textbf{H5. Inversion.} Circle inversion is the map
\(p \mapsto p / |p|^2\). Apply this transformation to your coordinate
\(p\) before drawing a shape. What happens to a line? What happens to a
circle not passing through the origin? Experiment with different shapes.

\begin{center}\rule{0.5\linewidth}{0.5pt}\end{center}

\subsection{Project: Grid Patterns}\label{project-grid-patterns}

This extended project introduces a powerful technique---using modular
arithmetic to repeat patterns across the plane. We'll build up the
machinery carefully, since we'll use it again in Day 2 to create grids
of Julia sets.

\subsubsection{Part 1: Setting Up a Grid of Square
Cells}\label{part-1-setting-up-a-grid-of-square-cells}

We want to tile the screen with square cells---say, 4 cells across. The
challenge: the screen isn't square, so we need to handle the aspect
ratio.

Let's say we want \texttt{N} columns of cells. Each cell has width
\(L = \text{screen\_width} / N\) in pixels, and since cells are square,
height \(L\) as well. The number of rows depends on the screen's aspect
ratio.

Working in our normalized coordinates (after aspect correction), the
screen spans roughly \([-2 \cdot \text{aspect}, 2 \cdot \text{aspect}]\)
in \(x\) and \([-2, 2]\) in \(y\). If we want cells of side length \(L\)
in these coordinates:

\begin{Shaded}
\begin{Highlighting}[]
\DataTypeTok{float}\NormalTok{ aspect }\OperatorTok{=}\NormalTok{ iResolution}\OperatorTok{.}\FunctionTok{x} \OperatorTok{/}\NormalTok{ iResolution}\OperatorTok{.}\FunctionTok{y}\OperatorTok{;}
\DataTypeTok{float}\NormalTok{ N }\OperatorTok{=} \FloatTok{5.0}\OperatorTok{;}  \CommentTok{// number of columns}
\DataTypeTok{float}\NormalTok{ L }\OperatorTok{=} \OperatorTok{(}\FloatTok{4.0} \OperatorTok{*}\NormalTok{ aspect}\OperatorTok{)} \OperatorTok{/}\NormalTok{ N}\OperatorTok{;}  \CommentTok{// cell size in our coordinate system}
\end{Highlighting}
\end{Shaded}

Now each cell is an \(L \times L\) square.

\subsubsection{Part 2: Cell Coordinates and
Identity}\label{part-2-cell-coordinates-and-identity}

For each pixel, we want two things:

\begin{enumerate}
\def\labelenumi{\arabic{enumi}.}
\tightlist
\item
  \textbf{Which cell are we in?} Integer coordinates \((i, j)\)
  identifying the cell.
\item
  \textbf{Where in the cell are we?} Local coordinates ranging from
  \(-L/2\) to \(L/2\), with \((0,0)\) at the cell center.
\end{enumerate}

\begin{Shaded}
\begin{Highlighting}[]
\DataTypeTok{vec2}\NormalTok{ cell\_id }\OperatorTok{=} \BuiltInTok{floor}\OperatorTok{(}\NormalTok{p }\OperatorTok{/}\NormalTok{ L}\OperatorTok{);}
\DataTypeTok{vec2}\NormalTok{ cell\_p }\OperatorTok{=} \BuiltInTok{mod}\OperatorTok{(}\NormalTok{p }\OperatorTok{+} \DataTypeTok{vec2}\OperatorTok{(}\NormalTok{L}\OperatorTok{/}\FloatTok{2.0}\OperatorTok{,}\NormalTok{ L}\OperatorTok{/}\FloatTok{2.0}\OperatorTok{),}\NormalTok{ L}\OperatorTok{)} \OperatorTok{{-}} \DataTypeTok{vec2}\OperatorTok{(}\NormalTok{L}\OperatorTok{/}\FloatTok{2.0}\OperatorTok{,}\NormalTok{ L}\OperatorTok{/}\FloatTok{2.0}\OperatorTok{);}
\end{Highlighting}
\end{Shaded}

The \texttt{cell\_id} tells us which cell; the \texttt{cell\_p} gives
local coordinates within that cell.

If we want local coordinates normalized to \([-1, 1]\) (useful for
drawing things at a standard scale), we can rescale:

\begin{Shaded}
\begin{Highlighting}[]
\DataTypeTok{vec2}\NormalTok{ local }\OperatorTok{=}\NormalTok{ cell\_p }\OperatorTok{/} \OperatorTok{(}\NormalTok{L }\OperatorTok{/} \FloatTok{2.0}\OperatorTok{);}  \CommentTok{// now in [{-}1, 1] x [{-}1, 1]}
\end{Highlighting}
\end{Shaded}

This is exactly the setup we'll need for Day 2, where each cell will
contain a Julia set with its own coordinate system.

\subsubsection{Part 3: Drawing in Each
Cell}\label{part-3-drawing-in-each-cell}

Now draw something using the local coordinates. A filled circle at the
center of each cell:

\begin{Shaded}
\begin{Highlighting}[]
\DataTypeTok{float}\NormalTok{ d }\OperatorTok{=} \BuiltInTok{length}\OperatorTok{(}\NormalTok{cell\_p}\OperatorTok{);}
\DataTypeTok{float}\NormalTok{ r }\OperatorTok{=}\NormalTok{ L }\OperatorTok{*} \FloatTok{0.4}\OperatorTok{;}  \CommentTok{// radius relative to cell size}

\DataTypeTok{vec3}\NormalTok{ color}\OperatorTok{;}
\KeywordTok{if} \OperatorTok{(}\NormalTok{d }\OperatorTok{\textless{}}\NormalTok{ r}\OperatorTok{)} \OperatorTok{\{}
\NormalTok{    color }\OperatorTok{=} \DataTypeTok{vec3}\OperatorTok{(}\FloatTok{1.0}\OperatorTok{,} \FloatTok{1.0}\OperatorTok{,} \FloatTok{0.0}\OperatorTok{);}
\OperatorTok{\}} \KeywordTok{else} \OperatorTok{\{}
\NormalTok{    color }\OperatorTok{=} \DataTypeTok{vec3}\OperatorTok{(}\FloatTok{0.1}\OperatorTok{,} \FloatTok{0.1}\OperatorTok{,} \FloatTok{0.3}\OperatorTok{);}
\OperatorTok{\}}
\end{Highlighting}
\end{Shaded}

\pandocbounded{\includegraphics[keepaspectratio]{./demos/day1/grid-circles/screenshot.png}}

Try changing \texttt{N} to get more or fewer columns. The cells stay
square regardless of screen shape.

\subsubsection{Part 4: Varying by Cell}\label{part-4-varying-by-cell}

The \texttt{cell\_id} lets each cell behave differently. Some ideas:

\textbf{Checkerboard background:}

\begin{Shaded}
\begin{Highlighting}[]
\DataTypeTok{float}\NormalTok{ checker }\OperatorTok{=} \BuiltInTok{mod}\OperatorTok{(}\NormalTok{cell\_id}\OperatorTok{.}\FunctionTok{x} \OperatorTok{+}\NormalTok{ cell\_id}\OperatorTok{.}\FunctionTok{y}\OperatorTok{,} \FloatTok{2.0}\OperatorTok{);}
\DataTypeTok{vec3}\NormalTok{ bg }\OperatorTok{=} \BuiltInTok{mix}\OperatorTok{(}\DataTypeTok{vec3}\OperatorTok{(}\FloatTok{0.2}\OperatorTok{,} \FloatTok{0.2}\OperatorTok{,} \FloatTok{0.3}\OperatorTok{),} \DataTypeTok{vec3}\OperatorTok{(}\FloatTok{0.3}\OperatorTok{,} \FloatTok{0.2}\OperatorTok{,} \FloatTok{0.2}\OperatorTok{),}\NormalTok{ checker}\OperatorTok{);}
\end{Highlighting}
\end{Shaded}

\textbf{Radius varying by cell:}

\begin{Shaded}
\begin{Highlighting}[]
\DataTypeTok{float}\NormalTok{ r }\OperatorTok{=}\NormalTok{ L }\OperatorTok{*} \OperatorTok{(}\FloatTok{0.2} \OperatorTok{+} \FloatTok{0.15} \OperatorTok{*} \BuiltInTok{mod}\OperatorTok{(}\NormalTok{cell\_id}\OperatorTok{.}\FunctionTok{x} \OperatorTok{+}\NormalTok{ cell\_id}\OperatorTok{.}\FunctionTok{y}\OperatorTok{,} \FloatTok{3.0}\OperatorTok{));}
\end{Highlighting}
\end{Shaded}

\textbf{Wave animation:}

\begin{Shaded}
\begin{Highlighting}[]
\DataTypeTok{float}\NormalTok{ cell\_dist }\OperatorTok{=} \BuiltInTok{length}\OperatorTok{(}\NormalTok{cell\_id}\OperatorTok{);}
\DataTypeTok{float}\NormalTok{ r }\OperatorTok{=}\NormalTok{ L }\OperatorTok{*} \OperatorTok{(}\FloatTok{0.3} \OperatorTok{+} \FloatTok{0.1} \OperatorTok{*} \BuiltInTok{sin}\OperatorTok{(}\NormalTok{iTime }\OperatorTok{*} \FloatTok{2.0} \OperatorTok{{-}}\NormalTok{ cell\_dist }\OperatorTok{*} \FloatTok{0.5}\OperatorTok{));}
\end{Highlighting}
\end{Shaded}

\subsubsection{Part 5: Design Challenge}\label{part-5-design-challenge}

Design a grid-based pattern that you find visually interesting. Some
directions:

\textbf{Connecting shapes:} Draw shapes that connect across cell
boundaries. Quarter-circles in each corner create a continuous network.
What implicit curves tile seamlessly?

\textbf{Alternating motifs:} Use \texttt{cell\_id} to alternate between
different shapes---circles in some cells, rings in others, or different
orientations.

\textbf{Color fields:} Map \texttt{cell\_id} to colors using distance
from origin, stripes, or a palette.

\textbf{Phase shifts:} Animate cells with different phase offsets to
create waves or ripples.

\textbf{Using local coordinates:} Draw something more complex in each
cell using the \([-1,1]\) local coordinate system---perhaps a small
implicit curve, or a pattern that changes based on \texttt{cell\_id}.

The goal is to produce an image you'd be happy to hang on a wall.

\subsection{Project: Fourier Epicycles}\label{project-fourier-epicycles}

This project builds a visualization of Fourier series using
epicycles---circles whose centers sit on the circumferences of other
circles. This is how Ptolemy modeled planetary motion, and it turns out
to be exactly how Fourier series work geometrically.

\subsubsection{Part 1: The Idea}\label{part-1-the-idea}

Any periodic function can be written as a sum of sines and cosines.
Geometrically, \(\sin(n\omega t)\) and \(\cos(n\omega t)\) describe a
point moving around a circle of frequency \(n\omega\). Adding these
components corresponds to stacking circles: each circle's center rides
on the previous circle's edge.

For example, the square wave has Fourier series:
\[f(t) = \sum_{n=1,3,5,...} \frac{1}{n} \sin(n\omega t)\]

This means circles with: - Radii:
\(1, \frac{1}{3}, \frac{1}{5}, \frac{1}{7}, ...\) - Frequencies:
\(\omega, 3\omega, 5\omega, 7\omega, ...\)

The more terms we add, the closer the final point's \(y\)-coordinate
approximates a square wave.

\subsubsection{Part 2: Drawing Circles}\label{part-2-drawing-circles}

Start by drawing a chain of circles. Each circle is centered at the
current position, and the next position is computed by moving along the
circle:

\begin{Shaded}
\begin{Highlighting}[]
\DataTypeTok{vec2}\NormalTok{ pos }\OperatorTok{=} \DataTypeTok{vec2}\OperatorTok{(}\FloatTok{0.0}\OperatorTok{,} \FloatTok{0.0}\OperatorTok{);}  \CommentTok{// start at origin}

\KeywordTok{for} \OperatorTok{(}\DataTypeTok{int}\NormalTok{ i }\OperatorTok{=} \DecValTok{0}\OperatorTok{;}\NormalTok{ i }\OperatorTok{\textless{}}\NormalTok{ N}\OperatorTok{;}\NormalTok{ i}\OperatorTok{++)} \OperatorTok{\{}
    \DataTypeTok{int}\NormalTok{ n }\OperatorTok{=} \DecValTok{2} \OperatorTok{*}\NormalTok{ i }\OperatorTok{+} \DecValTok{1}\OperatorTok{;}  \CommentTok{// 1, 3, 5, 7, ...}
    \DataTypeTok{float}\NormalTok{ r }\OperatorTok{=}\NormalTok{ scale }\OperatorTok{/} \DataTypeTok{float}\OperatorTok{(}\NormalTok{n}\OperatorTok{);}
    \DataTypeTok{float}\NormalTok{ freq }\OperatorTok{=} \DataTypeTok{float}\OperatorTok{(}\NormalTok{n}\OperatorTok{)} \OperatorTok{*}\NormalTok{ omega}\OperatorTok{;}
    
    \CommentTok{// Draw circle at current position}
    \DataTypeTok{float}\NormalTok{ d\_circle }\OperatorTok{=} \BuiltInTok{abs}\OperatorTok{(}\BuiltInTok{length}\OperatorTok{(}\NormalTok{p }\OperatorTok{{-}}\NormalTok{ pos}\OperatorTok{)} \OperatorTok{{-}}\NormalTok{ r}\OperatorTok{);}
    \KeywordTok{if} \OperatorTok{(}\NormalTok{d\_circle }\OperatorTok{\textless{}} \FloatTok{0.02}\OperatorTok{)} \OperatorTok{\{}
        \CommentTok{// color the circle}
    \OperatorTok{\}}
    
    \CommentTok{// Move to next position}
\NormalTok{    pos }\OperatorTok{=}\NormalTok{ pos }\OperatorTok{+}\NormalTok{ r }\OperatorTok{*} \DataTypeTok{vec2}\OperatorTok{(}\BuiltInTok{cos}\OperatorTok{(}\NormalTok{freq }\OperatorTok{*}\NormalTok{ iTime}\OperatorTok{),} \BuiltInTok{sin}\OperatorTok{(}\NormalTok{freq }\OperatorTok{*}\NormalTok{ iTime}\OperatorTok{));}
\OperatorTok{\}}

\CommentTok{// Draw final point}
\DataTypeTok{float}\NormalTok{ d\_point }\OperatorTok{=} \BuiltInTok{length}\OperatorTok{(}\NormalTok{p }\OperatorTok{{-}}\NormalTok{ pos}\OperatorTok{);}
\KeywordTok{if} \OperatorTok{(}\NormalTok{d\_point }\OperatorTok{\textless{}} \FloatTok{0.08}\OperatorTok{)} \OperatorTok{\{}
    \CommentTok{// bright color}
\OperatorTok{\}}
\end{Highlighting}
\end{Shaded}

Try this with \texttt{N\ =\ 1}, then \texttt{N\ =\ 3}, then
\texttt{N\ =\ 7}. Watch how more circles create more complex motion.

\subsubsection{Part 3: The Line Segment
SDF}\label{part-3-the-line-segment-sdf}

To draw the arms connecting circle centers, we need the signed distance
function for a line segment. Given endpoints \(a\) and \(b\), the
distance from point \(p\) to the segment is:

\begin{Shaded}
\begin{Highlighting}[]
\DataTypeTok{float} \FunctionTok{sd\_segment}\OperatorTok{(}\DataTypeTok{vec2}\NormalTok{ p}\OperatorTok{,} \DataTypeTok{vec2}\NormalTok{ a}\OperatorTok{,} \DataTypeTok{vec2}\NormalTok{ b}\OperatorTok{)} \OperatorTok{\{}
    \DataTypeTok{vec2}\NormalTok{ pa }\OperatorTok{=}\NormalTok{ p }\OperatorTok{{-}}\NormalTok{ a}\OperatorTok{;}
    \DataTypeTok{vec2}\NormalTok{ ba }\OperatorTok{=}\NormalTok{ b }\OperatorTok{{-}}\NormalTok{ a}\OperatorTok{;}
    \DataTypeTok{float}\NormalTok{ t }\OperatorTok{=} \BuiltInTok{clamp}\OperatorTok{(}\BuiltInTok{dot}\OperatorTok{(}\NormalTok{pa}\OperatorTok{,}\NormalTok{ ba}\OperatorTok{)} \OperatorTok{/} \BuiltInTok{dot}\OperatorTok{(}\NormalTok{ba}\OperatorTok{,}\NormalTok{ ba}\OperatorTok{),} \FloatTok{0.0}\OperatorTok{,} \FloatTok{1.0}\OperatorTok{);}
    \KeywordTok{return} \BuiltInTok{length}\OperatorTok{(}\NormalTok{pa }\OperatorTok{{-}}\NormalTok{ ba }\OperatorTok{*}\NormalTok{ t}\OperatorTok{);}
\OperatorTok{\}}
\end{Highlighting}
\end{Shaded}

The math: we project \(p - a\) onto the line direction \(b - a\), clamp
to \([0, 1]\) to stay within the segment, then measure the distance to
that closest point.

\subsubsection{Part 4: Connecting the
Arms}\label{part-4-connecting-the-arms}

Now modify your loop to also draw line segments:

\begin{Shaded}
\begin{Highlighting}[]
\DataTypeTok{vec2}\NormalTok{ pos }\OperatorTok{=} \DataTypeTok{vec2}\OperatorTok{(}\FloatTok{0.0}\OperatorTok{,} \FloatTok{0.0}\OperatorTok{);}

\KeywordTok{for} \OperatorTok{(}\DataTypeTok{int}\NormalTok{ i }\OperatorTok{=} \DecValTok{0}\OperatorTok{;}\NormalTok{ i }\OperatorTok{\textless{}}\NormalTok{ N}\OperatorTok{;}\NormalTok{ i}\OperatorTok{++)} \OperatorTok{\{}
    \DataTypeTok{int}\NormalTok{ n }\OperatorTok{=} \DecValTok{2} \OperatorTok{*}\NormalTok{ i }\OperatorTok{+} \DecValTok{1}\OperatorTok{;}
    \DataTypeTok{float}\NormalTok{ r }\OperatorTok{=}\NormalTok{ scale }\OperatorTok{/} \DataTypeTok{float}\OperatorTok{(}\NormalTok{n}\OperatorTok{);}
    \DataTypeTok{float}\NormalTok{ freq }\OperatorTok{=} \DataTypeTok{float}\OperatorTok{(}\NormalTok{n}\OperatorTok{)} \OperatorTok{*}\NormalTok{ omega}\OperatorTok{;}
    
    \DataTypeTok{vec2}\NormalTok{ next\_pos }\OperatorTok{=}\NormalTok{ pos }\OperatorTok{+}\NormalTok{ r }\OperatorTok{*} \DataTypeTok{vec2}\OperatorTok{(}\BuiltInTok{cos}\OperatorTok{(}\NormalTok{freq }\OperatorTok{*}\NormalTok{ iTime}\OperatorTok{),} \BuiltInTok{sin}\OperatorTok{(}\NormalTok{freq }\OperatorTok{*}\NormalTok{ iTime}\OperatorTok{));}
    
    \CommentTok{// Draw circle}
    \DataTypeTok{float}\NormalTok{ d\_circle }\OperatorTok{=} \BuiltInTok{abs}\OperatorTok{(}\BuiltInTok{length}\OperatorTok{(}\NormalTok{p }\OperatorTok{{-}}\NormalTok{ pos}\OperatorTok{)} \OperatorTok{{-}}\NormalTok{ r}\OperatorTok{);}
    \KeywordTok{if} \OperatorTok{(}\NormalTok{d\_circle }\OperatorTok{\textless{}} \FloatTok{0.02}\OperatorTok{)} \OperatorTok{\{}
        \CommentTok{// faint circle color}
    \OperatorTok{\}}
    
    \CommentTok{// Draw arm from pos to next\_pos}
    \DataTypeTok{float}\NormalTok{ d\_arm }\OperatorTok{=} \FunctionTok{sd\_segment}\OperatorTok{(}\NormalTok{p}\OperatorTok{,}\NormalTok{ pos}\OperatorTok{,}\NormalTok{ next\_pos}\OperatorTok{);}
    \KeywordTok{if} \OperatorTok{(}\NormalTok{d\_arm }\OperatorTok{\textless{}} \FloatTok{0.015}\OperatorTok{)} \OperatorTok{\{}
        \CommentTok{// arm color}
    \OperatorTok{\}}
    
\NormalTok{    pos }\OperatorTok{=}\NormalTok{ next\_pos}\OperatorTok{;}
\OperatorTok{\}}
\end{Highlighting}
\end{Shaded}

\subsubsection{Part 5: Polish and
Explore}\label{part-5-polish-and-explore}

Now make it beautiful:

\textbf{Fading circles:} Later circles are smaller and less important.
Fade their brightness:

\begin{Shaded}
\begin{Highlighting}[]
\DataTypeTok{float}\NormalTok{ fade }\OperatorTok{=} \FloatTok{1.0} \OperatorTok{{-}} \DataTypeTok{float}\OperatorTok{(}\NormalTok{i}\OperatorTok{)} \OperatorTok{/} \DataTypeTok{float}\OperatorTok{(}\NormalTok{N}\OperatorTok{);}
\end{Highlighting}
\end{Shaded}

\textbf{Color variation:} Color circles differently based on their
index, or based on their frequency.

\textbf{Different waves:} The square wave uses odd harmonics with
\(1/n\) coefficients. Try: - Triangle wave: odd harmonics with \(1/n^2\)
coefficients (alternating signs) - Sawtooth wave: all harmonics with
\(1/n\) coefficients

\textbf{Mouse control:} Map \texttt{iMouse.x} to the number of terms, so
dragging adds or removes circles.

The goal: create a mesmerizing animation that reveals the geometry
hidden inside Fourier series.

\cleardoublepage
\phantomsection
\addcontentsline{toc}{part}{Appendices}
\appendix

\chapter{Appendix: Day 1 Shader Code}\label{appendix-day-1-shader-code}

This appendix provides complete, standalone code for each shader
referenced in Day 1. Each listing can be copied directly into
\href{https://www.shadertoy.com/new}{Shadertoy} and run immediately.

\begin{Shaded}
\begin{Highlighting}[]
\DataTypeTok{vec2} \FunctionTok{normalize\_coord}\OperatorTok{(}\DataTypeTok{vec2}\NormalTok{ coord}\OperatorTok{)} \OperatorTok{\{}
    \DataTypeTok{vec2}\NormalTok{ uv }\OperatorTok{=}\NormalTok{ coord }\OperatorTok{/}\NormalTok{ iResolution}\OperatorTok{.}\FunctionTok{xy}\OperatorTok{;}
\NormalTok{    uv }\OperatorTok{=}\NormalTok{ uv }\OperatorTok{{-}} \DataTypeTok{vec2}\OperatorTok{(}\FloatTok{0.5}\OperatorTok{,} \FloatTok{0.5}\OperatorTok{);}
\NormalTok{    uv}\OperatorTok{.}\FunctionTok{x} \OperatorTok{*=}\NormalTok{ iResolution}\OperatorTok{.}\FunctionTok{x} \OperatorTok{/}\NormalTok{ iResolution}\OperatorTok{.}\FunctionTok{y}\OperatorTok{;}
    \KeywordTok{return}\NormalTok{ uv }\OperatorTok{*} \FloatTok{4.0}\OperatorTok{;}
\OperatorTok{\}}

\DataTypeTok{void} \FunctionTok{mainImage}\OperatorTok{(}\DataTypeTok{out} \DataTypeTok{vec4}\NormalTok{ fragColor}\OperatorTok{,} \DataTypeTok{in} \DataTypeTok{vec2}\NormalTok{ fragCoord}\OperatorTok{)}
\OperatorTok{\{}
    \DataTypeTok{vec2}\NormalTok{ p }\OperatorTok{=} \FunctionTok{normalize\_coord}\OperatorTok{(}\NormalTok{fragCoord}\OperatorTok{);}
    \DataTypeTok{vec2}\NormalTok{ sun }\OperatorTok{=} \FunctionTok{normalize\_coord}\OperatorTok{(}\NormalTok{iMouse}\OperatorTok{.}\FunctionTok{xy}\OperatorTok{);}
    
    \CommentTok{// Earth orbits the sun}
    \DataTypeTok{float}\NormalTok{ orbit\_radius }\OperatorTok{=} \FloatTok{0.8}\OperatorTok{;}
    \DataTypeTok{vec2}\NormalTok{ earth }\OperatorTok{=}\NormalTok{ sun }\OperatorTok{+}\NormalTok{ orbit\_radius }\OperatorTok{*} \DataTypeTok{vec2}\OperatorTok{(}\BuiltInTok{cos}\OperatorTok{(}\NormalTok{iTime}\OperatorTok{),} \BuiltInTok{sin}\OperatorTok{(}\NormalTok{iTime}\OperatorTok{));}
    
    \CommentTok{// Draw sun (larger, yellow)}
    \DataTypeTok{float}\NormalTok{ d\_sun }\OperatorTok{=} \BuiltInTok{length}\OperatorTok{(}\NormalTok{p }\OperatorTok{{-}}\NormalTok{ sun}\OperatorTok{);}
    \CommentTok{// Draw earth (smaller, blue)}
    \DataTypeTok{float}\NormalTok{ d\_earth }\OperatorTok{=} \BuiltInTok{length}\OperatorTok{(}\NormalTok{p }\OperatorTok{{-}}\NormalTok{ earth}\OperatorTok{);}
    
    \DataTypeTok{vec3}\NormalTok{ color }\OperatorTok{=} \DataTypeTok{vec3}\OperatorTok{(}\FloatTok{0.02}\OperatorTok{,} \FloatTok{0.02}\OperatorTok{,} \FloatTok{0.05}\OperatorTok{);}  \CommentTok{// dark background}
    \KeywordTok{if} \OperatorTok{(}\NormalTok{d\_sun }\OperatorTok{\textless{}} \FloatTok{0.3}\OperatorTok{)} \OperatorTok{\{}
\NormalTok{        color }\OperatorTok{=} \DataTypeTok{vec3}\OperatorTok{(}\FloatTok{1.0}\OperatorTok{,} \FloatTok{0.9}\OperatorTok{,} \FloatTok{0.2}\OperatorTok{);}  \CommentTok{// yellow sun}
    \OperatorTok{\}}
    \KeywordTok{if} \OperatorTok{(}\NormalTok{d\_earth }\OperatorTok{\textless{}} \FloatTok{0.15}\OperatorTok{)} \OperatorTok{\{}
\NormalTok{        color }\OperatorTok{=} \DataTypeTok{vec3}\OperatorTok{(}\FloatTok{0.2}\OperatorTok{,} \FloatTok{0.5}\OperatorTok{,} \FloatTok{1.0}\OperatorTok{);}  \CommentTok{// blue earth}
    \OperatorTok{\}}
    
\NormalTok{    fragColor }\OperatorTok{=} \DataTypeTok{vec4}\OperatorTok{(}\NormalTok{color}\OperatorTok{,} \FloatTok{1.0}\OperatorTok{);}
\OperatorTok{\}}
\end{Highlighting}
\end{Shaded}

\section{A1. red}\label{a1.-red}

The simplest shader: every pixel is red.

\begin{Shaded}
\begin{Highlighting}[]
\DataTypeTok{void} \FunctionTok{mainImage}\OperatorTok{(}\DataTypeTok{out} \DataTypeTok{vec4}\NormalTok{ fragColor}\OperatorTok{,} \DataTypeTok{in} \DataTypeTok{vec2}\NormalTok{ fragCoord}\OperatorTok{)}
\OperatorTok{\{}
\NormalTok{    fragColor }\OperatorTok{=} \DataTypeTok{vec4}\OperatorTok{(}\FloatTok{1.0}\OperatorTok{,} \FloatTok{0.0}\OperatorTok{,} \FloatTok{0.0}\OperatorTok{,} \FloatTok{1.0}\OperatorTok{);}
\OperatorTok{\}}
\end{Highlighting}
\end{Shaded}

\begin{center}\rule{0.5\linewidth}{0.5pt}\end{center}

\section{A2. red-pulsing}\label{a2.-red-pulsing}

Red channel oscillates with time.

\begin{Shaded}
\begin{Highlighting}[]
\DataTypeTok{void} \FunctionTok{mainImage}\OperatorTok{(}\DataTypeTok{out} \DataTypeTok{vec4}\NormalTok{ fragColor}\OperatorTok{,} \DataTypeTok{in} \DataTypeTok{vec2}\NormalTok{ fragCoord}\OperatorTok{)}
\OperatorTok{\{}
    \DataTypeTok{float}\NormalTok{ red }\OperatorTok{=} \FloatTok{0.5} \OperatorTok{+} \FloatTok{0.5} \OperatorTok{*} \BuiltInTok{sin}\OperatorTok{(}\NormalTok{iTime}\OperatorTok{);}
\NormalTok{    fragColor }\OperatorTok{=} \DataTypeTok{vec4}\OperatorTok{(}\NormalTok{red}\OperatorTok{,} \FloatTok{0.0}\OperatorTok{,} \FloatTok{0.0}\OperatorTok{,} \FloatTok{1.0}\OperatorTok{);}
\OperatorTok{\}}
\end{Highlighting}
\end{Shaded}

\begin{center}\rule{0.5\linewidth}{0.5pt}\end{center}

\section{A3. coordinates}\label{a3.-coordinates}

Visualizing the coordinate system as color.

\begin{Shaded}
\begin{Highlighting}[]
\DataTypeTok{void} \FunctionTok{mainImage}\OperatorTok{(}\DataTypeTok{out} \DataTypeTok{vec4}\NormalTok{ fragColor}\OperatorTok{,} \DataTypeTok{in} \DataTypeTok{vec2}\NormalTok{ fragCoord}\OperatorTok{)}
\OperatorTok{\{}
    \DataTypeTok{vec2}\NormalTok{ uv }\OperatorTok{=}\NormalTok{ fragCoord }\OperatorTok{/}\NormalTok{ iResolution}\OperatorTok{.}\FunctionTok{xy}\OperatorTok{;}
\NormalTok{    fragColor }\OperatorTok{=} \DataTypeTok{vec4}\OperatorTok{(}\NormalTok{uv}\OperatorTok{.}\FunctionTok{x}\OperatorTok{,}\NormalTok{ uv}\OperatorTok{.}\FunctionTok{y}\OperatorTok{,} \FloatTok{0.0}\OperatorTok{,} \FloatTok{1.0}\OperatorTok{);}
\OperatorTok{\}}
\end{Highlighting}
\end{Shaded}

\begin{center}\rule{0.5\linewidth}{0.5pt}\end{center}

\section{A4. half-plane}\label{a4.-half-plane}

Dividing the plane into two regions based on y-coordinate.

\begin{Shaded}
\begin{Highlighting}[]
\DataTypeTok{void} \FunctionTok{mainImage}\OperatorTok{(}\DataTypeTok{out} \DataTypeTok{vec4}\NormalTok{ fragColor}\OperatorTok{,} \DataTypeTok{in} \DataTypeTok{vec2}\NormalTok{ fragCoord}\OperatorTok{)}
\OperatorTok{\{}
    \DataTypeTok{vec2}\NormalTok{ uv }\OperatorTok{=}\NormalTok{ fragCoord }\OperatorTok{/}\NormalTok{ iResolution}\OperatorTok{.}\FunctionTok{xy}\OperatorTok{;}
\NormalTok{    uv }\OperatorTok{=}\NormalTok{ uv }\OperatorTok{{-}} \DataTypeTok{vec2}\OperatorTok{(}\FloatTok{0.5}\OperatorTok{,} \FloatTok{0.5}\OperatorTok{);}
\NormalTok{    uv}\OperatorTok{.}\FunctionTok{x} \OperatorTok{*=}\NormalTok{ iResolution}\OperatorTok{.}\FunctionTok{x} \OperatorTok{/}\NormalTok{ iResolution}\OperatorTok{.}\FunctionTok{y}\OperatorTok{;}
    \DataTypeTok{vec2}\NormalTok{ p }\OperatorTok{=}\NormalTok{ uv }\OperatorTok{*} \FloatTok{4.0}\OperatorTok{;}
    
    \DataTypeTok{float}\NormalTok{ L }\OperatorTok{=}\NormalTok{ p}\OperatorTok{.}\FunctionTok{y}\OperatorTok{;}
    
    \DataTypeTok{vec3}\NormalTok{ color}\OperatorTok{;}
    \KeywordTok{if} \OperatorTok{(}\NormalTok{L }\OperatorTok{\textless{}} \FloatTok{0.0}\OperatorTok{)} \OperatorTok{\{}
\NormalTok{        color }\OperatorTok{=} \DataTypeTok{vec3}\OperatorTok{(}\FloatTok{1.0}\OperatorTok{,} \FloatTok{0.0}\OperatorTok{,} \FloatTok{0.0}\OperatorTok{);}  \CommentTok{// red below}
    \OperatorTok{\}} \KeywordTok{else} \OperatorTok{\{}
\NormalTok{        color }\OperatorTok{=} \DataTypeTok{vec3}\OperatorTok{(}\FloatTok{0.0}\OperatorTok{,} \FloatTok{0.0}\OperatorTok{,} \FloatTok{1.0}\OperatorTok{);}  \CommentTok{// blue above}
    \OperatorTok{\}}
    
\NormalTok{    fragColor }\OperatorTok{=} \DataTypeTok{vec4}\OperatorTok{(}\NormalTok{color}\OperatorTok{,} \FloatTok{1.0}\OperatorTok{);}
\OperatorTok{\}}
\end{Highlighting}
\end{Shaded}

\begin{center}\rule{0.5\linewidth}{0.5pt}\end{center}

\section{A5. half-plane-animated}\label{a5.-half-plane-animated}

Animated line dividing the plane, with rotating normal and shifting
offset.

\begin{Shaded}
\begin{Highlighting}[]
\DataTypeTok{void} \FunctionTok{mainImage}\OperatorTok{(}\DataTypeTok{out} \DataTypeTok{vec4}\NormalTok{ fragColor}\OperatorTok{,} \DataTypeTok{in} \DataTypeTok{vec2}\NormalTok{ fragCoord}\OperatorTok{)}
\OperatorTok{\{}
    \DataTypeTok{vec2}\NormalTok{ uv }\OperatorTok{=}\NormalTok{ fragCoord }\OperatorTok{/}\NormalTok{ iResolution}\OperatorTok{.}\FunctionTok{xy}\OperatorTok{;}
\NormalTok{    uv }\OperatorTok{=}\NormalTok{ uv }\OperatorTok{{-}} \DataTypeTok{vec2}\OperatorTok{(}\FloatTok{0.5}\OperatorTok{,} \FloatTok{0.5}\OperatorTok{);}
\NormalTok{    uv}\OperatorTok{.}\FunctionTok{x} \OperatorTok{*=}\NormalTok{ iResolution}\OperatorTok{.}\FunctionTok{x} \OperatorTok{/}\NormalTok{ iResolution}\OperatorTok{.}\FunctionTok{y}\OperatorTok{;}
    \DataTypeTok{vec2}\NormalTok{ p }\OperatorTok{=}\NormalTok{ uv }\OperatorTok{*} \FloatTok{4.0}\OperatorTok{;}
    
    \DataTypeTok{float}\NormalTok{ a }\OperatorTok{=} \BuiltInTok{cos}\OperatorTok{(}\NormalTok{iTime}\OperatorTok{);}
    \DataTypeTok{float}\NormalTok{ b }\OperatorTok{=} \BuiltInTok{sin}\OperatorTok{(}\NormalTok{iTime}\OperatorTok{);}
    \DataTypeTok{float}\NormalTok{ c }\OperatorTok{=} \FloatTok{0.5} \OperatorTok{*} \BuiltInTok{sin}\OperatorTok{(}\NormalTok{iTime }\OperatorTok{*} \FloatTok{0.7}\OperatorTok{);}
    \DataTypeTok{float}\NormalTok{ L }\OperatorTok{=}\NormalTok{ a }\OperatorTok{*}\NormalTok{ p}\OperatorTok{.}\FunctionTok{x} \OperatorTok{+}\NormalTok{ b }\OperatorTok{*}\NormalTok{ p}\OperatorTok{.}\FunctionTok{y} \OperatorTok{+}\NormalTok{ c}\OperatorTok{;}
    
    \DataTypeTok{vec3}\NormalTok{ color}\OperatorTok{;}
    \KeywordTok{if} \OperatorTok{(}\NormalTok{L }\OperatorTok{\textless{}} \FloatTok{0.0}\OperatorTok{)} \OperatorTok{\{}
\NormalTok{        color }\OperatorTok{=} \DataTypeTok{vec3}\OperatorTok{(}\FloatTok{1.0}\OperatorTok{,} \FloatTok{0.0}\OperatorTok{,} \FloatTok{0.0}\OperatorTok{);}
    \OperatorTok{\}} \KeywordTok{else} \OperatorTok{\{}
\NormalTok{        color }\OperatorTok{=} \DataTypeTok{vec3}\OperatorTok{(}\FloatTok{0.0}\OperatorTok{,} \FloatTok{0.0}\OperatorTok{,} \FloatTok{1.0}\OperatorTok{);}
    \OperatorTok{\}}
    
\NormalTok{    fragColor }\OperatorTok{=} \DataTypeTok{vec4}\OperatorTok{(}\NormalTok{color}\OperatorTok{,} \FloatTok{1.0}\OperatorTok{);}
\OperatorTok{\}}
\end{Highlighting}
\end{Shaded}

\begin{center}\rule{0.5\linewidth}{0.5pt}\end{center}

\section{A6. circle}\label{a6.-circle}

Filled circle centered at the origin.

\begin{Shaded}
\begin{Highlighting}[]
\DataTypeTok{void} \FunctionTok{mainImage}\OperatorTok{(}\DataTypeTok{out} \DataTypeTok{vec4}\NormalTok{ fragColor}\OperatorTok{,} \DataTypeTok{in} \DataTypeTok{vec2}\NormalTok{ fragCoord}\OperatorTok{)}
\OperatorTok{\{}
    \DataTypeTok{vec2}\NormalTok{ uv }\OperatorTok{=}\NormalTok{ fragCoord }\OperatorTok{/}\NormalTok{ iResolution}\OperatorTok{.}\FunctionTok{xy}\OperatorTok{;}
\NormalTok{    uv }\OperatorTok{=}\NormalTok{ uv }\OperatorTok{{-}} \DataTypeTok{vec2}\OperatorTok{(}\FloatTok{0.5}\OperatorTok{,} \FloatTok{0.5}\OperatorTok{);}
\NormalTok{    uv}\OperatorTok{.}\FunctionTok{x} \OperatorTok{*=}\NormalTok{ iResolution}\OperatorTok{.}\FunctionTok{x} \OperatorTok{/}\NormalTok{ iResolution}\OperatorTok{.}\FunctionTok{y}\OperatorTok{;}
    \DataTypeTok{vec2}\NormalTok{ p }\OperatorTok{=}\NormalTok{ uv }\OperatorTok{*} \FloatTok{4.0}\OperatorTok{;}
    
    \DataTypeTok{float}\NormalTok{ d }\OperatorTok{=} \BuiltInTok{length}\OperatorTok{(}\NormalTok{p}\OperatorTok{);}
    \DataTypeTok{float}\NormalTok{ r }\OperatorTok{=} \FloatTok{1.0}\OperatorTok{;}
    \DataTypeTok{float}\NormalTok{ f }\OperatorTok{=}\NormalTok{ d }\OperatorTok{{-}}\NormalTok{ r}\OperatorTok{;}
    
    \DataTypeTok{vec3}\NormalTok{ color}\OperatorTok{;}
    \KeywordTok{if} \OperatorTok{(}\NormalTok{f }\OperatorTok{\textless{}} \FloatTok{0.0}\OperatorTok{)} \OperatorTok{\{}
\NormalTok{        color }\OperatorTok{=} \DataTypeTok{vec3}\OperatorTok{(}\FloatTok{1.0}\OperatorTok{,} \FloatTok{1.0}\OperatorTok{,} \FloatTok{0.0}\OperatorTok{);}  \CommentTok{// yellow inside}
    \OperatorTok{\}} \KeywordTok{else} \OperatorTok{\{}
\NormalTok{        color }\OperatorTok{=} \DataTypeTok{vec3}\OperatorTok{(}\FloatTok{0.1}\OperatorTok{,} \FloatTok{0.1}\OperatorTok{,} \FloatTok{0.3}\OperatorTok{);}  \CommentTok{// dark blue outside}
    \OperatorTok{\}}
    
\NormalTok{    fragColor }\OperatorTok{=} \DataTypeTok{vec4}\OperatorTok{(}\NormalTok{color}\OperatorTok{,} \FloatTok{1.0}\OperatorTok{);}
\OperatorTok{\}}
\end{Highlighting}
\end{Shaded}

\begin{center}\rule{0.5\linewidth}{0.5pt}\end{center}

\section{A7. circle-curve}\label{a7.-circle-curve}

Circle outline (ring) centered at the origin.

\begin{Shaded}
\begin{Highlighting}[]
\DataTypeTok{void} \FunctionTok{mainImage}\OperatorTok{(}\DataTypeTok{out} \DataTypeTok{vec4}\NormalTok{ fragColor}\OperatorTok{,} \DataTypeTok{in} \DataTypeTok{vec2}\NormalTok{ fragCoord}\OperatorTok{)}
\OperatorTok{\{}
    \DataTypeTok{vec2}\NormalTok{ uv }\OperatorTok{=}\NormalTok{ fragCoord }\OperatorTok{/}\NormalTok{ iResolution}\OperatorTok{.}\FunctionTok{xy}\OperatorTok{;}
\NormalTok{    uv }\OperatorTok{=}\NormalTok{ uv }\OperatorTok{{-}} \DataTypeTok{vec2}\OperatorTok{(}\FloatTok{0.5}\OperatorTok{,} \FloatTok{0.5}\OperatorTok{);}
\NormalTok{    uv}\OperatorTok{.}\FunctionTok{x} \OperatorTok{*=}\NormalTok{ iResolution}\OperatorTok{.}\FunctionTok{x} \OperatorTok{/}\NormalTok{ iResolution}\OperatorTok{.}\FunctionTok{y}\OperatorTok{;}
    \DataTypeTok{vec2}\NormalTok{ p }\OperatorTok{=}\NormalTok{ uv }\OperatorTok{*} \FloatTok{4.0}\OperatorTok{;}
    
    \DataTypeTok{float}\NormalTok{ d }\OperatorTok{=} \BuiltInTok{length}\OperatorTok{(}\NormalTok{p}\OperatorTok{);}
    \DataTypeTok{float}\NormalTok{ r }\OperatorTok{=} \FloatTok{1.0}\OperatorTok{;}
    \DataTypeTok{float}\NormalTok{ eps }\OperatorTok{=} \FloatTok{0.1}\OperatorTok{;}
    \DataTypeTok{float}\NormalTok{ f }\OperatorTok{=} \BuiltInTok{abs}\OperatorTok{(}\NormalTok{d }\OperatorTok{{-}}\NormalTok{ r}\OperatorTok{)} \OperatorTok{{-}}\NormalTok{ eps}\OperatorTok{;}
    
    \DataTypeTok{vec3}\NormalTok{ color}\OperatorTok{;}
    \KeywordTok{if} \OperatorTok{(}\NormalTok{f }\OperatorTok{\textless{}} \FloatTok{0.0}\OperatorTok{)} \OperatorTok{\{}
\NormalTok{        color }\OperatorTok{=} \DataTypeTok{vec3}\OperatorTok{(}\FloatTok{1.0}\OperatorTok{,} \FloatTok{1.0}\OperatorTok{,} \FloatTok{0.0}\OperatorTok{);}  \CommentTok{// yellow ring}
    \OperatorTok{\}} \KeywordTok{else} \OperatorTok{\{}
\NormalTok{        color }\OperatorTok{=} \DataTypeTok{vec3}\OperatorTok{(}\FloatTok{0.1}\OperatorTok{,} \FloatTok{0.1}\OperatorTok{,} \FloatTok{0.3}\OperatorTok{);}  \CommentTok{// dark background}
    \OperatorTok{\}}
    
\NormalTok{    fragColor }\OperatorTok{=} \DataTypeTok{vec4}\OperatorTok{(}\NormalTok{color}\OperatorTok{,} \FloatTok{1.0}\OperatorTok{);}
\OperatorTok{\}}
\end{Highlighting}
\end{Shaded}

\begin{center}\rule{0.5\linewidth}{0.5pt}\end{center}

\section{A8. parabola}\label{a8.-parabola}

The parabola \(y = x^2\) rendered as an implicit curve.

\begin{Shaded}
\begin{Highlighting}[]
\DataTypeTok{void} \FunctionTok{mainImage}\OperatorTok{(}\DataTypeTok{out} \DataTypeTok{vec4}\NormalTok{ fragColor}\OperatorTok{,} \DataTypeTok{in} \DataTypeTok{vec2}\NormalTok{ fragCoord}\OperatorTok{)}
\OperatorTok{\{}
    \DataTypeTok{vec2}\NormalTok{ uv }\OperatorTok{=}\NormalTok{ fragCoord }\OperatorTok{/}\NormalTok{ iResolution}\OperatorTok{.}\FunctionTok{xy}\OperatorTok{;}
\NormalTok{    uv }\OperatorTok{=}\NormalTok{ uv }\OperatorTok{{-}} \DataTypeTok{vec2}\OperatorTok{(}\FloatTok{0.5}\OperatorTok{,} \FloatTok{0.5}\OperatorTok{);}
\NormalTok{    uv}\OperatorTok{.}\FunctionTok{x} \OperatorTok{*=}\NormalTok{ iResolution}\OperatorTok{.}\FunctionTok{x} \OperatorTok{/}\NormalTok{ iResolution}\OperatorTok{.}\FunctionTok{y}\OperatorTok{;}
    \DataTypeTok{vec2}\NormalTok{ p }\OperatorTok{=}\NormalTok{ uv }\OperatorTok{*} \FloatTok{4.0}\OperatorTok{;}
    
    \DataTypeTok{float}\NormalTok{ F }\OperatorTok{=}\NormalTok{ p}\OperatorTok{.}\FunctionTok{y} \OperatorTok{{-}}\NormalTok{ p}\OperatorTok{.}\FunctionTok{x} \OperatorTok{*}\NormalTok{ p}\OperatorTok{.}\FunctionTok{x}\OperatorTok{;}
    \DataTypeTok{float}\NormalTok{ eps }\OperatorTok{=} \FloatTok{0.1}\OperatorTok{;}
    
    \DataTypeTok{vec3}\NormalTok{ color}\OperatorTok{;}
    \KeywordTok{if} \OperatorTok{(}\BuiltInTok{abs}\OperatorTok{(}\NormalTok{F}\OperatorTok{)} \OperatorTok{\textless{}}\NormalTok{ eps}\OperatorTok{)} \OperatorTok{\{}
\NormalTok{        color }\OperatorTok{=} \DataTypeTok{vec3}\OperatorTok{(}\FloatTok{1.0}\OperatorTok{,} \FloatTok{1.0}\OperatorTok{,} \FloatTok{0.0}\OperatorTok{);}  \CommentTok{// yellow curve}
    \OperatorTok{\}} \KeywordTok{else} \OperatorTok{\{}
\NormalTok{        color }\OperatorTok{=} \DataTypeTok{vec3}\OperatorTok{(}\FloatTok{0.1}\OperatorTok{,} \FloatTok{0.1}\OperatorTok{,} \FloatTok{0.3}\OperatorTok{);}  \CommentTok{// dark background}
    \OperatorTok{\}}
    
\NormalTok{    fragColor }\OperatorTok{=} \DataTypeTok{vec4}\OperatorTok{(}\NormalTok{color}\OperatorTok{,} \FloatTok{1.0}\OperatorTok{);}
\OperatorTok{\}}
\end{Highlighting}
\end{Shaded}

\begin{center}\rule{0.5\linewidth}{0.5pt}\end{center}

\section{A9. lemniscate-naive}\label{a9.-lemniscate-naive}

Lemniscate of Bernoulli with naive thresholding (non-uniform thickness).

\begin{Shaded}
\begin{Highlighting}[]
\DataTypeTok{void} \FunctionTok{mainImage}\OperatorTok{(}\DataTypeTok{out} \DataTypeTok{vec4}\NormalTok{ fragColor}\OperatorTok{,} \DataTypeTok{in} \DataTypeTok{vec2}\NormalTok{ fragCoord}\OperatorTok{)}
\OperatorTok{\{}
    \DataTypeTok{vec2}\NormalTok{ uv }\OperatorTok{=}\NormalTok{ fragCoord }\OperatorTok{/}\NormalTok{ iResolution}\OperatorTok{.}\FunctionTok{xy}\OperatorTok{;}
\NormalTok{    uv }\OperatorTok{=}\NormalTok{ uv }\OperatorTok{{-}} \DataTypeTok{vec2}\OperatorTok{(}\FloatTok{0.5}\OperatorTok{,} \FloatTok{0.5}\OperatorTok{);}
\NormalTok{    uv}\OperatorTok{.}\FunctionTok{x} \OperatorTok{*=}\NormalTok{ iResolution}\OperatorTok{.}\FunctionTok{x} \OperatorTok{/}\NormalTok{ iResolution}\OperatorTok{.}\FunctionTok{y}\OperatorTok{;}
    \DataTypeTok{vec2}\NormalTok{ p }\OperatorTok{=}\NormalTok{ uv }\OperatorTok{*} \FloatTok{4.0}\OperatorTok{;}
    
    \DataTypeTok{float}\NormalTok{ a }\OperatorTok{=} \FloatTok{1.5}\OperatorTok{;}
    \DataTypeTok{float}\NormalTok{ r2 }\OperatorTok{=} \BuiltInTok{dot}\OperatorTok{(}\NormalTok{p}\OperatorTok{,}\NormalTok{ p}\OperatorTok{);}
    \DataTypeTok{float}\NormalTok{ F }\OperatorTok{=}\NormalTok{ r2 }\OperatorTok{*}\NormalTok{ r2 }\OperatorTok{{-}}\NormalTok{ a }\OperatorTok{*}\NormalTok{ a }\OperatorTok{*} \OperatorTok{(}\NormalTok{p}\OperatorTok{.}\FunctionTok{x} \OperatorTok{*}\NormalTok{ p}\OperatorTok{.}\FunctionTok{x} \OperatorTok{{-}}\NormalTok{ p}\OperatorTok{.}\FunctionTok{y} \OperatorTok{*}\NormalTok{ p}\OperatorTok{.}\FunctionTok{y}\OperatorTok{);}
    
    \DataTypeTok{float}\NormalTok{ eps }\OperatorTok{=} \FloatTok{0.15}\OperatorTok{;}
    
    \DataTypeTok{vec3}\NormalTok{ color}\OperatorTok{;}
    \KeywordTok{if} \OperatorTok{(}\BuiltInTok{abs}\OperatorTok{(}\NormalTok{F}\OperatorTok{)} \OperatorTok{\textless{}}\NormalTok{ eps}\OperatorTok{)} \OperatorTok{\{}
\NormalTok{        color }\OperatorTok{=} \DataTypeTok{vec3}\OperatorTok{(}\FloatTok{1.0}\OperatorTok{,} \FloatTok{1.0}\OperatorTok{,} \FloatTok{0.0}\OperatorTok{);}
    \OperatorTok{\}} \KeywordTok{else} \OperatorTok{\{}
\NormalTok{        color }\OperatorTok{=} \DataTypeTok{vec3}\OperatorTok{(}\FloatTok{0.1}\OperatorTok{,} \FloatTok{0.1}\OperatorTok{,} \FloatTok{0.3}\OperatorTok{);}
    \OperatorTok{\}}
    
\NormalTok{    fragColor }\OperatorTok{=} \DataTypeTok{vec4}\OperatorTok{(}\NormalTok{color}\OperatorTok{,} \FloatTok{1.0}\OperatorTok{);}
\OperatorTok{\}}
\end{Highlighting}
\end{Shaded}

\begin{center}\rule{0.5\linewidth}{0.5pt}\end{center}

\section{A10. lemniscate-gradient}\label{a10.-lemniscate-gradient}

Lemniscate with gradient correction for uniform thickness.

\begin{Shaded}
\begin{Highlighting}[]
\DataTypeTok{void} \FunctionTok{mainImage}\OperatorTok{(}\DataTypeTok{out} \DataTypeTok{vec4}\NormalTok{ fragColor}\OperatorTok{,} \DataTypeTok{in} \DataTypeTok{vec2}\NormalTok{ fragCoord}\OperatorTok{)}
\OperatorTok{\{}
    \DataTypeTok{vec2}\NormalTok{ uv }\OperatorTok{=}\NormalTok{ fragCoord }\OperatorTok{/}\NormalTok{ iResolution}\OperatorTok{.}\FunctionTok{xy}\OperatorTok{;}
\NormalTok{    uv }\OperatorTok{=}\NormalTok{ uv }\OperatorTok{{-}} \DataTypeTok{vec2}\OperatorTok{(}\FloatTok{0.5}\OperatorTok{,} \FloatTok{0.5}\OperatorTok{);}
\NormalTok{    uv}\OperatorTok{.}\FunctionTok{x} \OperatorTok{*=}\NormalTok{ iResolution}\OperatorTok{.}\FunctionTok{x} \OperatorTok{/}\NormalTok{ iResolution}\OperatorTok{.}\FunctionTok{y}\OperatorTok{;}
    \DataTypeTok{vec2}\NormalTok{ p }\OperatorTok{=}\NormalTok{ uv }\OperatorTok{*} \FloatTok{4.0}\OperatorTok{;}
    
    \DataTypeTok{float}\NormalTok{ a }\OperatorTok{=} \FloatTok{1.5}\OperatorTok{;}
    \DataTypeTok{float}\NormalTok{ r2 }\OperatorTok{=} \BuiltInTok{dot}\OperatorTok{(}\NormalTok{p}\OperatorTok{,}\NormalTok{ p}\OperatorTok{);}
    \DataTypeTok{float}\NormalTok{ F }\OperatorTok{=}\NormalTok{ r2 }\OperatorTok{*}\NormalTok{ r2 }\OperatorTok{{-}}\NormalTok{ a }\OperatorTok{*}\NormalTok{ a }\OperatorTok{*} \OperatorTok{(}\NormalTok{p}\OperatorTok{.}\FunctionTok{x} \OperatorTok{*}\NormalTok{ p}\OperatorTok{.}\FunctionTok{x} \OperatorTok{{-}}\NormalTok{ p}\OperatorTok{.}\FunctionTok{y} \OperatorTok{*}\NormalTok{ p}\OperatorTok{.}\FunctionTok{y}\OperatorTok{);}
    
    \DataTypeTok{vec2}\NormalTok{ grad }\OperatorTok{=} \DataTypeTok{vec2}\OperatorTok{(}
        \FloatTok{4.0} \OperatorTok{*}\NormalTok{ p}\OperatorTok{.}\FunctionTok{x} \OperatorTok{*}\NormalTok{ r2 }\OperatorTok{{-}} \FloatTok{2.0} \OperatorTok{*}\NormalTok{ a }\OperatorTok{*}\NormalTok{ a }\OperatorTok{*}\NormalTok{ p}\OperatorTok{.}\FunctionTok{x}\OperatorTok{,}
        \FloatTok{4.0} \OperatorTok{*}\NormalTok{ p}\OperatorTok{.}\FunctionTok{y} \OperatorTok{*}\NormalTok{ r2 }\OperatorTok{+} \FloatTok{2.0} \OperatorTok{*}\NormalTok{ a }\OperatorTok{*}\NormalTok{ a }\OperatorTok{*}\NormalTok{ p}\OperatorTok{.}\FunctionTok{y}
    \OperatorTok{);}
    
    \DataTypeTok{float}\NormalTok{ dist }\OperatorTok{=} \BuiltInTok{abs}\OperatorTok{(}\NormalTok{F}\OperatorTok{)} \OperatorTok{/} \BuiltInTok{max}\OperatorTok{(}\BuiltInTok{length}\OperatorTok{(}\NormalTok{grad}\OperatorTok{),} \FloatTok{0.01}\OperatorTok{);}
    \DataTypeTok{float}\NormalTok{ eps }\OperatorTok{=} \FloatTok{0.05}\OperatorTok{;}
    
    \DataTypeTok{vec3}\NormalTok{ color}\OperatorTok{;}
    \KeywordTok{if} \OperatorTok{(}\NormalTok{dist }\OperatorTok{\textless{}}\NormalTok{ eps}\OperatorTok{)} \OperatorTok{\{}
\NormalTok{        color }\OperatorTok{=} \DataTypeTok{vec3}\OperatorTok{(}\FloatTok{1.0}\OperatorTok{,} \FloatTok{1.0}\OperatorTok{,} \FloatTok{0.0}\OperatorTok{);}
    \OperatorTok{\}} \KeywordTok{else} \OperatorTok{\{}
\NormalTok{        color }\OperatorTok{=} \DataTypeTok{vec3}\OperatorTok{(}\FloatTok{0.1}\OperatorTok{,} \FloatTok{0.1}\OperatorTok{,} \FloatTok{0.3}\OperatorTok{);}
    \OperatorTok{\}}
    
\NormalTok{    fragColor }\OperatorTok{=} \DataTypeTok{vec4}\OperatorTok{(}\NormalTok{color}\OperatorTok{,} \FloatTok{1.0}\OperatorTok{);}
\OperatorTok{\}}
\end{Highlighting}
\end{Shaded}

\begin{center}\rule{0.5\linewidth}{0.5pt}\end{center}

\section{A11. lemniscate-animated}\label{a11.-lemniscate-animated}

Cassini ovals animated through the lemniscate transition.

\begin{Shaded}
\begin{Highlighting}[]
\DataTypeTok{void} \FunctionTok{mainImage}\OperatorTok{(}\DataTypeTok{out} \DataTypeTok{vec4}\NormalTok{ fragColor}\OperatorTok{,} \DataTypeTok{in} \DataTypeTok{vec2}\NormalTok{ fragCoord}\OperatorTok{)}
\OperatorTok{\{}
    \DataTypeTok{vec2}\NormalTok{ uv }\OperatorTok{=}\NormalTok{ fragCoord }\OperatorTok{/}\NormalTok{ iResolution}\OperatorTok{.}\FunctionTok{xy}\OperatorTok{;}
\NormalTok{    uv }\OperatorTok{=}\NormalTok{ uv }\OperatorTok{{-}} \DataTypeTok{vec2}\OperatorTok{(}\FloatTok{0.5}\OperatorTok{,} \FloatTok{0.5}\OperatorTok{);}
\NormalTok{    uv}\OperatorTok{.}\FunctionTok{x} \OperatorTok{*=}\NormalTok{ iResolution}\OperatorTok{.}\FunctionTok{x} \OperatorTok{/}\NormalTok{ iResolution}\OperatorTok{.}\FunctionTok{y}\OperatorTok{;}
    \DataTypeTok{vec2}\NormalTok{ p }\OperatorTok{=}\NormalTok{ uv }\OperatorTok{*} \FloatTok{4.0}\OperatorTok{;}
    
    \CommentTok{// Cassini oval parameters}
    \DataTypeTok{float}\NormalTok{ c }\OperatorTok{=} \FloatTok{1.0}\OperatorTok{;}  \CommentTok{// half{-}distance between foci}
    \DataTypeTok{float}\NormalTok{ a }\OperatorTok{=} \FloatTok{0.8} \OperatorTok{+} \FloatTok{0.4} \OperatorTok{*} \BuiltInTok{sin}\OperatorTok{(}\NormalTok{iTime }\OperatorTok{*} \FloatTok{0.5}\OperatorTok{);}  \CommentTok{// animate through transition}
    
    \CommentTok{// Implicit equation: (x² + y²)² {-} 2c²(x² {-} y²) = a⁴ {-} c⁴}
    \DataTypeTok{float}\NormalTok{ r2 }\OperatorTok{=} \BuiltInTok{dot}\OperatorTok{(}\NormalTok{p}\OperatorTok{,}\NormalTok{ p}\OperatorTok{);}
    \DataTypeTok{float}\NormalTok{ c2 }\OperatorTok{=}\NormalTok{ c }\OperatorTok{*}\NormalTok{ c}\OperatorTok{;}
    \DataTypeTok{float}\NormalTok{ a4 }\OperatorTok{=}\NormalTok{ a }\OperatorTok{*}\NormalTok{ a }\OperatorTok{*}\NormalTok{ a }\OperatorTok{*}\NormalTok{ a}\OperatorTok{;}
    \DataTypeTok{float}\NormalTok{ c4 }\OperatorTok{=}\NormalTok{ c2 }\OperatorTok{*}\NormalTok{ c2}\OperatorTok{;}
    \DataTypeTok{float}\NormalTok{ F }\OperatorTok{=}\NormalTok{ r2 }\OperatorTok{*}\NormalTok{ r2 }\OperatorTok{{-}} \FloatTok{2.0} \OperatorTok{*}\NormalTok{ c2 }\OperatorTok{*} \OperatorTok{(}\NormalTok{p}\OperatorTok{.}\FunctionTok{x} \OperatorTok{*}\NormalTok{ p}\OperatorTok{.}\FunctionTok{x} \OperatorTok{{-}}\NormalTok{ p}\OperatorTok{.}\FunctionTok{y} \OperatorTok{*}\NormalTok{ p}\OperatorTok{.}\FunctionTok{y}\OperatorTok{)} \OperatorTok{{-}} \OperatorTok{(}\NormalTok{a4 }\OperatorTok{{-}}\NormalTok{ c4}\OperatorTok{);}
    
    \CommentTok{// Gradient}
    \DataTypeTok{vec2}\NormalTok{ grad }\OperatorTok{=} \DataTypeTok{vec2}\OperatorTok{(}
        \FloatTok{4.0} \OperatorTok{*}\NormalTok{ p}\OperatorTok{.}\FunctionTok{x} \OperatorTok{*}\NormalTok{ r2 }\OperatorTok{{-}} \FloatTok{4.0} \OperatorTok{*}\NormalTok{ c2 }\OperatorTok{*}\NormalTok{ p}\OperatorTok{.}\FunctionTok{x}\OperatorTok{,}
        \FloatTok{4.0} \OperatorTok{*}\NormalTok{ p}\OperatorTok{.}\FunctionTok{y} \OperatorTok{*}\NormalTok{ r2 }\OperatorTok{+} \FloatTok{4.0} \OperatorTok{*}\NormalTok{ c2 }\OperatorTok{*}\NormalTok{ p}\OperatorTok{.}\FunctionTok{y}
    \OperatorTok{);}
    
    \DataTypeTok{float}\NormalTok{ dist }\OperatorTok{=} \BuiltInTok{abs}\OperatorTok{(}\NormalTok{F}\OperatorTok{)} \OperatorTok{/} \BuiltInTok{max}\OperatorTok{(}\BuiltInTok{length}\OperatorTok{(}\NormalTok{grad}\OperatorTok{),} \FloatTok{0.01}\OperatorTok{);}
    \DataTypeTok{float}\NormalTok{ eps }\OperatorTok{=} \FloatTok{0.05}\OperatorTok{;}
    
    \DataTypeTok{vec3}\NormalTok{ color}\OperatorTok{;}
    \KeywordTok{if} \OperatorTok{(}\NormalTok{dist }\OperatorTok{\textless{}}\NormalTok{ eps}\OperatorTok{)} \OperatorTok{\{}
\NormalTok{        color }\OperatorTok{=} \DataTypeTok{vec3}\OperatorTok{(}\FloatTok{1.0}\OperatorTok{,} \FloatTok{1.0}\OperatorTok{,} \FloatTok{0.0}\OperatorTok{);}
    \OperatorTok{\}} \KeywordTok{else} \OperatorTok{\{}
\NormalTok{        color }\OperatorTok{=} \DataTypeTok{vec3}\OperatorTok{(}\FloatTok{0.1}\OperatorTok{,} \FloatTok{0.1}\OperatorTok{,} \FloatTok{0.3}\OperatorTok{);}
    \OperatorTok{\}}
    
\NormalTok{    fragColor }\OperatorTok{=} \DataTypeTok{vec4}\OperatorTok{(}\NormalTok{color}\OperatorTok{,} \FloatTok{1.0}\OperatorTok{);}
\OperatorTok{\}}
\end{Highlighting}
\end{Shaded}

\begin{center}\rule{0.5\linewidth}{0.5pt}\end{center}

\section{A12. circle-mouse}\label{a12.-circle-mouse}

Circle that follows the mouse position.

\begin{Shaded}
\begin{Highlighting}[]
\DataTypeTok{void} \FunctionTok{mainImage}\OperatorTok{(}\DataTypeTok{out} \DataTypeTok{vec4}\NormalTok{ fragColor}\OperatorTok{,} \DataTypeTok{in} \DataTypeTok{vec2}\NormalTok{ fragCoord}\OperatorTok{)}
\OperatorTok{\{}
    \CommentTok{// Normalize fragment coordinate}
    \DataTypeTok{vec2}\NormalTok{ uv }\OperatorTok{=}\NormalTok{ fragCoord }\OperatorTok{/}\NormalTok{ iResolution}\OperatorTok{.}\FunctionTok{xy}\OperatorTok{;}
\NormalTok{    uv }\OperatorTok{=}\NormalTok{ uv }\OperatorTok{{-}} \DataTypeTok{vec2}\OperatorTok{(}\FloatTok{0.5}\OperatorTok{,} \FloatTok{0.5}\OperatorTok{);}
\NormalTok{    uv}\OperatorTok{.}\FunctionTok{x} \OperatorTok{*=}\NormalTok{ iResolution}\OperatorTok{.}\FunctionTok{x} \OperatorTok{/}\NormalTok{ iResolution}\OperatorTok{.}\FunctionTok{y}\OperatorTok{;}
    \DataTypeTok{vec2}\NormalTok{ p }\OperatorTok{=}\NormalTok{ uv }\OperatorTok{*} \FloatTok{4.0}\OperatorTok{;}
    
    \CommentTok{// Normalize mouse coordinate the same way}
    \DataTypeTok{vec2}\NormalTok{ mouse }\OperatorTok{=}\NormalTok{ iMouse}\OperatorTok{.}\FunctionTok{xy} \OperatorTok{/}\NormalTok{ iResolution}\OperatorTok{.}\FunctionTok{xy}\OperatorTok{;}
\NormalTok{    mouse }\OperatorTok{=}\NormalTok{ mouse }\OperatorTok{{-}} \DataTypeTok{vec2}\OperatorTok{(}\FloatTok{0.5}\OperatorTok{,} \FloatTok{0.5}\OperatorTok{);}
\NormalTok{    mouse}\OperatorTok{.}\FunctionTok{x} \OperatorTok{*=}\NormalTok{ iResolution}\OperatorTok{.}\FunctionTok{x} \OperatorTok{/}\NormalTok{ iResolution}\OperatorTok{.}\FunctionTok{y}\OperatorTok{;}
\NormalTok{    mouse }\OperatorTok{=}\NormalTok{ mouse }\OperatorTok{*} \FloatTok{4.0}\OperatorTok{;}
    
    \CommentTok{// Circle centered at mouse}
    \DataTypeTok{float}\NormalTok{ d }\OperatorTok{=} \BuiltInTok{length}\OperatorTok{(}\NormalTok{p }\OperatorTok{{-}}\NormalTok{ mouse}\OperatorTok{);}
    \DataTypeTok{float}\NormalTok{ r }\OperatorTok{=} \FloatTok{0.5}\OperatorTok{;}
    
    \DataTypeTok{vec3}\NormalTok{ color}\OperatorTok{;}
    \KeywordTok{if} \OperatorTok{(}\NormalTok{d }\OperatorTok{\textless{}}\NormalTok{ r}\OperatorTok{)} \OperatorTok{\{}
\NormalTok{        color }\OperatorTok{=} \DataTypeTok{vec3}\OperatorTok{(}\FloatTok{1.0}\OperatorTok{,} \FloatTok{0.9}\OperatorTok{,} \FloatTok{0.2}\OperatorTok{);}
    \OperatorTok{\}} \KeywordTok{else} \OperatorTok{\{}
\NormalTok{        color }\OperatorTok{=} \DataTypeTok{vec3}\OperatorTok{(}\FloatTok{0.1}\OperatorTok{,} \FloatTok{0.1}\OperatorTok{,} \FloatTok{0.3}\OperatorTok{);}
    \OperatorTok{\}}
    
\NormalTok{    fragColor }\OperatorTok{=} \DataTypeTok{vec4}\OperatorTok{(}\NormalTok{color}\OperatorTok{,} \FloatTok{1.0}\OperatorTok{);}
\OperatorTok{\}}
\end{Highlighting}
\end{Shaded}

\begin{center}\rule{0.5\linewidth}{0.5pt}\end{center}

\section{A13. sun-earth}\label{a13.-sun-earth}

Sun at mouse click position with orbiting earth.

\begin{Shaded}
\begin{Highlighting}[]
\DataTypeTok{vec2} \FunctionTok{normalize\_coord}\OperatorTok{(}\DataTypeTok{vec2}\NormalTok{ coord}\OperatorTok{)} \OperatorTok{\{}
    \DataTypeTok{vec2}\NormalTok{ uv }\OperatorTok{=}\NormalTok{ coord }\OperatorTok{/}\NormalTok{ iResolution}\OperatorTok{.}\FunctionTok{xy}\OperatorTok{;}
\NormalTok{    uv }\OperatorTok{=}\NormalTok{ uv }\OperatorTok{{-}} \DataTypeTok{vec2}\OperatorTok{(}\FloatTok{0.5}\OperatorTok{,} \FloatTok{0.5}\OperatorTok{);}
\NormalTok{    uv}\OperatorTok{.}\FunctionTok{x} \OperatorTok{*=}\NormalTok{ iResolution}\OperatorTok{.}\FunctionTok{x} \OperatorTok{/}\NormalTok{ iResolution}\OperatorTok{.}\FunctionTok{y}\OperatorTok{;}
    \KeywordTok{return}\NormalTok{ uv }\OperatorTok{*} \FloatTok{4.0}\OperatorTok{;}
\OperatorTok{\}}

\DataTypeTok{void} \FunctionTok{mainImage}\OperatorTok{(}\DataTypeTok{out} \DataTypeTok{vec4}\NormalTok{ fragColor}\OperatorTok{,} \DataTypeTok{in} \DataTypeTok{vec2}\NormalTok{ fragCoord}\OperatorTok{)}
\OperatorTok{\{}
    \DataTypeTok{vec2}\NormalTok{ p }\OperatorTok{=} \FunctionTok{normalize\_coord}\OperatorTok{(}\NormalTok{fragCoord}\OperatorTok{);}
    
    \CommentTok{// Use iMouse.zw (last click position) so sun stays put}
    \DataTypeTok{vec2}\NormalTok{ sun }\OperatorTok{=} \FunctionTok{normalize\_coord}\OperatorTok{(}\NormalTok{iMouse}\OperatorTok{.}\FunctionTok{zw}\OperatorTok{);}
    
    \CommentTok{// Earth orbits the sun}
    \DataTypeTok{float}\NormalTok{ orbit\_radius }\OperatorTok{=} \FloatTok{0.8}\OperatorTok{;}
    \DataTypeTok{vec2}\NormalTok{ earth }\OperatorTok{=}\NormalTok{ sun }\OperatorTok{+}\NormalTok{ orbit\_radius }\OperatorTok{*} \DataTypeTok{vec2}\OperatorTok{(}\BuiltInTok{cos}\OperatorTok{(}\NormalTok{iTime}\OperatorTok{),} \BuiltInTok{sin}\OperatorTok{(}\NormalTok{iTime}\OperatorTok{));}
    
    \CommentTok{// Draw sun (larger, yellow)}
    \DataTypeTok{float}\NormalTok{ d\_sun }\OperatorTok{=} \BuiltInTok{length}\OperatorTok{(}\NormalTok{p }\OperatorTok{{-}}\NormalTok{ sun}\OperatorTok{);}
    \CommentTok{// Draw earth (smaller, blue)}
    \DataTypeTok{float}\NormalTok{ d\_earth }\OperatorTok{=} \BuiltInTok{length}\OperatorTok{(}\NormalTok{p }\OperatorTok{{-}}\NormalTok{ earth}\OperatorTok{);}
    
    \DataTypeTok{vec3}\NormalTok{ color }\OperatorTok{=} \DataTypeTok{vec3}\OperatorTok{(}\FloatTok{0.02}\OperatorTok{,} \FloatTok{0.02}\OperatorTok{,} \FloatTok{0.05}\OperatorTok{);}  \CommentTok{// dark background}
    \KeywordTok{if} \OperatorTok{(}\NormalTok{d\_sun }\OperatorTok{\textless{}} \FloatTok{0.3}\OperatorTok{)} \OperatorTok{\{}
\NormalTok{        color }\OperatorTok{=} \DataTypeTok{vec3}\OperatorTok{(}\FloatTok{1.0}\OperatorTok{,} \FloatTok{0.9}\OperatorTok{,} \FloatTok{0.2}\OperatorTok{);}  \CommentTok{// yellow sun}
    \OperatorTok{\}}
    \KeywordTok{if} \OperatorTok{(}\NormalTok{d\_earth }\OperatorTok{\textless{}} \FloatTok{0.15}\OperatorTok{)} \OperatorTok{\{}
\NormalTok{        color }\OperatorTok{=} \DataTypeTok{vec3}\OperatorTok{(}\FloatTok{0.2}\OperatorTok{,} \FloatTok{0.5}\OperatorTok{,} \FloatTok{1.0}\OperatorTok{);}  \CommentTok{// blue earth}
    \OperatorTok{\}}
    
\NormalTok{    fragColor }\OperatorTok{=} \DataTypeTok{vec4}\OperatorTok{(}\NormalTok{color}\OperatorTok{,} \FloatTok{1.0}\OperatorTok{);}
\OperatorTok{\}}
\end{Highlighting}
\end{Shaded}

\begin{center}\rule{0.5\linewidth}{0.5pt}\end{center}

\section{A14. folium-mouse}\label{a14.-folium-mouse}

Folium of Descartes with mouse-controlled level set.

\begin{Shaded}
\begin{Highlighting}[]
\DataTypeTok{vec2} \FunctionTok{normalize\_coord}\OperatorTok{(}\DataTypeTok{vec2}\NormalTok{ coord}\OperatorTok{)} \OperatorTok{\{}
    \DataTypeTok{vec2}\NormalTok{ uv }\OperatorTok{=}\NormalTok{ coord }\OperatorTok{/}\NormalTok{ iResolution}\OperatorTok{.}\FunctionTok{xy}\OperatorTok{;}
\NormalTok{    uv }\OperatorTok{=}\NormalTok{ uv }\OperatorTok{{-}} \DataTypeTok{vec2}\OperatorTok{(}\FloatTok{0.5}\OperatorTok{,} \FloatTok{0.5}\OperatorTok{);}
\NormalTok{    uv}\OperatorTok{.}\FunctionTok{x} \OperatorTok{*=}\NormalTok{ iResolution}\OperatorTok{.}\FunctionTok{x} \OperatorTok{/}\NormalTok{ iResolution}\OperatorTok{.}\FunctionTok{y}\OperatorTok{;}
    \KeywordTok{return}\NormalTok{ uv }\OperatorTok{*} \FloatTok{4.0}\OperatorTok{;}
\OperatorTok{\}}

\DataTypeTok{void} \FunctionTok{mainImage}\OperatorTok{(}\DataTypeTok{out} \DataTypeTok{vec4}\NormalTok{ fragColor}\OperatorTok{,} \DataTypeTok{in} \DataTypeTok{vec2}\NormalTok{ fragCoord}\OperatorTok{)}
\OperatorTok{\{}
    \DataTypeTok{vec2}\NormalTok{ p }\OperatorTok{=} \FunctionTok{normalize\_coord}\OperatorTok{(}\NormalTok{fragCoord}\OperatorTok{);}
    
    \CommentTok{// Fixed parameter a}
    \DataTypeTok{float}\NormalTok{ a }\OperatorTok{=} \FloatTok{1.5}\OperatorTok{;}
    
    \CommentTok{// Map mouse x to level set value c in [{-}2, 2]}
    \DataTypeTok{float}\NormalTok{ c }\OperatorTok{=} \BuiltInTok{mix}\OperatorTok{({-}}\FloatTok{2.0}\OperatorTok{,} \FloatTok{2.0}\OperatorTok{,}\NormalTok{ iMouse}\OperatorTok{.}\FunctionTok{x} \OperatorTok{/}\NormalTok{ iResolution}\OperatorTok{.}\FunctionTok{x}\OperatorTok{);}
    
    \CommentTok{// Folium of Descartes: x³ + y³ {-} 3axy = c}
    \DataTypeTok{float}\NormalTok{ F }\OperatorTok{=}\NormalTok{ p}\OperatorTok{.}\FunctionTok{x}\OperatorTok{*}\NormalTok{p}\OperatorTok{.}\FunctionTok{x}\OperatorTok{*}\NormalTok{p}\OperatorTok{.}\FunctionTok{x} \OperatorTok{+}\NormalTok{ p}\OperatorTok{.}\FunctionTok{y}\OperatorTok{*}\NormalTok{p}\OperatorTok{.}\FunctionTok{y}\OperatorTok{*}\NormalTok{p}\OperatorTok{.}\FunctionTok{y} \OperatorTok{{-}} \FloatTok{3.0}\OperatorTok{*}\NormalTok{a}\OperatorTok{*}\NormalTok{p}\OperatorTok{.}\FunctionTok{x}\OperatorTok{*}\NormalTok{p}\OperatorTok{.}\FunctionTok{y} \OperatorTok{{-}}\NormalTok{ c}\OperatorTok{;}
    
    \CommentTok{// Gradient: ∇F = (3x² {-} 3ay, 3y² {-} 3ax)}
    \DataTypeTok{vec2}\NormalTok{ grad }\OperatorTok{=} \DataTypeTok{vec2}\OperatorTok{(}\FloatTok{3.0}\OperatorTok{*}\NormalTok{p}\OperatorTok{.}\FunctionTok{x}\OperatorTok{*}\NormalTok{p}\OperatorTok{.}\FunctionTok{x} \OperatorTok{{-}} \FloatTok{3.0}\OperatorTok{*}\NormalTok{a}\OperatorTok{*}\NormalTok{p}\OperatorTok{.}\FunctionTok{y}\OperatorTok{,} \FloatTok{3.0}\OperatorTok{*}\NormalTok{p}\OperatorTok{.}\FunctionTok{y}\OperatorTok{*}\NormalTok{p}\OperatorTok{.}\FunctionTok{y} \OperatorTok{{-}} \FloatTok{3.0}\OperatorTok{*}\NormalTok{a}\OperatorTok{*}\NormalTok{p}\OperatorTok{.}\FunctionTok{x}\OperatorTok{);}
    \DataTypeTok{float}\NormalTok{ dist }\OperatorTok{=} \BuiltInTok{abs}\OperatorTok{(}\NormalTok{F}\OperatorTok{)} \OperatorTok{/} \BuiltInTok{max}\OperatorTok{(}\BuiltInTok{length}\OperatorTok{(}\NormalTok{grad}\OperatorTok{),} \FloatTok{0.01}\OperatorTok{);}
    
    \DataTypeTok{vec3}\NormalTok{ color}\OperatorTok{;}
    \KeywordTok{if} \OperatorTok{(}\NormalTok{dist }\OperatorTok{\textless{}} \FloatTok{0.05}\OperatorTok{)} \OperatorTok{\{}
\NormalTok{        color }\OperatorTok{=} \DataTypeTok{vec3}\OperatorTok{(}\FloatTok{1.0}\OperatorTok{,} \FloatTok{1.0}\OperatorTok{,} \FloatTok{0.0}\OperatorTok{);}
    \OperatorTok{\}} \KeywordTok{else} \OperatorTok{\{}
\NormalTok{        color }\OperatorTok{=} \DataTypeTok{vec3}\OperatorTok{(}\FloatTok{0.1}\OperatorTok{,} \FloatTok{0.1}\OperatorTok{,} \FloatTok{0.3}\OperatorTok{);}
    \OperatorTok{\}}
    
\NormalTok{    fragColor }\OperatorTok{=} \DataTypeTok{vec4}\OperatorTok{(}\NormalTok{color}\OperatorTok{,} \FloatTok{1.0}\OperatorTok{);}
\OperatorTok{\}}
\end{Highlighting}
\end{Shaded}

\begin{center}\rule{0.5\linewidth}{0.5pt}\end{center}

\section{A15. elliptic-family}\label{a15.-elliptic-family}

Family of elliptic curves with mouse-controlled center parameters.

\begin{Shaded}
\begin{Highlighting}[]
\DataTypeTok{vec2} \FunctionTok{normalize\_coord}\OperatorTok{(}\DataTypeTok{vec2}\NormalTok{ coord}\OperatorTok{)} \OperatorTok{\{}
    \DataTypeTok{vec2}\NormalTok{ uv }\OperatorTok{=}\NormalTok{ coord }\OperatorTok{/}\NormalTok{ iResolution}\OperatorTok{.}\FunctionTok{xy}\OperatorTok{;}
\NormalTok{    uv }\OperatorTok{=}\NormalTok{ uv }\OperatorTok{{-}} \DataTypeTok{vec2}\OperatorTok{(}\FloatTok{0.5}\OperatorTok{,} \FloatTok{0.5}\OperatorTok{);}
\NormalTok{    uv}\OperatorTok{.}\FunctionTok{x} \OperatorTok{*=}\NormalTok{ iResolution}\OperatorTok{.}\FunctionTok{x} \OperatorTok{/}\NormalTok{ iResolution}\OperatorTok{.}\FunctionTok{y}\OperatorTok{;}
    \KeywordTok{return}\NormalTok{ uv }\OperatorTok{*} \FloatTok{4.0}\OperatorTok{;}
\OperatorTok{\}}

\DataTypeTok{void} \FunctionTok{mainImage}\OperatorTok{(}\DataTypeTok{out} \DataTypeTok{vec4}\NormalTok{ fragColor}\OperatorTok{,} \DataTypeTok{in} \DataTypeTok{vec2}\NormalTok{ fragCoord}\OperatorTok{)}
\OperatorTok{\{}
    \DataTypeTok{vec2}\NormalTok{ p }\OperatorTok{=} \FunctionTok{normalize\_coord}\OperatorTok{(}\NormalTok{fragCoord}\OperatorTok{);}
    
    \CommentTok{// Mouse controls the central (a, b) of our family}
    \DataTypeTok{float}\NormalTok{ a\_center }\OperatorTok{=} \BuiltInTok{mix}\OperatorTok{({-}}\FloatTok{3.0}\OperatorTok{,} \FloatTok{1.0}\OperatorTok{,}\NormalTok{ iMouse}\OperatorTok{.}\FunctionTok{x} \OperatorTok{/}\NormalTok{ iResolution}\OperatorTok{.}\FunctionTok{x}\OperatorTok{);}
    \DataTypeTok{float}\NormalTok{ b\_center }\OperatorTok{=} \BuiltInTok{mix}\OperatorTok{({-}}\FloatTok{2.0}\OperatorTok{,} \FloatTok{2.0}\OperatorTok{,}\NormalTok{ iMouse}\OperatorTok{.}\FunctionTok{y} \OperatorTok{/}\NormalTok{ iResolution}\OperatorTok{.}\FunctionTok{y}\OperatorTok{);}
    
    \DataTypeTok{vec3}\NormalTok{ color }\OperatorTok{=} \DataTypeTok{vec3}\OperatorTok{(}\FloatTok{0.05}\OperatorTok{,} \FloatTok{0.05}\OperatorTok{,} \FloatTok{0.1}\OperatorTok{);}  \CommentTok{// dark background}
    
    \CommentTok{// Draw curves for a range of a values around a\_center}
    \KeywordTok{for} \OperatorTok{(}\DataTypeTok{float}\NormalTok{ i }\OperatorTok{=} \OperatorTok{{-}}\FloatTok{3.0}\OperatorTok{;}\NormalTok{ i }\OperatorTok{\textless{}=} \FloatTok{3.0}\OperatorTok{;}\NormalTok{ i }\OperatorTok{+=} \FloatTok{1.0}\OperatorTok{)} \OperatorTok{\{}
        \DataTypeTok{float}\NormalTok{ a }\OperatorTok{=}\NormalTok{ a\_center }\OperatorTok{+}\NormalTok{ i }\OperatorTok{*} \FloatTok{0.5}\OperatorTok{;}  \CommentTok{// more spacing}
        \DataTypeTok{float}\NormalTok{ b }\OperatorTok{=}\NormalTok{ b\_center}\OperatorTok{;}
        
        \CommentTok{// Elliptic curve: y² = x³ + ax + b}
        \DataTypeTok{float}\NormalTok{ F }\OperatorTok{=}\NormalTok{ p}\OperatorTok{.}\FunctionTok{y} \OperatorTok{*}\NormalTok{ p}\OperatorTok{.}\FunctionTok{y} \OperatorTok{{-}}\NormalTok{ p}\OperatorTok{.}\FunctionTok{x} \OperatorTok{*}\NormalTok{ p}\OperatorTok{.}\FunctionTok{x} \OperatorTok{*}\NormalTok{ p}\OperatorTok{.}\FunctionTok{x} \OperatorTok{{-}}\NormalTok{ a }\OperatorTok{*}\NormalTok{ p}\OperatorTok{.}\FunctionTok{x} \OperatorTok{{-}}\NormalTok{ b}\OperatorTok{;}
        
        \CommentTok{// Gradient}
        \DataTypeTok{vec2}\NormalTok{ grad }\OperatorTok{=} \DataTypeTok{vec2}\OperatorTok{({-}}\FloatTok{3.0} \OperatorTok{*}\NormalTok{ p}\OperatorTok{.}\FunctionTok{x} \OperatorTok{*}\NormalTok{ p}\OperatorTok{.}\FunctionTok{x} \OperatorTok{{-}}\NormalTok{ a}\OperatorTok{,} \FloatTok{2.0} \OperatorTok{*}\NormalTok{ p}\OperatorTok{.}\FunctionTok{y}\OperatorTok{);}
        \DataTypeTok{float}\NormalTok{ dist }\OperatorTok{=} \BuiltInTok{abs}\OperatorTok{(}\NormalTok{F}\OperatorTok{)} \OperatorTok{/} \BuiltInTok{max}\OperatorTok{(}\BuiltInTok{length}\OperatorTok{(}\NormalTok{grad}\OperatorTok{),} \FloatTok{0.01}\OperatorTok{);}
        
        \CommentTok{// Brightness fades quickly: central curve bright, outer curves fade to background}
        \DataTypeTok{float}\NormalTok{ t }\OperatorTok{=} \BuiltInTok{abs}\OperatorTok{(}\NormalTok{i}\OperatorTok{)} \OperatorTok{/} \FloatTok{3.0}\OperatorTok{;}  \CommentTok{// 0 at center, 1 at edges}
        \DataTypeTok{float}\NormalTok{ brightness }\OperatorTok{=} \FloatTok{1.0} \OperatorTok{{-}}\NormalTok{ t }\OperatorTok{*}\NormalTok{ t}\OperatorTok{;}  \CommentTok{// quadratic falloff}
        
        \KeywordTok{if} \OperatorTok{(}\NormalTok{dist }\OperatorTok{\textless{}} \FloatTok{0.03} \OperatorTok{\&\&}\NormalTok{ brightness }\OperatorTok{\textgreater{}} \FloatTok{0.05}\OperatorTok{)} \OperatorTok{\{}
            \CommentTok{// Check discriminant for this specific curve}
            \DataTypeTok{float}\NormalTok{ disc }\OperatorTok{=} \FloatTok{4.0} \OperatorTok{*}\NormalTok{ a }\OperatorTok{*}\NormalTok{ a }\OperatorTok{*}\NormalTok{ a }\OperatorTok{+} \FloatTok{27.0} \OperatorTok{*}\NormalTok{ b }\OperatorTok{*}\NormalTok{ b}\OperatorTok{;}
            \KeywordTok{if} \OperatorTok{(}\BuiltInTok{abs}\OperatorTok{(}\NormalTok{disc}\OperatorTok{)} \OperatorTok{\textless{}} \FloatTok{0.3}\OperatorTok{)} \OperatorTok{\{}
\NormalTok{                color }\OperatorTok{=} \BuiltInTok{mix}\OperatorTok{(}\NormalTok{color}\OperatorTok{,} \DataTypeTok{vec3}\OperatorTok{(}\FloatTok{1.0}\OperatorTok{,} \FloatTok{0.3}\OperatorTok{,} \FloatTok{0.3}\OperatorTok{),}\NormalTok{ brightness}\OperatorTok{);}  \CommentTok{// red for singular}
            \OperatorTok{\}} \KeywordTok{else} \OperatorTok{\{}
\NormalTok{                color }\OperatorTok{=} \BuiltInTok{mix}\OperatorTok{(}\NormalTok{color}\OperatorTok{,} \DataTypeTok{vec3}\OperatorTok{(}\FloatTok{1.0}\OperatorTok{,} \FloatTok{1.0}\OperatorTok{,} \FloatTok{0.5}\OperatorTok{),}\NormalTok{ brightness}\OperatorTok{);}  \CommentTok{// yellow for smooth}
            \OperatorTok{\}}
        \OperatorTok{\}}
    \OperatorTok{\}}
    
\NormalTok{    fragColor }\OperatorTok{=} \DataTypeTok{vec4}\OperatorTok{(}\NormalTok{color}\OperatorTok{,} \FloatTok{1.0}\OperatorTok{);}
\OperatorTok{\}}
\end{Highlighting}
\end{Shaded}

\begin{center}\rule{0.5\linewidth}{0.5pt}\end{center}

\section{A16. grid-circles}\label{a16.-grid-circles}

Grid of circles with square cells.

\begin{Shaded}
\begin{Highlighting}[]
\DataTypeTok{void} \FunctionTok{mainImage}\OperatorTok{(}\DataTypeTok{out} \DataTypeTok{vec4}\NormalTok{ fragColor}\OperatorTok{,} \DataTypeTok{in} \DataTypeTok{vec2}\NormalTok{ fragCoord}\OperatorTok{)}
\OperatorTok{\{}
    \DataTypeTok{vec2}\NormalTok{ uv }\OperatorTok{=}\NormalTok{ fragCoord }\OperatorTok{/}\NormalTok{ iResolution}\OperatorTok{.}\FunctionTok{xy}\OperatorTok{;}
\NormalTok{    uv }\OperatorTok{=}\NormalTok{ uv }\OperatorTok{{-}} \DataTypeTok{vec2}\OperatorTok{(}\FloatTok{0.5}\OperatorTok{,} \FloatTok{0.5}\OperatorTok{);}
\NormalTok{    uv}\OperatorTok{.}\FunctionTok{x} \OperatorTok{*=}\NormalTok{ iResolution}\OperatorTok{.}\FunctionTok{x} \OperatorTok{/}\NormalTok{ iResolution}\OperatorTok{.}\FunctionTok{y}\OperatorTok{;}
    \DataTypeTok{vec2}\NormalTok{ p }\OperatorTok{=}\NormalTok{ uv }\OperatorTok{*} \FloatTok{4.0}\OperatorTok{;}
    
    \DataTypeTok{float}\NormalTok{ aspect }\OperatorTok{=}\NormalTok{ iResolution}\OperatorTok{.}\FunctionTok{x} \OperatorTok{/}\NormalTok{ iResolution}\OperatorTok{.}\FunctionTok{y}\OperatorTok{;}
    \DataTypeTok{float}\NormalTok{ N }\OperatorTok{=} \FloatTok{5.0}\OperatorTok{;}  \CommentTok{// number of columns}
    \DataTypeTok{float}\NormalTok{ L }\OperatorTok{=} \OperatorTok{(}\FloatTok{4.0} \OperatorTok{*}\NormalTok{ aspect}\OperatorTok{)} \OperatorTok{/}\NormalTok{ N}\OperatorTok{;}  \CommentTok{// cell size}
    
    \DataTypeTok{vec2}\NormalTok{ cell\_id }\OperatorTok{=} \BuiltInTok{floor}\OperatorTok{(}\NormalTok{p }\OperatorTok{/}\NormalTok{ L}\OperatorTok{);}
    \DataTypeTok{vec2}\NormalTok{ cell\_p }\OperatorTok{=} \BuiltInTok{mod}\OperatorTok{(}\NormalTok{p }\OperatorTok{+} \DataTypeTok{vec2}\OperatorTok{(}\NormalTok{L}\OperatorTok{/}\FloatTok{2.0}\OperatorTok{,}\NormalTok{ L}\OperatorTok{/}\FloatTok{2.0}\OperatorTok{),}\NormalTok{ L}\OperatorTok{)} \OperatorTok{{-}} \DataTypeTok{vec2}\OperatorTok{(}\NormalTok{L}\OperatorTok{/}\FloatTok{2.0}\OperatorTok{,}\NormalTok{ L}\OperatorTok{/}\FloatTok{2.0}\OperatorTok{);}
    
    \CommentTok{// Checkerboard background}
    \DataTypeTok{float}\NormalTok{ checker }\OperatorTok{=} \BuiltInTok{mod}\OperatorTok{(}\NormalTok{cell\_id}\OperatorTok{.}\FunctionTok{x} \OperatorTok{+}\NormalTok{ cell\_id}\OperatorTok{.}\FunctionTok{y}\OperatorTok{,} \FloatTok{2.0}\OperatorTok{);}
    \DataTypeTok{vec3}\NormalTok{ bg }\OperatorTok{=} \BuiltInTok{mix}\OperatorTok{(}\DataTypeTok{vec3}\OperatorTok{(}\FloatTok{0.15}\OperatorTok{,} \FloatTok{0.15}\OperatorTok{,} \FloatTok{0.25}\OperatorTok{),} \DataTypeTok{vec3}\OperatorTok{(}\FloatTok{0.25}\OperatorTok{,} \FloatTok{0.15}\OperatorTok{,} \FloatTok{0.15}\OperatorTok{),}\NormalTok{ checker}\OperatorTok{);}
    
    \CommentTok{// Circle in each cell}
    \DataTypeTok{float}\NormalTok{ d }\OperatorTok{=} \BuiltInTok{length}\OperatorTok{(}\NormalTok{cell\_p}\OperatorTok{);}
    \DataTypeTok{float}\NormalTok{ r }\OperatorTok{=}\NormalTok{ L }\OperatorTok{*} \FloatTok{0.35}\OperatorTok{;}
    
    \DataTypeTok{vec3}\NormalTok{ color}\OperatorTok{;}
    \KeywordTok{if} \OperatorTok{(}\NormalTok{d }\OperatorTok{\textless{}}\NormalTok{ r}\OperatorTok{)} \OperatorTok{\{}
\NormalTok{        color }\OperatorTok{=} \DataTypeTok{vec3}\OperatorTok{(}\FloatTok{1.0}\OperatorTok{,} \FloatTok{1.0}\OperatorTok{,} \FloatTok{0.0}\OperatorTok{);}
    \OperatorTok{\}} \KeywordTok{else} \OperatorTok{\{}
\NormalTok{        color }\OperatorTok{=}\NormalTok{ bg}\OperatorTok{;}
    \OperatorTok{\}}
    
\NormalTok{    fragColor }\OperatorTok{=} \DataTypeTok{vec4}\OperatorTok{(}\NormalTok{color}\OperatorTok{,} \FloatTok{1.0}\OperatorTok{);}
\OperatorTok{\}}
\end{Highlighting}
\end{Shaded}

\begin{center}\rule{0.5\linewidth}{0.5pt}\end{center}

\section{Notes}\label{notes}

\subsection{Coordinate Setup}\label{coordinate-setup}

Most shaders use this standard coordinate setup:

\begin{Shaded}
\begin{Highlighting}[]
\DataTypeTok{vec2}\NormalTok{ uv }\OperatorTok{=}\NormalTok{ fragCoord }\OperatorTok{/}\NormalTok{ iResolution}\OperatorTok{.}\FunctionTok{xy}\OperatorTok{;}   \CommentTok{// normalize to [0,1]}
\NormalTok{uv }\OperatorTok{=}\NormalTok{ uv }\OperatorTok{{-}} \DataTypeTok{vec2}\OperatorTok{(}\FloatTok{0.5}\OperatorTok{,} \FloatTok{0.5}\OperatorTok{);}               \CommentTok{// center origin}
\NormalTok{uv}\OperatorTok{.}\FunctionTok{x} \OperatorTok{*=}\NormalTok{ iResolution}\OperatorTok{.}\FunctionTok{x} \OperatorTok{/}\NormalTok{ iResolution}\OperatorTok{.}\FunctionTok{y}\OperatorTok{;}  \CommentTok{// aspect correction}
\DataTypeTok{vec2}\NormalTok{ p }\OperatorTok{=}\NormalTok{ uv }\OperatorTok{*} \FloatTok{4.0}\OperatorTok{;}                      \CommentTok{// scale to [{-}2, 2] range}
\end{Highlighting}
\end{Shaded}

\subsection{Helper Function}\label{helper-function}

For shaders using mouse input, we define:

\begin{Shaded}
\begin{Highlighting}[]
\DataTypeTok{vec2} \FunctionTok{normalize\_coord}\OperatorTok{(}\DataTypeTok{vec2}\NormalTok{ coord}\OperatorTok{)} \OperatorTok{\{}
    \DataTypeTok{vec2}\NormalTok{ uv }\OperatorTok{=}\NormalTok{ coord }\OperatorTok{/}\NormalTok{ iResolution}\OperatorTok{.}\FunctionTok{xy}\OperatorTok{;}
\NormalTok{    uv }\OperatorTok{=}\NormalTok{ uv }\OperatorTok{{-}} \DataTypeTok{vec2}\OperatorTok{(}\FloatTok{0.5}\OperatorTok{,} \FloatTok{0.5}\OperatorTok{);}
\NormalTok{    uv}\OperatorTok{.}\FunctionTok{x} \OperatorTok{*=}\NormalTok{ iResolution}\OperatorTok{.}\FunctionTok{x} \OperatorTok{/}\NormalTok{ iResolution}\OperatorTok{.}\FunctionTok{y}\OperatorTok{;}
    \KeywordTok{return}\NormalTok{ uv }\OperatorTok{*} \FloatTok{4.0}\OperatorTok{;}
\OperatorTok{\}}
\end{Highlighting}
\end{Shaded}

\subsection{Gradient Correction}\label{gradient-correction-1}

For implicit curves \(F(x,y) = 0\) with uniform thickness:

\begin{Shaded}
\begin{Highlighting}[]
\DataTypeTok{float}\NormalTok{ dist }\OperatorTok{=} \BuiltInTok{abs}\OperatorTok{(}\NormalTok{F}\OperatorTok{)} \OperatorTok{/} \BuiltInTok{max}\OperatorTok{(}\BuiltInTok{length}\OperatorTok{(}\NormalTok{grad}\OperatorTok{),} \FloatTok{0.01}\OperatorTok{);}
\end{Highlighting}
\end{Shaded}

where \texttt{grad} is \(\nabla F\) computed analytically.

\chapter{Appendix: Day 1 Homework
Solutions}\label{appendix-day-1-homework-solutions}

\begin{center}\rule{0.5\linewidth}{0.5pt}\end{center}

\section{Checkpoints}\label{checkpoints-1}

\subsection{C1. Solid Colors}\label{c1.-solid-colors}

\textbf{(a) Green:}

\begin{Shaded}
\begin{Highlighting}[]
\DataTypeTok{void} \FunctionTok{mainImage}\OperatorTok{(}\DataTypeTok{out} \DataTypeTok{vec4}\NormalTok{ fragColor}\OperatorTok{,} \DataTypeTok{in} \DataTypeTok{vec2}\NormalTok{ fragCoord}\OperatorTok{)}
\OperatorTok{\{}
\NormalTok{    fragColor }\OperatorTok{=} \DataTypeTok{vec4}\OperatorTok{(}\FloatTok{0.0}\OperatorTok{,} \FloatTok{1.0}\OperatorTok{,} \FloatTok{0.0}\OperatorTok{,} \FloatTok{1.0}\OperatorTok{);}
\OperatorTok{\}}
\end{Highlighting}
\end{Shaded}

\textbf{(b) Cyan:}

\begin{Shaded}
\begin{Highlighting}[]
\DataTypeTok{void} \FunctionTok{mainImage}\OperatorTok{(}\DataTypeTok{out} \DataTypeTok{vec4}\NormalTok{ fragColor}\OperatorTok{,} \DataTypeTok{in} \DataTypeTok{vec2}\NormalTok{ fragCoord}\OperatorTok{)}
\OperatorTok{\{}
\NormalTok{    fragColor }\OperatorTok{=} \DataTypeTok{vec4}\OperatorTok{(}\FloatTok{0.0}\OperatorTok{,} \FloatTok{1.0}\OperatorTok{,} \FloatTok{1.0}\OperatorTok{,} \FloatTok{1.0}\OperatorTok{);}
\OperatorTok{\}}
\end{Highlighting}
\end{Shaded}

\textbf{(c) Custom color (example: orange):}

\begin{Shaded}
\begin{Highlighting}[]
\DataTypeTok{void} \FunctionTok{mainImage}\OperatorTok{(}\DataTypeTok{out} \DataTypeTok{vec4}\NormalTok{ fragColor}\OperatorTok{,} \DataTypeTok{in} \DataTypeTok{vec2}\NormalTok{ fragCoord}\OperatorTok{)}
\OperatorTok{\{}
\NormalTok{    fragColor }\OperatorTok{=} \DataTypeTok{vec4}\OperatorTok{(}\FloatTok{1.0}\OperatorTok{,} \FloatTok{0.5}\OperatorTok{,} \FloatTok{0.2}\OperatorTok{,} \FloatTok{1.0}\OperatorTok{);}
\OperatorTok{\}}
\end{Highlighting}
\end{Shaded}

\begin{center}\rule{0.5\linewidth}{0.5pt}\end{center}

\subsection{C2. Vertical Split}\label{c2.-vertical-split}

\begin{Shaded}
\begin{Highlighting}[]
\DataTypeTok{void} \FunctionTok{mainImage}\OperatorTok{(}\DataTypeTok{out} \DataTypeTok{vec4}\NormalTok{ fragColor}\OperatorTok{,} \DataTypeTok{in} \DataTypeTok{vec2}\NormalTok{ fragCoord}\OperatorTok{)}
\OperatorTok{\{}
    \DataTypeTok{vec2}\NormalTok{ uv }\OperatorTok{=}\NormalTok{ fragCoord }\OperatorTok{/}\NormalTok{ iResolution}\OperatorTok{.}\FunctionTok{xy}\OperatorTok{;}
\NormalTok{    uv }\OperatorTok{=}\NormalTok{ uv }\OperatorTok{{-}} \DataTypeTok{vec2}\OperatorTok{(}\FloatTok{0.5}\OperatorTok{,} \FloatTok{0.5}\OperatorTok{);}
\NormalTok{    uv}\OperatorTok{.}\FunctionTok{x} \OperatorTok{*=}\NormalTok{ iResolution}\OperatorTok{.}\FunctionTok{x} \OperatorTok{/}\NormalTok{ iResolution}\OperatorTok{.}\FunctionTok{y}\OperatorTok{;}
    \DataTypeTok{vec2}\NormalTok{ p }\OperatorTok{=}\NormalTok{ uv }\OperatorTok{*} \FloatTok{4.0}\OperatorTok{;}
    
    \DataTypeTok{float}\NormalTok{ L }\OperatorTok{=}\NormalTok{ p}\OperatorTok{.}\FunctionTok{x}\OperatorTok{;}  \CommentTok{// changed from p.y}
    
    \DataTypeTok{vec3}\NormalTok{ color}\OperatorTok{;}
    \KeywordTok{if} \OperatorTok{(}\NormalTok{L }\OperatorTok{\textless{}} \FloatTok{0.0}\OperatorTok{)} \OperatorTok{\{}
\NormalTok{        color }\OperatorTok{=} \DataTypeTok{vec3}\OperatorTok{(}\FloatTok{1.0}\OperatorTok{,} \FloatTok{0.0}\OperatorTok{,} \FloatTok{0.0}\OperatorTok{);}  \CommentTok{// red on left}
    \OperatorTok{\}} \KeywordTok{else} \OperatorTok{\{}
\NormalTok{        color }\OperatorTok{=} \DataTypeTok{vec3}\OperatorTok{(}\FloatTok{0.0}\OperatorTok{,} \FloatTok{0.0}\OperatorTok{,} \FloatTok{1.0}\OperatorTok{);}  \CommentTok{// blue on right}
    \OperatorTok{\}}
    
\NormalTok{    fragColor }\OperatorTok{=} \DataTypeTok{vec4}\OperatorTok{(}\NormalTok{color}\OperatorTok{,} \FloatTok{1.0}\OperatorTok{);}
\OperatorTok{\}}
\end{Highlighting}
\end{Shaded}

\begin{center}\rule{0.5\linewidth}{0.5pt}\end{center}

\subsection{C3. Off-Center Circle}\label{c3.-off-center-circle}

\begin{Shaded}
\begin{Highlighting}[]
\DataTypeTok{void} \FunctionTok{mainImage}\OperatorTok{(}\DataTypeTok{out} \DataTypeTok{vec4}\NormalTok{ fragColor}\OperatorTok{,} \DataTypeTok{in} \DataTypeTok{vec2}\NormalTok{ fragCoord}\OperatorTok{)}
\OperatorTok{\{}
    \DataTypeTok{vec2}\NormalTok{ uv }\OperatorTok{=}\NormalTok{ fragCoord }\OperatorTok{/}\NormalTok{ iResolution}\OperatorTok{.}\FunctionTok{xy}\OperatorTok{;}
\NormalTok{    uv }\OperatorTok{=}\NormalTok{ uv }\OperatorTok{{-}} \DataTypeTok{vec2}\OperatorTok{(}\FloatTok{0.5}\OperatorTok{,} \FloatTok{0.5}\OperatorTok{);}
\NormalTok{    uv}\OperatorTok{.}\FunctionTok{x} \OperatorTok{*=}\NormalTok{ iResolution}\OperatorTok{.}\FunctionTok{x} \OperatorTok{/}\NormalTok{ iResolution}\OperatorTok{.}\FunctionTok{y}\OperatorTok{;}
    \DataTypeTok{vec2}\NormalTok{ p }\OperatorTok{=}\NormalTok{ uv }\OperatorTok{*} \FloatTok{4.0}\OperatorTok{;}
    
    \DataTypeTok{vec2}\NormalTok{ center }\OperatorTok{=} \DataTypeTok{vec2}\OperatorTok{(}\FloatTok{1.0}\OperatorTok{,} \FloatTok{1.0}\OperatorTok{);}
    \DataTypeTok{float}\NormalTok{ d }\OperatorTok{=} \BuiltInTok{length}\OperatorTok{(}\NormalTok{p }\OperatorTok{{-}}\NormalTok{ center}\OperatorTok{);}
    \DataTypeTok{float}\NormalTok{ r }\OperatorTok{=} \FloatTok{0.5}\OperatorTok{;}
    
    \DataTypeTok{vec3}\NormalTok{ color}\OperatorTok{;}
    \KeywordTok{if} \OperatorTok{(}\NormalTok{d }\OperatorTok{\textless{}}\NormalTok{ r}\OperatorTok{)} \OperatorTok{\{}
\NormalTok{        color }\OperatorTok{=} \DataTypeTok{vec3}\OperatorTok{(}\FloatTok{1.0}\OperatorTok{,} \FloatTok{1.0}\OperatorTok{,} \FloatTok{0.0}\OperatorTok{);}
    \OperatorTok{\}} \KeywordTok{else} \OperatorTok{\{}
\NormalTok{        color }\OperatorTok{=} \DataTypeTok{vec3}\OperatorTok{(}\FloatTok{0.1}\OperatorTok{,} \FloatTok{0.1}\OperatorTok{,} \FloatTok{0.3}\OperatorTok{);}
    \OperatorTok{\}}
    
\NormalTok{    fragColor }\OperatorTok{=} \DataTypeTok{vec4}\OperatorTok{(}\NormalTok{color}\OperatorTok{,} \FloatTok{1.0}\OperatorTok{);}
\OperatorTok{\}}
\end{Highlighting}
\end{Shaded}

\begin{center}\rule{0.5\linewidth}{0.5pt}\end{center}

\subsection{C4. Pulsing Circle}\label{c4.-pulsing-circle}

\begin{Shaded}
\begin{Highlighting}[]
\DataTypeTok{void} \FunctionTok{mainImage}\OperatorTok{(}\DataTypeTok{out} \DataTypeTok{vec4}\NormalTok{ fragColor}\OperatorTok{,} \DataTypeTok{in} \DataTypeTok{vec2}\NormalTok{ fragCoord}\OperatorTok{)}
\OperatorTok{\{}
    \DataTypeTok{vec2}\NormalTok{ uv }\OperatorTok{=}\NormalTok{ fragCoord }\OperatorTok{/}\NormalTok{ iResolution}\OperatorTok{.}\FunctionTok{xy}\OperatorTok{;}
\NormalTok{    uv }\OperatorTok{=}\NormalTok{ uv }\OperatorTok{{-}} \DataTypeTok{vec2}\OperatorTok{(}\FloatTok{0.5}\OperatorTok{,} \FloatTok{0.5}\OperatorTok{);}
\NormalTok{    uv}\OperatorTok{.}\FunctionTok{x} \OperatorTok{*=}\NormalTok{ iResolution}\OperatorTok{.}\FunctionTok{x} \OperatorTok{/}\NormalTok{ iResolution}\OperatorTok{.}\FunctionTok{y}\OperatorTok{;}
    \DataTypeTok{vec2}\NormalTok{ p }\OperatorTok{=}\NormalTok{ uv }\OperatorTok{*} \FloatTok{4.0}\OperatorTok{;}
    
    \DataTypeTok{float}\NormalTok{ d }\OperatorTok{=} \BuiltInTok{length}\OperatorTok{(}\NormalTok{p}\OperatorTok{);}
    \DataTypeTok{float}\NormalTok{ r }\OperatorTok{=} \FloatTok{1.0} \OperatorTok{+} \FloatTok{0.5} \OperatorTok{*} \BuiltInTok{sin}\OperatorTok{(}\NormalTok{iTime}\OperatorTok{);}  \CommentTok{// oscillates between 0.5 and 1.5}
    
    \DataTypeTok{vec3}\NormalTok{ color}\OperatorTok{;}
    \KeywordTok{if} \OperatorTok{(}\NormalTok{d }\OperatorTok{\textless{}}\NormalTok{ r}\OperatorTok{)} \OperatorTok{\{}
\NormalTok{        color }\OperatorTok{=} \DataTypeTok{vec3}\OperatorTok{(}\FloatTok{1.0}\OperatorTok{,} \FloatTok{1.0}\OperatorTok{,} \FloatTok{0.0}\OperatorTok{);}
    \OperatorTok{\}} \KeywordTok{else} \OperatorTok{\{}
\NormalTok{        color }\OperatorTok{=} \DataTypeTok{vec3}\OperatorTok{(}\FloatTok{0.1}\OperatorTok{,} \FloatTok{0.1}\OperatorTok{,} \FloatTok{0.3}\OperatorTok{);}
    \OperatorTok{\}}
    
\NormalTok{    fragColor }\OperatorTok{=} \DataTypeTok{vec4}\OperatorTok{(}\NormalTok{color}\OperatorTok{,} \FloatTok{1.0}\OperatorTok{);}
\OperatorTok{\}}
\end{Highlighting}
\end{Shaded}

\begin{center}\rule{0.5\linewidth}{0.5pt}\end{center}

\subsection{C5. Ring Thickness}\label{c5.-ring-thickness}

\begin{Shaded}
\begin{Highlighting}[]
\DataTypeTok{void} \FunctionTok{mainImage}\OperatorTok{(}\DataTypeTok{out} \DataTypeTok{vec4}\NormalTok{ fragColor}\OperatorTok{,} \DataTypeTok{in} \DataTypeTok{vec2}\NormalTok{ fragCoord}\OperatorTok{)}
\OperatorTok{\{}
    \DataTypeTok{vec2}\NormalTok{ uv }\OperatorTok{=}\NormalTok{ fragCoord }\OperatorTok{/}\NormalTok{ iResolution}\OperatorTok{.}\FunctionTok{xy}\OperatorTok{;}
\NormalTok{    uv }\OperatorTok{=}\NormalTok{ uv }\OperatorTok{{-}} \DataTypeTok{vec2}\OperatorTok{(}\FloatTok{0.5}\OperatorTok{,} \FloatTok{0.5}\OperatorTok{);}
\NormalTok{    uv}\OperatorTok{.}\FunctionTok{x} \OperatorTok{*=}\NormalTok{ iResolution}\OperatorTok{.}\FunctionTok{x} \OperatorTok{/}\NormalTok{ iResolution}\OperatorTok{.}\FunctionTok{y}\OperatorTok{;}
    \DataTypeTok{vec2}\NormalTok{ p }\OperatorTok{=}\NormalTok{ uv }\OperatorTok{*} \FloatTok{4.0}\OperatorTok{;}
    
    \DataTypeTok{float}\NormalTok{ d }\OperatorTok{=} \BuiltInTok{length}\OperatorTok{(}\NormalTok{p}\OperatorTok{);}
    \DataTypeTok{float}\NormalTok{ r }\OperatorTok{=} \FloatTok{1.0}\OperatorTok{;}
    \DataTypeTok{float}\NormalTok{ eps }\OperatorTok{=} \FloatTok{0.1}\OperatorTok{;}  \CommentTok{// try 0.02, 0.05, 0.1, 0.2, 0.5}
    
    \DataTypeTok{vec3}\NormalTok{ color}\OperatorTok{;}
    \KeywordTok{if} \OperatorTok{(}\BuiltInTok{abs}\OperatorTok{(}\NormalTok{d }\OperatorTok{{-}}\NormalTok{ r}\OperatorTok{)} \OperatorTok{\textless{}}\NormalTok{ eps}\OperatorTok{)} \OperatorTok{\{}
\NormalTok{        color }\OperatorTok{=} \DataTypeTok{vec3}\OperatorTok{(}\FloatTok{1.0}\OperatorTok{,} \FloatTok{1.0}\OperatorTok{,} \FloatTok{0.0}\OperatorTok{);}  \CommentTok{// yellow ring}
    \OperatorTok{\}} \KeywordTok{else} \OperatorTok{\{}
\NormalTok{        color }\OperatorTok{=} \DataTypeTok{vec3}\OperatorTok{(}\FloatTok{0.1}\OperatorTok{,} \FloatTok{0.1}\OperatorTok{,} \FloatTok{0.3}\OperatorTok{);}
    \OperatorTok{\}}
    
\NormalTok{    fragColor }\OperatorTok{=} \DataTypeTok{vec4}\OperatorTok{(}\NormalTok{color}\OperatorTok{,} \FloatTok{1.0}\OperatorTok{);}
\OperatorTok{\}}
\end{Highlighting}
\end{Shaded}

\textbf{Teaching note:} Students should observe that \texttt{eps}
controls the visual thickness of the ring. Smaller values give thinner
rings; larger values give thicker rings. The ring has total width
\(2\varepsilon\).

\begin{center}\rule{0.5\linewidth}{0.5pt}\end{center}

\section{Explorations}\label{explorations-1}

\subsection{E1. Concentric Rings}\label{e1.-concentric-rings}

\textbf{Step 1: Draw a few rings manually}

\begin{Shaded}
\begin{Highlighting}[]
\DataTypeTok{void} \FunctionTok{mainImage}\OperatorTok{(}\DataTypeTok{out} \DataTypeTok{vec4}\NormalTok{ fragColor}\OperatorTok{,} \DataTypeTok{in} \DataTypeTok{vec2}\NormalTok{ fragCoord}\OperatorTok{)}
\OperatorTok{\{}
    \DataTypeTok{vec2}\NormalTok{ uv }\OperatorTok{=}\NormalTok{ fragCoord }\OperatorTok{/}\NormalTok{ iResolution}\OperatorTok{.}\FunctionTok{xy}\OperatorTok{;}
\NormalTok{    uv }\OperatorTok{=}\NormalTok{ uv }\OperatorTok{{-}} \DataTypeTok{vec2}\OperatorTok{(}\FloatTok{0.5}\OperatorTok{,} \FloatTok{0.5}\OperatorTok{);}
\NormalTok{    uv}\OperatorTok{.}\FunctionTok{x} \OperatorTok{*=}\NormalTok{ iResolution}\OperatorTok{.}\FunctionTok{x} \OperatorTok{/}\NormalTok{ iResolution}\OperatorTok{.}\FunctionTok{y}\OperatorTok{;}
    \DataTypeTok{vec2}\NormalTok{ p }\OperatorTok{=}\NormalTok{ uv }\OperatorTok{*} \FloatTok{4.0}\OperatorTok{;}
    
    \DataTypeTok{float}\NormalTok{ d }\OperatorTok{=} \BuiltInTok{length}\OperatorTok{(}\NormalTok{p}\OperatorTok{);}
    \DataTypeTok{float}\NormalTok{ eps }\OperatorTok{=} \FloatTok{0.05}\OperatorTok{;}
    
    \DataTypeTok{vec3}\NormalTok{ color }\OperatorTok{=} \DataTypeTok{vec3}\OperatorTok{(}\FloatTok{0.1}\OperatorTok{,} \FloatTok{0.1}\OperatorTok{,} \FloatTok{0.3}\OperatorTok{);}  \CommentTok{// background}
    
    \CommentTok{// Draw rings at r = 0.5, 1.0, 1.5, 2.0}
    \KeywordTok{if} \OperatorTok{(}\BuiltInTok{abs}\OperatorTok{(}\NormalTok{d }\OperatorTok{{-}} \FloatTok{0.5}\OperatorTok{)} \OperatorTok{\textless{}}\NormalTok{ eps}\OperatorTok{)}\NormalTok{ color }\OperatorTok{=} \DataTypeTok{vec3}\OperatorTok{(}\FloatTok{1.0}\OperatorTok{,} \FloatTok{1.0}\OperatorTok{,} \FloatTok{0.0}\OperatorTok{);}
    \KeywordTok{if} \OperatorTok{(}\BuiltInTok{abs}\OperatorTok{(}\NormalTok{d }\OperatorTok{{-}} \FloatTok{1.0}\OperatorTok{)} \OperatorTok{\textless{}}\NormalTok{ eps}\OperatorTok{)}\NormalTok{ color }\OperatorTok{=} \DataTypeTok{vec3}\OperatorTok{(}\FloatTok{1.0}\OperatorTok{,} \FloatTok{1.0}\OperatorTok{,} \FloatTok{0.0}\OperatorTok{);}
    \KeywordTok{if} \OperatorTok{(}\BuiltInTok{abs}\OperatorTok{(}\NormalTok{d }\OperatorTok{{-}} \FloatTok{1.5}\OperatorTok{)} \OperatorTok{\textless{}}\NormalTok{ eps}\OperatorTok{)}\NormalTok{ color }\OperatorTok{=} \DataTypeTok{vec3}\OperatorTok{(}\FloatTok{1.0}\OperatorTok{,} \FloatTok{1.0}\OperatorTok{,} \FloatTok{0.0}\OperatorTok{);}
    \KeywordTok{if} \OperatorTok{(}\BuiltInTok{abs}\OperatorTok{(}\NormalTok{d }\OperatorTok{{-}} \FloatTok{2.0}\OperatorTok{)} \OperatorTok{\textless{}}\NormalTok{ eps}\OperatorTok{)}\NormalTok{ color }\OperatorTok{=} \DataTypeTok{vec3}\OperatorTok{(}\FloatTok{1.0}\OperatorTok{,} \FloatTok{1.0}\OperatorTok{,} \FloatTok{0.0}\OperatorTok{);}
    
\NormalTok{    fragColor }\OperatorTok{=} \DataTypeTok{vec4}\OperatorTok{(}\NormalTok{color}\OperatorTok{,} \FloatTok{1.0}\OperatorTok{);}
\OperatorTok{\}}
\end{Highlighting}
\end{Shaded}

\textbf{Step 2: Use a for loop}

\begin{Shaded}
\begin{Highlighting}[]
\DataTypeTok{void} \FunctionTok{mainImage}\OperatorTok{(}\DataTypeTok{out} \DataTypeTok{vec4}\NormalTok{ fragColor}\OperatorTok{,} \DataTypeTok{in} \DataTypeTok{vec2}\NormalTok{ fragCoord}\OperatorTok{)}
\OperatorTok{\{}
    \DataTypeTok{vec2}\NormalTok{ uv }\OperatorTok{=}\NormalTok{ fragCoord }\OperatorTok{/}\NormalTok{ iResolution}\OperatorTok{.}\FunctionTok{xy}\OperatorTok{;}
\NormalTok{    uv }\OperatorTok{=}\NormalTok{ uv }\OperatorTok{{-}} \DataTypeTok{vec2}\OperatorTok{(}\FloatTok{0.5}\OperatorTok{,} \FloatTok{0.5}\OperatorTok{);}
\NormalTok{    uv}\OperatorTok{.}\FunctionTok{x} \OperatorTok{*=}\NormalTok{ iResolution}\OperatorTok{.}\FunctionTok{x} \OperatorTok{/}\NormalTok{ iResolution}\OperatorTok{.}\FunctionTok{y}\OperatorTok{;}
    \DataTypeTok{vec2}\NormalTok{ p }\OperatorTok{=}\NormalTok{ uv }\OperatorTok{*} \FloatTok{4.0}\OperatorTok{;}
    
    \DataTypeTok{float}\NormalTok{ d }\OperatorTok{=} \BuiltInTok{length}\OperatorTok{(}\NormalTok{p}\OperatorTok{);}
    \DataTypeTok{float}\NormalTok{ eps }\OperatorTok{=} \FloatTok{0.05}\OperatorTok{;}
    
    \DataTypeTok{vec3}\NormalTok{ color }\OperatorTok{=} \DataTypeTok{vec3}\OperatorTok{(}\FloatTok{0.1}\OperatorTok{,} \FloatTok{0.1}\OperatorTok{,} \FloatTok{0.3}\OperatorTok{);}  \CommentTok{// background}
    
    \CommentTok{// Draw rings at r = 0.5, 1.0, 1.5, 2.0}
    \KeywordTok{for} \OperatorTok{(}\DataTypeTok{float}\NormalTok{ r }\OperatorTok{=} \FloatTok{0.5}\OperatorTok{;}\NormalTok{ r }\OperatorTok{\textless{}=} \FloatTok{2.0}\OperatorTok{;}\NormalTok{ r }\OperatorTok{+=} \FloatTok{0.5}\OperatorTok{)} \OperatorTok{\{}
        \KeywordTok{if} \OperatorTok{(}\BuiltInTok{abs}\OperatorTok{(}\NormalTok{d }\OperatorTok{{-}}\NormalTok{ r}\OperatorTok{)} \OperatorTok{\textless{}}\NormalTok{ eps}\OperatorTok{)} \OperatorTok{\{}
\NormalTok{            color }\OperatorTok{=} \DataTypeTok{vec3}\OperatorTok{(}\FloatTok{1.0}\OperatorTok{,} \FloatTok{1.0}\OperatorTok{,} \FloatTok{0.0}\OperatorTok{);}
        \OperatorTok{\}}
    \OperatorTok{\}}
    
\NormalTok{    fragColor }\OperatorTok{=} \DataTypeTok{vec4}\OperatorTok{(}\NormalTok{color}\OperatorTok{,} \FloatTok{1.0}\OperatorTok{);}
\OperatorTok{\}}
\end{Highlighting}
\end{Shaded}

\textbf{Step 3: Alternating colors with mod}

The \texttt{mod(x,\ y)} function returns the remainder when \texttt{x}
is divided by \texttt{y}. So \texttt{mod(i,\ 2.0)} alternates between 0
and 1 as \texttt{i} increases. We can use this to alternate colors:

\begin{Shaded}
\begin{Highlighting}[]
\DataTypeTok{void} \FunctionTok{mainImage}\OperatorTok{(}\DataTypeTok{out} \DataTypeTok{vec4}\NormalTok{ fragColor}\OperatorTok{,} \DataTypeTok{in} \DataTypeTok{vec2}\NormalTok{ fragCoord}\OperatorTok{)}
\OperatorTok{\{}
    \DataTypeTok{vec2}\NormalTok{ uv }\OperatorTok{=}\NormalTok{ fragCoord }\OperatorTok{/}\NormalTok{ iResolution}\OperatorTok{.}\FunctionTok{xy}\OperatorTok{;}
\NormalTok{    uv }\OperatorTok{=}\NormalTok{ uv }\OperatorTok{{-}} \DataTypeTok{vec2}\OperatorTok{(}\FloatTok{0.5}\OperatorTok{,} \FloatTok{0.5}\OperatorTok{);}
\NormalTok{    uv}\OperatorTok{.}\FunctionTok{x} \OperatorTok{*=}\NormalTok{ iResolution}\OperatorTok{.}\FunctionTok{x} \OperatorTok{/}\NormalTok{ iResolution}\OperatorTok{.}\FunctionTok{y}\OperatorTok{;}
    \DataTypeTok{vec2}\NormalTok{ p }\OperatorTok{=}\NormalTok{ uv }\OperatorTok{*} \FloatTok{4.0}\OperatorTok{;}
    
    \DataTypeTok{float}\NormalTok{ d }\OperatorTok{=} \BuiltInTok{length}\OperatorTok{(}\NormalTok{p}\OperatorTok{);}
    \DataTypeTok{float}\NormalTok{ eps }\OperatorTok{=} \FloatTok{0.05}\OperatorTok{;}
    
    \DataTypeTok{vec3}\NormalTok{ color }\OperatorTok{=} \DataTypeTok{vec3}\OperatorTok{(}\FloatTok{0.1}\OperatorTok{,} \FloatTok{0.1}\OperatorTok{,} \FloatTok{0.3}\OperatorTok{);}  \CommentTok{// background}
    
    \DataTypeTok{float}\NormalTok{ i }\OperatorTok{=} \FloatTok{0.0}\OperatorTok{;}
    \KeywordTok{for} \OperatorTok{(}\DataTypeTok{float}\NormalTok{ r }\OperatorTok{=} \FloatTok{0.5}\OperatorTok{;}\NormalTok{ r }\OperatorTok{\textless{}=} \FloatTok{2.0}\OperatorTok{;}\NormalTok{ r }\OperatorTok{+=} \FloatTok{0.5}\OperatorTok{)} \OperatorTok{\{}
        \KeywordTok{if} \OperatorTok{(}\BuiltInTok{abs}\OperatorTok{(}\NormalTok{d }\OperatorTok{{-}}\NormalTok{ r}\OperatorTok{)} \OperatorTok{\textless{}}\NormalTok{ eps}\OperatorTok{)} \OperatorTok{\{}
            \KeywordTok{if} \OperatorTok{(}\BuiltInTok{mod}\OperatorTok{(}\NormalTok{i}\OperatorTok{,} \FloatTok{2.0}\OperatorTok{)} \OperatorTok{\textless{}} \FloatTok{1.0}\OperatorTok{)} \OperatorTok{\{}
\NormalTok{                color }\OperatorTok{=} \DataTypeTok{vec3}\OperatorTok{(}\FloatTok{1.0}\OperatorTok{,} \FloatTok{1.0}\OperatorTok{,} \FloatTok{0.0}\OperatorTok{);}  \CommentTok{// yellow}
            \OperatorTok{\}} \KeywordTok{else} \OperatorTok{\{}
\NormalTok{                color }\OperatorTok{=} \DataTypeTok{vec3}\OperatorTok{(}\FloatTok{0.0}\OperatorTok{,} \FloatTok{1.0}\OperatorTok{,} \FloatTok{1.0}\OperatorTok{);}  \CommentTok{// cyan}
            \OperatorTok{\}}
        \OperatorTok{\}}
\NormalTok{        i }\OperatorTok{+=} \FloatTok{1.0}\OperatorTok{;}
    \OperatorTok{\}}
    
\NormalTok{    fragColor }\OperatorTok{=} \DataTypeTok{vec4}\OperatorTok{(}\NormalTok{color}\OperatorTok{,} \FloatTok{1.0}\OperatorTok{);}
\OperatorTok{\}}
\end{Highlighting}
\end{Shaded}

\begin{center}\rule{0.5\linewidth}{0.5pt}\end{center}

\subsection{E2. Moon Orbit}\label{e2.-moon-orbit}

\begin{Shaded}
\begin{Highlighting}[]
\DataTypeTok{vec2} \FunctionTok{normalize\_coord}\OperatorTok{(}\DataTypeTok{vec2}\NormalTok{ coord}\OperatorTok{)} \OperatorTok{\{}
    \DataTypeTok{vec2}\NormalTok{ uv }\OperatorTok{=}\NormalTok{ coord }\OperatorTok{/}\NormalTok{ iResolution}\OperatorTok{.}\FunctionTok{xy}\OperatorTok{;}
\NormalTok{    uv }\OperatorTok{=}\NormalTok{ uv }\OperatorTok{{-}} \DataTypeTok{vec2}\OperatorTok{(}\FloatTok{0.5}\OperatorTok{,} \FloatTok{0.5}\OperatorTok{);}
\NormalTok{    uv}\OperatorTok{.}\FunctionTok{x} \OperatorTok{*=}\NormalTok{ iResolution}\OperatorTok{.}\FunctionTok{x} \OperatorTok{/}\NormalTok{ iResolution}\OperatorTok{.}\FunctionTok{y}\OperatorTok{;}
    \KeywordTok{return}\NormalTok{ uv }\OperatorTok{*} \FloatTok{4.0}\OperatorTok{;}
\OperatorTok{\}}

\DataTypeTok{void} \FunctionTok{mainImage}\OperatorTok{(}\DataTypeTok{out} \DataTypeTok{vec4}\NormalTok{ fragColor}\OperatorTok{,} \DataTypeTok{in} \DataTypeTok{vec2}\NormalTok{ fragCoord}\OperatorTok{)}
\OperatorTok{\{}
    \DataTypeTok{vec2}\NormalTok{ p }\OperatorTok{=} \FunctionTok{normalize\_coord}\OperatorTok{(}\NormalTok{fragCoord}\OperatorTok{);}
    \DataTypeTok{vec2}\NormalTok{ sun }\OperatorTok{=} \FunctionTok{normalize\_coord}\OperatorTok{(}\NormalTok{iMouse}\OperatorTok{.}\FunctionTok{zw}\OperatorTok{);}
    
    \CommentTok{// Earth orbits the sun}
    \DataTypeTok{float}\NormalTok{ earth\_orbit }\OperatorTok{=} \FloatTok{1.2}\OperatorTok{;}
    \DataTypeTok{float}\NormalTok{ earth\_speed }\OperatorTok{=} \FloatTok{1.0}\OperatorTok{;}
    \DataTypeTok{vec2}\NormalTok{ earth }\OperatorTok{=}\NormalTok{ sun }\OperatorTok{+}\NormalTok{ earth\_orbit }\OperatorTok{*} \DataTypeTok{vec2}\OperatorTok{(}\BuiltInTok{cos}\OperatorTok{(}\NormalTok{iTime }\OperatorTok{*}\NormalTok{ earth\_speed}\OperatorTok{),} \BuiltInTok{sin}\OperatorTok{(}\NormalTok{iTime }\OperatorTok{*}\NormalTok{ earth\_speed}\OperatorTok{));}
    
    \CommentTok{// Moon orbits the earth (smaller orbit, faster speed)}
    \DataTypeTok{float}\NormalTok{ moon\_orbit }\OperatorTok{=} \FloatTok{0.25}\OperatorTok{;}
    \DataTypeTok{float}\NormalTok{ moon\_speed }\OperatorTok{=} \FloatTok{3.0}\OperatorTok{;}
    \DataTypeTok{vec2}\NormalTok{ moon }\OperatorTok{=}\NormalTok{ earth }\OperatorTok{+}\NormalTok{ moon\_orbit }\OperatorTok{*} \DataTypeTok{vec2}\OperatorTok{(}\BuiltInTok{cos}\OperatorTok{(}\NormalTok{iTime }\OperatorTok{*}\NormalTok{ moon\_speed}\OperatorTok{),} \BuiltInTok{sin}\OperatorTok{(}\NormalTok{iTime }\OperatorTok{*}\NormalTok{ moon\_speed}\OperatorTok{));}
    
    \CommentTok{// Distances}
    \DataTypeTok{float}\NormalTok{ d\_sun }\OperatorTok{=} \BuiltInTok{length}\OperatorTok{(}\NormalTok{p }\OperatorTok{{-}}\NormalTok{ sun}\OperatorTok{);}
    \DataTypeTok{float}\NormalTok{ d\_earth }\OperatorTok{=} \BuiltInTok{length}\OperatorTok{(}\NormalTok{p }\OperatorTok{{-}}\NormalTok{ earth}\OperatorTok{);}
    \DataTypeTok{float}\NormalTok{ d\_moon }\OperatorTok{=} \BuiltInTok{length}\OperatorTok{(}\NormalTok{p }\OperatorTok{{-}}\NormalTok{ moon}\OperatorTok{);}
    
    \CommentTok{// Draw (back to front: sun, earth, moon)}
    \DataTypeTok{vec3}\NormalTok{ color }\OperatorTok{=} \DataTypeTok{vec3}\OperatorTok{(}\FloatTok{0.02}\OperatorTok{,} \FloatTok{0.02}\OperatorTok{,} \FloatTok{0.05}\OperatorTok{);}
    
    \KeywordTok{if} \OperatorTok{(}\NormalTok{d\_sun }\OperatorTok{\textless{}} \FloatTok{0.3}\OperatorTok{)} \OperatorTok{\{}
\NormalTok{        color }\OperatorTok{=} \DataTypeTok{vec3}\OperatorTok{(}\FloatTok{1.0}\OperatorTok{,} \FloatTok{0.9}\OperatorTok{,} \FloatTok{0.2}\OperatorTok{);}  \CommentTok{// yellow sun}
    \OperatorTok{\}}
    \KeywordTok{if} \OperatorTok{(}\NormalTok{d\_earth }\OperatorTok{\textless{}} \FloatTok{0.15}\OperatorTok{)} \OperatorTok{\{}
\NormalTok{        color }\OperatorTok{=} \DataTypeTok{vec3}\OperatorTok{(}\FloatTok{0.2}\OperatorTok{,} \FloatTok{0.5}\OperatorTok{,} \FloatTok{1.0}\OperatorTok{);}  \CommentTok{// blue earth}
    \OperatorTok{\}}
    \KeywordTok{if} \OperatorTok{(}\NormalTok{d\_moon }\OperatorTok{\textless{}} \FloatTok{0.08}\OperatorTok{)} \OperatorTok{\{}
\NormalTok{        color }\OperatorTok{=} \DataTypeTok{vec3}\OperatorTok{(}\FloatTok{0.8}\OperatorTok{,} \FloatTok{0.8}\OperatorTok{,} \FloatTok{0.8}\OperatorTok{);}  \CommentTok{// gray moon}
    \OperatorTok{\}}
    
\NormalTok{    fragColor }\OperatorTok{=} \DataTypeTok{vec4}\OperatorTok{(}\NormalTok{color}\OperatorTok{,} \FloatTok{1.0}\OperatorTok{);}
\OperatorTok{\}}
\end{Highlighting}
\end{Shaded}

\textbf{Extension: Solar system with more planets}

\begin{Shaded}
\begin{Highlighting}[]
\DataTypeTok{vec2} \FunctionTok{normalize\_coord}\OperatorTok{(}\DataTypeTok{vec2}\NormalTok{ coord}\OperatorTok{)} \OperatorTok{\{}
    \DataTypeTok{vec2}\NormalTok{ uv }\OperatorTok{=}\NormalTok{ coord }\OperatorTok{/}\NormalTok{ iResolution}\OperatorTok{.}\FunctionTok{xy}\OperatorTok{;}
\NormalTok{    uv }\OperatorTok{=}\NormalTok{ uv }\OperatorTok{{-}} \DataTypeTok{vec2}\OperatorTok{(}\FloatTok{0.5}\OperatorTok{,} \FloatTok{0.5}\OperatorTok{);}
\NormalTok{    uv}\OperatorTok{.}\FunctionTok{x} \OperatorTok{*=}\NormalTok{ iResolution}\OperatorTok{.}\FunctionTok{x} \OperatorTok{/}\NormalTok{ iResolution}\OperatorTok{.}\FunctionTok{y}\OperatorTok{;}
    \KeywordTok{return}\NormalTok{ uv }\OperatorTok{*} \FloatTok{4.0}\OperatorTok{;}
\OperatorTok{\}}

\DataTypeTok{void} \FunctionTok{mainImage}\OperatorTok{(}\DataTypeTok{out} \DataTypeTok{vec4}\NormalTok{ fragColor}\OperatorTok{,} \DataTypeTok{in} \DataTypeTok{vec2}\NormalTok{ fragCoord}\OperatorTok{)}
\OperatorTok{\{}
    \DataTypeTok{vec2}\NormalTok{ p }\OperatorTok{=} \FunctionTok{normalize\_coord}\OperatorTok{(}\NormalTok{fragCoord}\OperatorTok{);}
    \DataTypeTok{vec2}\NormalTok{ sun }\OperatorTok{=} \DataTypeTok{vec2}\OperatorTok{(}\FloatTok{0.0}\OperatorTok{,} \FloatTok{0.0}\OperatorTok{);}  \CommentTok{// sun at center}
    
    \CommentTok{// Planet parameters: orbit radius, speed, size, color}
    \DataTypeTok{vec3}\NormalTok{ color }\OperatorTok{=} \DataTypeTok{vec3}\OperatorTok{(}\FloatTok{0.02}\OperatorTok{,} \FloatTok{0.02}\OperatorTok{,} \FloatTok{0.05}\OperatorTok{);}
    
    \CommentTok{// Sun}
    \KeywordTok{if} \OperatorTok{(}\BuiltInTok{length}\OperatorTok{(}\NormalTok{p }\OperatorTok{{-}}\NormalTok{ sun}\OperatorTok{)} \OperatorTok{\textless{}} \FloatTok{0.25}\OperatorTok{)} \OperatorTok{\{}
\NormalTok{        color }\OperatorTok{=} \DataTypeTok{vec3}\OperatorTok{(}\FloatTok{1.0}\OperatorTok{,} \FloatTok{0.9}\OperatorTok{,} \FloatTok{0.2}\OperatorTok{);}
    \OperatorTok{\}}
    
    \CommentTok{// Mercury}
    \DataTypeTok{vec2}\NormalTok{ mercury }\OperatorTok{=}\NormalTok{ sun }\OperatorTok{+} \FloatTok{0.5} \OperatorTok{*} \DataTypeTok{vec2}\OperatorTok{(}\BuiltInTok{cos}\OperatorTok{(}\NormalTok{iTime }\OperatorTok{*} \FloatTok{4.0}\OperatorTok{),} \BuiltInTok{sin}\OperatorTok{(}\NormalTok{iTime }\OperatorTok{*} \FloatTok{4.0}\OperatorTok{));}
    \KeywordTok{if} \OperatorTok{(}\BuiltInTok{length}\OperatorTok{(}\NormalTok{p }\OperatorTok{{-}}\NormalTok{ mercury}\OperatorTok{)} \OperatorTok{\textless{}} \FloatTok{0.05}\OperatorTok{)} \OperatorTok{\{}
\NormalTok{        color }\OperatorTok{=} \DataTypeTok{vec3}\OperatorTok{(}\FloatTok{0.7}\OperatorTok{,} \FloatTok{0.7}\OperatorTok{,} \FloatTok{0.7}\OperatorTok{);}
    \OperatorTok{\}}
    
    \CommentTok{// Venus}
    \DataTypeTok{vec2}\NormalTok{ venus }\OperatorTok{=}\NormalTok{ sun }\OperatorTok{+} \FloatTok{0.8} \OperatorTok{*} \DataTypeTok{vec2}\OperatorTok{(}\BuiltInTok{cos}\OperatorTok{(}\NormalTok{iTime }\OperatorTok{*} \FloatTok{2.5}\OperatorTok{),} \BuiltInTok{sin}\OperatorTok{(}\NormalTok{iTime }\OperatorTok{*} \FloatTok{2.5}\OperatorTok{));}
    \KeywordTok{if} \OperatorTok{(}\BuiltInTok{length}\OperatorTok{(}\NormalTok{p }\OperatorTok{{-}}\NormalTok{ venus}\OperatorTok{)} \OperatorTok{\textless{}} \FloatTok{0.08}\OperatorTok{)} \OperatorTok{\{}
\NormalTok{        color }\OperatorTok{=} \DataTypeTok{vec3}\OperatorTok{(}\FloatTok{0.9}\OperatorTok{,} \FloatTok{0.7}\OperatorTok{,} \FloatTok{0.4}\OperatorTok{);}
    \OperatorTok{\}}
    
    \CommentTok{// Earth with moon}
    \DataTypeTok{vec2}\NormalTok{ earth }\OperatorTok{=}\NormalTok{ sun }\OperatorTok{+} \FloatTok{1.2} \OperatorTok{*} \DataTypeTok{vec2}\OperatorTok{(}\BuiltInTok{cos}\OperatorTok{(}\NormalTok{iTime }\OperatorTok{*} \FloatTok{1.5}\OperatorTok{),} \BuiltInTok{sin}\OperatorTok{(}\NormalTok{iTime }\OperatorTok{*} \FloatTok{1.5}\OperatorTok{));}
    \DataTypeTok{vec2}\NormalTok{ moon }\OperatorTok{=}\NormalTok{ earth }\OperatorTok{+} \FloatTok{0.15} \OperatorTok{*} \DataTypeTok{vec2}\OperatorTok{(}\BuiltInTok{cos}\OperatorTok{(}\NormalTok{iTime }\OperatorTok{*} \FloatTok{5.0}\OperatorTok{),} \BuiltInTok{sin}\OperatorTok{(}\NormalTok{iTime }\OperatorTok{*} \FloatTok{5.0}\OperatorTok{));}
    \KeywordTok{if} \OperatorTok{(}\BuiltInTok{length}\OperatorTok{(}\NormalTok{p }\OperatorTok{{-}}\NormalTok{ earth}\OperatorTok{)} \OperatorTok{\textless{}} \FloatTok{0.1}\OperatorTok{)} \OperatorTok{\{}
\NormalTok{        color }\OperatorTok{=} \DataTypeTok{vec3}\OperatorTok{(}\FloatTok{0.2}\OperatorTok{,} \FloatTok{0.5}\OperatorTok{,} \FloatTok{1.0}\OperatorTok{);}
    \OperatorTok{\}}
    \KeywordTok{if} \OperatorTok{(}\BuiltInTok{length}\OperatorTok{(}\NormalTok{p }\OperatorTok{{-}}\NormalTok{ moon}\OperatorTok{)} \OperatorTok{\textless{}} \FloatTok{0.04}\OperatorTok{)} \OperatorTok{\{}
\NormalTok{        color }\OperatorTok{=} \DataTypeTok{vec3}\OperatorTok{(}\FloatTok{0.8}\OperatorTok{,} \FloatTok{0.8}\OperatorTok{,} \FloatTok{0.8}\OperatorTok{);}
    \OperatorTok{\}}
    
    \CommentTok{// Mars}
    \DataTypeTok{vec2}\NormalTok{ mars }\OperatorTok{=}\NormalTok{ sun }\OperatorTok{+} \FloatTok{1.6} \OperatorTok{*} \DataTypeTok{vec2}\OperatorTok{(}\BuiltInTok{cos}\OperatorTok{(}\NormalTok{iTime }\OperatorTok{*} \FloatTok{1.0}\OperatorTok{),} \BuiltInTok{sin}\OperatorTok{(}\NormalTok{iTime }\OperatorTok{*} \FloatTok{1.0}\OperatorTok{));}
    \KeywordTok{if} \OperatorTok{(}\BuiltInTok{length}\OperatorTok{(}\NormalTok{p }\OperatorTok{{-}}\NormalTok{ mars}\OperatorTok{)} \OperatorTok{\textless{}} \FloatTok{0.07}\OperatorTok{)} \OperatorTok{\{}
\NormalTok{        color }\OperatorTok{=} \DataTypeTok{vec3}\OperatorTok{(}\FloatTok{0.9}\OperatorTok{,} \FloatTok{0.4}\OperatorTok{,} \FloatTok{0.2}\OperatorTok{);}
    \OperatorTok{\}}
    
\NormalTok{    fragColor }\OperatorTok{=} \DataTypeTok{vec4}\OperatorTok{(}\NormalTok{color}\OperatorTok{,} \FloatTok{1.0}\OperatorTok{);}
\OperatorTok{\}}
\end{Highlighting}
\end{Shaded}

\begin{center}\rule{0.5\linewidth}{0.5pt}\end{center}

\subsection{E3. Your Favorite Curve}\label{e3.-your-favorite-curve}

\textbf{Example: Cardioid} \((x^2 + y^2 - ax)^2 = a^2(x^2 + y^2)\)

\begin{Shaded}
\begin{Highlighting}[]
\DataTypeTok{void} \FunctionTok{mainImage}\OperatorTok{(}\DataTypeTok{out} \DataTypeTok{vec4}\NormalTok{ fragColor}\OperatorTok{,} \DataTypeTok{in} \DataTypeTok{vec2}\NormalTok{ fragCoord}\OperatorTok{)}
\OperatorTok{\{}
    \DataTypeTok{vec2}\NormalTok{ uv }\OperatorTok{=}\NormalTok{ fragCoord }\OperatorTok{/}\NormalTok{ iResolution}\OperatorTok{.}\FunctionTok{xy}\OperatorTok{;}
\NormalTok{    uv }\OperatorTok{=}\NormalTok{ uv }\OperatorTok{{-}} \DataTypeTok{vec2}\OperatorTok{(}\FloatTok{0.5}\OperatorTok{,} \FloatTok{0.5}\OperatorTok{);}
\NormalTok{    uv}\OperatorTok{.}\FunctionTok{x} \OperatorTok{*=}\NormalTok{ iResolution}\OperatorTok{.}\FunctionTok{x} \OperatorTok{/}\NormalTok{ iResolution}\OperatorTok{.}\FunctionTok{y}\OperatorTok{;}
    \DataTypeTok{vec2}\NormalTok{ p }\OperatorTok{=}\NormalTok{ uv }\OperatorTok{*} \FloatTok{4.0}\OperatorTok{;}
    
    \DataTypeTok{float}\NormalTok{ a }\OperatorTok{=} \FloatTok{1.0}\OperatorTok{;}
    \DataTypeTok{float}\NormalTok{ r2 }\OperatorTok{=} \BuiltInTok{dot}\OperatorTok{(}\NormalTok{p}\OperatorTok{,}\NormalTok{ p}\OperatorTok{);}
    
    \CommentTok{// (x² + y² {-} ax)² = a²(x² + y²)}
    \CommentTok{// F = (r² {-} ax)² {-} a²r²}
    \DataTypeTok{float}\NormalTok{ lhs }\OperatorTok{=}\NormalTok{ r2 }\OperatorTok{{-}}\NormalTok{ a }\OperatorTok{*}\NormalTok{ p}\OperatorTok{.}\FunctionTok{x}\OperatorTok{;}
    \DataTypeTok{float}\NormalTok{ F }\OperatorTok{=}\NormalTok{ lhs }\OperatorTok{*}\NormalTok{ lhs }\OperatorTok{{-}}\NormalTok{ a }\OperatorTok{*}\NormalTok{ a }\OperatorTok{*}\NormalTok{ r2}\OperatorTok{;}
    
    \CommentTok{// Gradient (computed analytically)}
    \CommentTok{// dF/dx = 2(r² {-} ax)(2x {-} a) {-} 2a²x}
    \CommentTok{// dF/dy = 2(r² {-} ax)(2y) {-} 2a²y}
    \DataTypeTok{vec2}\NormalTok{ grad }\OperatorTok{=} \DataTypeTok{vec2}\OperatorTok{(}
        \FloatTok{2.0} \OperatorTok{*}\NormalTok{ lhs }\OperatorTok{*} \OperatorTok{(}\FloatTok{2.0} \OperatorTok{*}\NormalTok{ p}\OperatorTok{.}\FunctionTok{x} \OperatorTok{{-}}\NormalTok{ a}\OperatorTok{)} \OperatorTok{{-}} \FloatTok{2.0} \OperatorTok{*}\NormalTok{ a }\OperatorTok{*}\NormalTok{ a }\OperatorTok{*}\NormalTok{ p}\OperatorTok{.}\FunctionTok{x}\OperatorTok{,}
        \FloatTok{2.0} \OperatorTok{*}\NormalTok{ lhs }\OperatorTok{*} \OperatorTok{(}\FloatTok{2.0} \OperatorTok{*}\NormalTok{ p}\OperatorTok{.}\FunctionTok{y}\OperatorTok{)} \OperatorTok{{-}} \FloatTok{2.0} \OperatorTok{*}\NormalTok{ a }\OperatorTok{*}\NormalTok{ a }\OperatorTok{*}\NormalTok{ p}\OperatorTok{.}\FunctionTok{y}
    \OperatorTok{);}
    
    \DataTypeTok{float}\NormalTok{ dist }\OperatorTok{=} \BuiltInTok{abs}\OperatorTok{(}\NormalTok{F}\OperatorTok{)} \OperatorTok{/} \BuiltInTok{max}\OperatorTok{(}\BuiltInTok{length}\OperatorTok{(}\NormalTok{grad}\OperatorTok{),} \FloatTok{0.01}\OperatorTok{);}
    
    \DataTypeTok{vec3}\NormalTok{ color}\OperatorTok{;}
    \KeywordTok{if} \OperatorTok{(}\NormalTok{dist }\OperatorTok{\textless{}} \FloatTok{0.05}\OperatorTok{)} \OperatorTok{\{}
\NormalTok{        color }\OperatorTok{=} \DataTypeTok{vec3}\OperatorTok{(}\FloatTok{1.0}\OperatorTok{,} \FloatTok{0.5}\OperatorTok{,} \FloatTok{0.5}\OperatorTok{);}
    \OperatorTok{\}} \KeywordTok{else} \OperatorTok{\{}
\NormalTok{        color }\OperatorTok{=} \DataTypeTok{vec3}\OperatorTok{(}\FloatTok{0.1}\OperatorTok{,} \FloatTok{0.1}\OperatorTok{,} \FloatTok{0.3}\OperatorTok{);}
    \OperatorTok{\}}
    
\NormalTok{    fragColor }\OperatorTok{=} \DataTypeTok{vec4}\OperatorTok{(}\NormalTok{color}\OperatorTok{,} \FloatTok{1.0}\OperatorTok{);}
\OperatorTok{\}}
\end{Highlighting}
\end{Shaded}

\begin{center}\rule{0.5\linewidth}{0.5pt}\end{center}

\subsection{E4. Curve Explorer}\label{e4.-curve-explorer}

\textbf{Example: Superellipse} \(|x/a|^n + |y/b|^n = 1\), with exponent
\(n\) controlled by mouse.

\begin{Shaded}
\begin{Highlighting}[]
\DataTypeTok{vec2} \FunctionTok{normalize\_coord}\OperatorTok{(}\DataTypeTok{vec2}\NormalTok{ coord}\OperatorTok{)} \OperatorTok{\{}
    \DataTypeTok{vec2}\NormalTok{ uv }\OperatorTok{=}\NormalTok{ coord }\OperatorTok{/}\NormalTok{ iResolution}\OperatorTok{.}\FunctionTok{xy}\OperatorTok{;}
\NormalTok{    uv }\OperatorTok{=}\NormalTok{ uv }\OperatorTok{{-}} \DataTypeTok{vec2}\OperatorTok{(}\FloatTok{0.5}\OperatorTok{,} \FloatTok{0.5}\OperatorTok{);}
\NormalTok{    uv}\OperatorTok{.}\FunctionTok{x} \OperatorTok{*=}\NormalTok{ iResolution}\OperatorTok{.}\FunctionTok{x} \OperatorTok{/}\NormalTok{ iResolution}\OperatorTok{.}\FunctionTok{y}\OperatorTok{;}
    \KeywordTok{return}\NormalTok{ uv }\OperatorTok{*} \FloatTok{4.0}\OperatorTok{;}
\OperatorTok{\}}

\DataTypeTok{void} \FunctionTok{mainImage}\OperatorTok{(}\DataTypeTok{out} \DataTypeTok{vec4}\NormalTok{ fragColor}\OperatorTok{,} \DataTypeTok{in} \DataTypeTok{vec2}\NormalTok{ fragCoord}\OperatorTok{)}
\OperatorTok{\{}
    \DataTypeTok{vec2}\NormalTok{ p }\OperatorTok{=} \FunctionTok{normalize\_coord}\OperatorTok{(}\NormalTok{fragCoord}\OperatorTok{);}
    
    \CommentTok{// Map mouse x to exponent n in [0.5, 4.0]}
    \CommentTok{// n \textless{} 1: star shape, n = 1: diamond, n = 2: ellipse, n \textgreater{} 2: rounded rectangle}
    \DataTypeTok{float}\NormalTok{ n }\OperatorTok{=} \BuiltInTok{mix}\OperatorTok{(}\FloatTok{0.5}\OperatorTok{,} \FloatTok{4.0}\OperatorTok{,}\NormalTok{ iMouse}\OperatorTok{.}\FunctionTok{x} \OperatorTok{/}\NormalTok{ iResolution}\OperatorTok{.}\FunctionTok{x}\OperatorTok{);}
    
    \DataTypeTok{float}\NormalTok{ a }\OperatorTok{=} \FloatTok{1.5}\OperatorTok{;}
    \DataTypeTok{float}\NormalTok{ b }\OperatorTok{=} \FloatTok{1.0}\OperatorTok{;}
    
    \CommentTok{// Superellipse: |x/a|\^{}n + |y/b|\^{}n = 1}
    \CommentTok{// F = |x/a|\^{}n + |y/b|\^{}n {-} 1}
    \DataTypeTok{float}\NormalTok{ F }\OperatorTok{=} \BuiltInTok{pow}\OperatorTok{(}\BuiltInTok{abs}\OperatorTok{(}\NormalTok{p}\OperatorTok{.}\FunctionTok{x} \OperatorTok{/}\NormalTok{ a}\OperatorTok{),}\NormalTok{ n}\OperatorTok{)} \OperatorTok{+} \BuiltInTok{pow}\OperatorTok{(}\BuiltInTok{abs}\OperatorTok{(}\NormalTok{p}\OperatorTok{.}\FunctionTok{y} \OperatorTok{/}\NormalTok{ b}\OperatorTok{),}\NormalTok{ n}\OperatorTok{)} \OperatorTok{{-}} \FloatTok{1.0}\OperatorTok{;}
    
    \CommentTok{// Numerical gradient (analytical is messy for arbitrary n)}
    \DataTypeTok{float}\NormalTok{ eps\_grad }\OperatorTok{=} \FloatTok{0.01}\OperatorTok{;}
    \DataTypeTok{float}\NormalTok{ Fx }\OperatorTok{=} \BuiltInTok{pow}\OperatorTok{(}\BuiltInTok{abs}\OperatorTok{((}\NormalTok{p}\OperatorTok{.}\FunctionTok{x} \OperatorTok{+}\NormalTok{ eps\_grad}\OperatorTok{)} \OperatorTok{/}\NormalTok{ a}\OperatorTok{),}\NormalTok{ n}\OperatorTok{)} \OperatorTok{+} \BuiltInTok{pow}\OperatorTok{(}\BuiltInTok{abs}\OperatorTok{(}\NormalTok{p}\OperatorTok{.}\FunctionTok{y} \OperatorTok{/}\NormalTok{ b}\OperatorTok{),}\NormalTok{ n}\OperatorTok{)} \OperatorTok{{-}} \FloatTok{1.0}\OperatorTok{;}
    \DataTypeTok{float}\NormalTok{ Fy }\OperatorTok{=} \BuiltInTok{pow}\OperatorTok{(}\BuiltInTok{abs}\OperatorTok{(}\NormalTok{p}\OperatorTok{.}\FunctionTok{x} \OperatorTok{/}\NormalTok{ a}\OperatorTok{),}\NormalTok{ n}\OperatorTok{)} \OperatorTok{+} \BuiltInTok{pow}\OperatorTok{(}\BuiltInTok{abs}\OperatorTok{((}\NormalTok{p}\OperatorTok{.}\FunctionTok{y} \OperatorTok{+}\NormalTok{ eps\_grad}\OperatorTok{)} \OperatorTok{/}\NormalTok{ b}\OperatorTok{),}\NormalTok{ n}\OperatorTok{)} \OperatorTok{{-}} \FloatTok{1.0}\OperatorTok{;}
    \DataTypeTok{vec2}\NormalTok{ grad }\OperatorTok{=} \DataTypeTok{vec2}\OperatorTok{(}\NormalTok{Fx }\OperatorTok{{-}}\NormalTok{ F}\OperatorTok{,}\NormalTok{ Fy }\OperatorTok{{-}}\NormalTok{ F}\OperatorTok{)} \OperatorTok{/}\NormalTok{ eps\_grad}\OperatorTok{;}
    
    \DataTypeTok{float}\NormalTok{ dist }\OperatorTok{=} \BuiltInTok{abs}\OperatorTok{(}\NormalTok{F}\OperatorTok{)} \OperatorTok{/} \BuiltInTok{max}\OperatorTok{(}\BuiltInTok{length}\OperatorTok{(}\NormalTok{grad}\OperatorTok{),} \FloatTok{0.01}\OperatorTok{);}
    
    \DataTypeTok{vec3}\NormalTok{ color}\OperatorTok{;}
    \KeywordTok{if} \OperatorTok{(}\NormalTok{dist }\OperatorTok{\textless{}} \FloatTok{0.05}\OperatorTok{)} \OperatorTok{\{}
\NormalTok{        color }\OperatorTok{=} \DataTypeTok{vec3}\OperatorTok{(}\FloatTok{0.5}\OperatorTok{,} \FloatTok{1.0}\OperatorTok{,} \FloatTok{0.5}\OperatorTok{);}
    \OperatorTok{\}} \KeywordTok{else} \OperatorTok{\{}
\NormalTok{        color }\OperatorTok{=} \DataTypeTok{vec3}\OperatorTok{(}\FloatTok{0.1}\OperatorTok{,} \FloatTok{0.1}\OperatorTok{,} \FloatTok{0.3}\OperatorTok{);}
    \OperatorTok{\}}
    
\NormalTok{    fragColor }\OperatorTok{=} \DataTypeTok{vec4}\OperatorTok{(}\NormalTok{color}\OperatorTok{,} \FloatTok{1.0}\OperatorTok{);}
\OperatorTok{\}}
\end{Highlighting}
\end{Shaded}

\textbf{Teaching note:} Dragging from left to right morphs the shape
from a 4-pointed star (\(n < 1\)) through a diamond (\(n = 1\)), circle
(\(n = 2\)), to a rounded rectangle (\(n > 2\)). This family is called
the Lamé curves.

\begin{center}\rule{0.5\linewidth}{0.5pt}\end{center}

\subsection{E5. Two Circles (Venn
Diagram)}\label{e5.-two-circles-venn-diagram}

\textbf{Version 1: One circle in front of the other}

\begin{Shaded}
\begin{Highlighting}[]
\DataTypeTok{void} \FunctionTok{mainImage}\OperatorTok{(}\DataTypeTok{out} \DataTypeTok{vec4}\NormalTok{ fragColor}\OperatorTok{,} \DataTypeTok{in} \DataTypeTok{vec2}\NormalTok{ fragCoord}\OperatorTok{)}
\OperatorTok{\{}
    \DataTypeTok{vec2}\NormalTok{ uv }\OperatorTok{=}\NormalTok{ fragCoord }\OperatorTok{/}\NormalTok{ iResolution}\OperatorTok{.}\FunctionTok{xy}\OperatorTok{;}
\NormalTok{    uv }\OperatorTok{=}\NormalTok{ uv }\OperatorTok{{-}} \DataTypeTok{vec2}\OperatorTok{(}\FloatTok{0.5}\OperatorTok{,} \FloatTok{0.5}\OperatorTok{);}
\NormalTok{    uv}\OperatorTok{.}\FunctionTok{x} \OperatorTok{*=}\NormalTok{ iResolution}\OperatorTok{.}\FunctionTok{x} \OperatorTok{/}\NormalTok{ iResolution}\OperatorTok{.}\FunctionTok{y}\OperatorTok{;}
    \DataTypeTok{vec2}\NormalTok{ p }\OperatorTok{=}\NormalTok{ uv }\OperatorTok{*} \FloatTok{4.0}\OperatorTok{;}
    
    \CommentTok{// Two circles}
    \DataTypeTok{vec2}\NormalTok{ c1 }\OperatorTok{=} \DataTypeTok{vec2}\OperatorTok{({-}}\FloatTok{0.6}\OperatorTok{,} \FloatTok{0.0}\OperatorTok{);}
    \DataTypeTok{vec2}\NormalTok{ c2 }\OperatorTok{=} \DataTypeTok{vec2}\OperatorTok{(}\FloatTok{0.6}\OperatorTok{,} \FloatTok{0.0}\OperatorTok{);}
    \DataTypeTok{float}\NormalTok{ r }\OperatorTok{=} \FloatTok{1.0}\OperatorTok{;}
    
    \DataTypeTok{float}\NormalTok{ d1 }\OperatorTok{=} \BuiltInTok{length}\OperatorTok{(}\NormalTok{p }\OperatorTok{{-}}\NormalTok{ c1}\OperatorTok{);}
    \DataTypeTok{float}\NormalTok{ d2 }\OperatorTok{=} \BuiltInTok{length}\OperatorTok{(}\NormalTok{p }\OperatorTok{{-}}\NormalTok{ c2}\OperatorTok{);}
    
    \CommentTok{// Draw back to front: circle 1 first, then circle 2 on top}
    \DataTypeTok{vec3}\NormalTok{ color }\OperatorTok{=} \DataTypeTok{vec3}\OperatorTok{(}\FloatTok{0.1}\OperatorTok{,} \FloatTok{0.1}\OperatorTok{,} \FloatTok{0.2}\OperatorTok{);}  \CommentTok{// background}
    
    \KeywordTok{if} \OperatorTok{(}\NormalTok{d1 }\OperatorTok{\textless{}}\NormalTok{ r}\OperatorTok{)} \OperatorTok{\{}
\NormalTok{        color }\OperatorTok{=} \DataTypeTok{vec3}\OperatorTok{(}\FloatTok{1.0}\OperatorTok{,} \FloatTok{0.3}\OperatorTok{,} \FloatTok{0.3}\OperatorTok{);}  \CommentTok{// red circle 1}
    \OperatorTok{\}}
    \KeywordTok{if} \OperatorTok{(}\NormalTok{d2 }\OperatorTok{\textless{}}\NormalTok{ r}\OperatorTok{)} \OperatorTok{\{}
\NormalTok{        color }\OperatorTok{=} \DataTypeTok{vec3}\OperatorTok{(}\FloatTok{0.3}\OperatorTok{,} \FloatTok{0.3}\OperatorTok{,} \FloatTok{1.0}\OperatorTok{);}  \CommentTok{// blue circle 2 (drawn on top)}
    \OperatorTok{\}}
    
\NormalTok{    fragColor }\OperatorTok{=} \DataTypeTok{vec4}\OperatorTok{(}\NormalTok{color}\OperatorTok{,} \FloatTok{1.0}\OperatorTok{);}
\OperatorTok{\}}
\end{Highlighting}
\end{Shaded}

\textbf{Version 2: Venn diagram with three colors}

\begin{Shaded}
\begin{Highlighting}[]
\DataTypeTok{void} \FunctionTok{mainImage}\OperatorTok{(}\DataTypeTok{out} \DataTypeTok{vec4}\NormalTok{ fragColor}\OperatorTok{,} \DataTypeTok{in} \DataTypeTok{vec2}\NormalTok{ fragCoord}\OperatorTok{)}
\OperatorTok{\{}
    \DataTypeTok{vec2}\NormalTok{ uv }\OperatorTok{=}\NormalTok{ fragCoord }\OperatorTok{/}\NormalTok{ iResolution}\OperatorTok{.}\FunctionTok{xy}\OperatorTok{;}
\NormalTok{    uv }\OperatorTok{=}\NormalTok{ uv }\OperatorTok{{-}} \DataTypeTok{vec2}\OperatorTok{(}\FloatTok{0.5}\OperatorTok{,} \FloatTok{0.5}\OperatorTok{);}
\NormalTok{    uv}\OperatorTok{.}\FunctionTok{x} \OperatorTok{*=}\NormalTok{ iResolution}\OperatorTok{.}\FunctionTok{x} \OperatorTok{/}\NormalTok{ iResolution}\OperatorTok{.}\FunctionTok{y}\OperatorTok{;}
    \DataTypeTok{vec2}\NormalTok{ p }\OperatorTok{=}\NormalTok{ uv }\OperatorTok{*} \FloatTok{4.0}\OperatorTok{;}
    
    \CommentTok{// Two circles}
    \DataTypeTok{vec2}\NormalTok{ c1 }\OperatorTok{=} \DataTypeTok{vec2}\OperatorTok{({-}}\FloatTok{0.6}\OperatorTok{,} \FloatTok{0.0}\OperatorTok{);}
    \DataTypeTok{vec2}\NormalTok{ c2 }\OperatorTok{=} \DataTypeTok{vec2}\OperatorTok{(}\FloatTok{0.6}\OperatorTok{,} \FloatTok{0.0}\OperatorTok{);}
    \DataTypeTok{float}\NormalTok{ r }\OperatorTok{=} \FloatTok{1.0}\OperatorTok{;}
    
    \DataTypeTok{float}\NormalTok{ d1 }\OperatorTok{=} \BuiltInTok{length}\OperatorTok{(}\NormalTok{p }\OperatorTok{{-}}\NormalTok{ c1}\OperatorTok{);}
    \DataTypeTok{float}\NormalTok{ d2 }\OperatorTok{=} \BuiltInTok{length}\OperatorTok{(}\NormalTok{p }\OperatorTok{{-}}\NormalTok{ c2}\OperatorTok{);}
    
    \DataTypeTok{bool}\NormalTok{ in1 }\OperatorTok{=}\NormalTok{ d1 }\OperatorTok{\textless{}}\NormalTok{ r}\OperatorTok{;}
    \DataTypeTok{bool}\NormalTok{ in2 }\OperatorTok{=}\NormalTok{ d2 }\OperatorTok{\textless{}}\NormalTok{ r}\OperatorTok{;}
    
    \DataTypeTok{vec3}\NormalTok{ color}\OperatorTok{;}
    \KeywordTok{if} \OperatorTok{(}\NormalTok{in1 }\OperatorTok{\&\&}\NormalTok{ in2}\OperatorTok{)} \OperatorTok{\{}
\NormalTok{        color }\OperatorTok{=} \DataTypeTok{vec3}\OperatorTok{(}\FloatTok{1.0}\OperatorTok{,} \FloatTok{1.0}\OperatorTok{,} \FloatTok{0.0}\OperatorTok{);}  \CommentTok{// yellow intersection}
    \OperatorTok{\}} \KeywordTok{else} \KeywordTok{if} \OperatorTok{(}\NormalTok{in1}\OperatorTok{)} \OperatorTok{\{}
\NormalTok{        color }\OperatorTok{=} \DataTypeTok{vec3}\OperatorTok{(}\FloatTok{1.0}\OperatorTok{,} \FloatTok{0.3}\OperatorTok{,} \FloatTok{0.3}\OperatorTok{);}  \CommentTok{// red circle 1 only}
    \OperatorTok{\}} \KeywordTok{else} \KeywordTok{if} \OperatorTok{(}\NormalTok{in2}\OperatorTok{)} \OperatorTok{\{}
\NormalTok{        color }\OperatorTok{=} \DataTypeTok{vec3}\OperatorTok{(}\FloatTok{0.3}\OperatorTok{,} \FloatTok{0.3}\OperatorTok{,} \FloatTok{1.0}\OperatorTok{);}  \CommentTok{// blue circle 2 only}
    \OperatorTok{\}} \KeywordTok{else} \OperatorTok{\{}
\NormalTok{        color }\OperatorTok{=} \DataTypeTok{vec3}\OperatorTok{(}\FloatTok{0.1}\OperatorTok{,} \FloatTok{0.1}\OperatorTok{,} \FloatTok{0.2}\OperatorTok{);}  \CommentTok{// background}
    \OperatorTok{\}}
    
\NormalTok{    fragColor }\OperatorTok{=} \DataTypeTok{vec4}\OperatorTok{(}\NormalTok{color}\OperatorTok{,} \FloatTok{1.0}\OperatorTok{);}
\OperatorTok{\}}
\end{Highlighting}
\end{Shaded}

\begin{center}\rule{0.5\linewidth}{0.5pt}\end{center}

\section{Challenges}\label{challenges-1}

\subsection{H1. Parabola Graphing
Calculator}\label{h1.-parabola-graphing-calculator}

\begin{Shaded}
\begin{Highlighting}[]
\DataTypeTok{vec2} \FunctionTok{normalize\_coord}\OperatorTok{(}\DataTypeTok{vec2}\NormalTok{ coord}\OperatorTok{)} \OperatorTok{\{}
    \DataTypeTok{vec2}\NormalTok{ uv }\OperatorTok{=}\NormalTok{ coord }\OperatorTok{/}\NormalTok{ iResolution}\OperatorTok{.}\FunctionTok{xy}\OperatorTok{;}
\NormalTok{    uv }\OperatorTok{=}\NormalTok{ uv }\OperatorTok{{-}} \DataTypeTok{vec2}\OperatorTok{(}\FloatTok{0.5}\OperatorTok{,} \FloatTok{0.5}\OperatorTok{);}
\NormalTok{    uv}\OperatorTok{.}\FunctionTok{x} \OperatorTok{*=}\NormalTok{ iResolution}\OperatorTok{.}\FunctionTok{x} \OperatorTok{/}\NormalTok{ iResolution}\OperatorTok{.}\FunctionTok{y}\OperatorTok{;}
    \KeywordTok{return}\NormalTok{ uv }\OperatorTok{*} \FloatTok{4.0}\OperatorTok{;}
\OperatorTok{\}}

\DataTypeTok{void} \FunctionTok{mainImage}\OperatorTok{(}\DataTypeTok{out} \DataTypeTok{vec4}\NormalTok{ fragColor}\OperatorTok{,} \DataTypeTok{in} \DataTypeTok{vec2}\NormalTok{ fragCoord}\OperatorTok{)}
\OperatorTok{\{}
    \DataTypeTok{vec2}\NormalTok{ p }\OperatorTok{=} \FunctionTok{normalize\_coord}\OperatorTok{(}\NormalTok{fragCoord}\OperatorTok{);}
    
    \CommentTok{// Map mouse to coefficients}
    \CommentTok{// a in [{-}2, 2], b in [{-}3, 3], c fixed at {-}1}
    \DataTypeTok{float}\NormalTok{ a }\OperatorTok{=} \BuiltInTok{mix}\OperatorTok{({-}}\FloatTok{2.0}\OperatorTok{,} \FloatTok{2.0}\OperatorTok{,}\NormalTok{ iMouse}\OperatorTok{.}\FunctionTok{x} \OperatorTok{/}\NormalTok{ iResolution}\OperatorTok{.}\FunctionTok{x}\OperatorTok{);}
    \DataTypeTok{float}\NormalTok{ b }\OperatorTok{=} \BuiltInTok{mix}\OperatorTok{({-}}\FloatTok{3.0}\OperatorTok{,} \FloatTok{3.0}\OperatorTok{,}\NormalTok{ iMouse}\OperatorTok{.}\FunctionTok{y} \OperatorTok{/}\NormalTok{ iResolution}\OperatorTok{.}\FunctionTok{y}\OperatorTok{);}
    \DataTypeTok{float}\NormalTok{ c }\OperatorTok{=} \OperatorTok{{-}}\FloatTok{1.0}\OperatorTok{;}
    
    \DataTypeTok{vec3}\NormalTok{ color }\OperatorTok{=} \DataTypeTok{vec3}\OperatorTok{(}\FloatTok{0.1}\OperatorTok{,} \FloatTok{0.1}\OperatorTok{,} \FloatTok{0.15}\OperatorTok{);}  \CommentTok{// background}
    
    \CommentTok{// Draw axes}
    \DataTypeTok{float}\NormalTok{ axis\_thickness }\OperatorTok{=} \FloatTok{0.03}\OperatorTok{;}
    \KeywordTok{if} \OperatorTok{(}\BuiltInTok{abs}\OperatorTok{(}\NormalTok{p}\OperatorTok{.}\FunctionTok{x}\OperatorTok{)} \OperatorTok{\textless{}}\NormalTok{ axis\_thickness}\OperatorTok{)} \OperatorTok{\{}
\NormalTok{        color }\OperatorTok{=} \DataTypeTok{vec3}\OperatorTok{(}\FloatTok{0.3}\OperatorTok{,} \FloatTok{0.3}\OperatorTok{,} \FloatTok{0.3}\OperatorTok{);}  \CommentTok{// y{-}axis}
    \OperatorTok{\}}
    \KeywordTok{if} \OperatorTok{(}\BuiltInTok{abs}\OperatorTok{(}\NormalTok{p}\OperatorTok{.}\FunctionTok{y}\OperatorTok{)} \OperatorTok{\textless{}}\NormalTok{ axis\_thickness}\OperatorTok{)} \OperatorTok{\{}
\NormalTok{        color }\OperatorTok{=} \DataTypeTok{vec3}\OperatorTok{(}\FloatTok{0.3}\OperatorTok{,} \FloatTok{0.3}\OperatorTok{,} \FloatTok{0.3}\OperatorTok{);}  \CommentTok{// x{-}axis}
    \OperatorTok{\}}
    
    \CommentTok{// Draw parabola: y = ax² + bx + c}
    \CommentTok{// Implicit: F = y {-} ax² {-} bx {-} c}
    \DataTypeTok{float}\NormalTok{ F }\OperatorTok{=}\NormalTok{ p}\OperatorTok{.}\FunctionTok{y} \OperatorTok{{-}}\NormalTok{ a }\OperatorTok{*}\NormalTok{ p}\OperatorTok{.}\FunctionTok{x} \OperatorTok{*}\NormalTok{ p}\OperatorTok{.}\FunctionTok{x} \OperatorTok{{-}}\NormalTok{ b }\OperatorTok{*}\NormalTok{ p}\OperatorTok{.}\FunctionTok{x} \OperatorTok{{-}}\NormalTok{ c}\OperatorTok{;}
    \DataTypeTok{vec2}\NormalTok{ grad }\OperatorTok{=} \DataTypeTok{vec2}\OperatorTok{({-}}\FloatTok{2.0} \OperatorTok{*}\NormalTok{ a }\OperatorTok{*}\NormalTok{ p}\OperatorTok{.}\FunctionTok{x} \OperatorTok{{-}}\NormalTok{ b}\OperatorTok{,} \FloatTok{1.0}\OperatorTok{);}
    \DataTypeTok{float}\NormalTok{ dist }\OperatorTok{=} \BuiltInTok{abs}\OperatorTok{(}\NormalTok{F}\OperatorTok{)} \OperatorTok{/} \BuiltInTok{length}\OperatorTok{(}\NormalTok{grad}\OperatorTok{);}
    
    \KeywordTok{if} \OperatorTok{(}\NormalTok{dist }\OperatorTok{\textless{}} \FloatTok{0.05}\OperatorTok{)} \OperatorTok{\{}
\NormalTok{        color }\OperatorTok{=} \DataTypeTok{vec3}\OperatorTok{(}\FloatTok{1.0}\OperatorTok{,} \FloatTok{1.0}\OperatorTok{,} \FloatTok{0.0}\OperatorTok{);}  \CommentTok{// yellow parabola}
    \OperatorTok{\}}
    
    \CommentTok{// Find and draw roots}
    \CommentTok{// ax² + bx + c = 0}
    \CommentTok{// x = ({-}b ± sqrt(b² {-} 4ac)) / 2a}
    \DataTypeTok{float}\NormalTok{ discriminant }\OperatorTok{=}\NormalTok{ b }\OperatorTok{*}\NormalTok{ b }\OperatorTok{{-}} \FloatTok{4.0} \OperatorTok{*}\NormalTok{ a }\OperatorTok{*}\NormalTok{ c}\OperatorTok{;}
    
    \KeywordTok{if} \OperatorTok{(}\NormalTok{discriminant }\OperatorTok{\textgreater{}=} \FloatTok{0.0} \OperatorTok{\&\&} \BuiltInTok{abs}\OperatorTok{(}\NormalTok{a}\OperatorTok{)} \OperatorTok{\textgreater{}} \FloatTok{0.01}\OperatorTok{)} \OperatorTok{\{}
        \DataTypeTok{float}\NormalTok{ sqrt\_disc }\OperatorTok{=} \BuiltInTok{sqrt}\OperatorTok{(}\NormalTok{discriminant}\OperatorTok{);}
        \DataTypeTok{float}\NormalTok{ x1 }\OperatorTok{=} \OperatorTok{({-}}\NormalTok{b }\OperatorTok{+}\NormalTok{ sqrt\_disc}\OperatorTok{)} \OperatorTok{/} \OperatorTok{(}\FloatTok{2.0} \OperatorTok{*}\NormalTok{ a}\OperatorTok{);}
        \DataTypeTok{float}\NormalTok{ x2 }\OperatorTok{=} \OperatorTok{({-}}\NormalTok{b }\OperatorTok{{-}}\NormalTok{ sqrt\_disc}\OperatorTok{)} \OperatorTok{/} \OperatorTok{(}\FloatTok{2.0} \OperatorTok{*}\NormalTok{ a}\OperatorTok{);}
        
        \CommentTok{// Draw circles around roots}
        \DataTypeTok{float}\NormalTok{ root\_radius }\OperatorTok{=} \FloatTok{0.15}\OperatorTok{;}
        \DataTypeTok{float}\NormalTok{ root\_thickness }\OperatorTok{=} \FloatTok{0.03}\OperatorTok{;}
        
        \DataTypeTok{float}\NormalTok{ d1 }\OperatorTok{=} \BuiltInTok{abs}\OperatorTok{(}\BuiltInTok{length}\OperatorTok{(}\NormalTok{p }\OperatorTok{{-}} \DataTypeTok{vec2}\OperatorTok{(}\NormalTok{x1}\OperatorTok{,} \FloatTok{0.0}\OperatorTok{))} \OperatorTok{{-}}\NormalTok{ root\_radius}\OperatorTok{);}
        \DataTypeTok{float}\NormalTok{ d2 }\OperatorTok{=} \BuiltInTok{abs}\OperatorTok{(}\BuiltInTok{length}\OperatorTok{(}\NormalTok{p }\OperatorTok{{-}} \DataTypeTok{vec2}\OperatorTok{(}\NormalTok{x2}\OperatorTok{,} \FloatTok{0.0}\OperatorTok{))} \OperatorTok{{-}}\NormalTok{ root\_radius}\OperatorTok{);}
        
        \KeywordTok{if} \OperatorTok{(}\NormalTok{d1 }\OperatorTok{\textless{}}\NormalTok{ root\_thickness}\OperatorTok{)} \OperatorTok{\{}
\NormalTok{            color }\OperatorTok{=} \DataTypeTok{vec3}\OperatorTok{(}\FloatTok{1.0}\OperatorTok{,} \FloatTok{0.3}\OperatorTok{,} \FloatTok{0.3}\OperatorTok{);}  \CommentTok{// red circle at root 1}
        \OperatorTok{\}}
        \KeywordTok{if} \OperatorTok{(}\NormalTok{d2 }\OperatorTok{\textless{}}\NormalTok{ root\_thickness}\OperatorTok{)} \OperatorTok{\{}
\NormalTok{            color }\OperatorTok{=} \DataTypeTok{vec3}\OperatorTok{(}\FloatTok{0.3}\OperatorTok{,} \FloatTok{1.0}\OperatorTok{,} \FloatTok{0.3}\OperatorTok{);}  \CommentTok{// green circle at root 2}
        \OperatorTok{\}}
    \OperatorTok{\}}
    
\NormalTok{    fragColor }\OperatorTok{=} \DataTypeTok{vec4}\OperatorTok{(}\NormalTok{color}\OperatorTok{,} \FloatTok{1.0}\OperatorTok{);}
\OperatorTok{\}}
\end{Highlighting}
\end{Shaded}

\begin{center}\rule{0.5\linewidth}{0.5pt}\end{center}

\subsection{H2. Elliptic Curve
Explorer}\label{h2.-elliptic-curve-explorer}

\begin{Shaded}
\begin{Highlighting}[]
\DataTypeTok{vec2} \FunctionTok{normalize\_coord}\OperatorTok{(}\DataTypeTok{vec2}\NormalTok{ coord}\OperatorTok{)} \OperatorTok{\{}
    \DataTypeTok{vec2}\NormalTok{ uv }\OperatorTok{=}\NormalTok{ coord }\OperatorTok{/}\NormalTok{ iResolution}\OperatorTok{.}\FunctionTok{xy}\OperatorTok{;}
\NormalTok{    uv }\OperatorTok{=}\NormalTok{ uv }\OperatorTok{{-}} \DataTypeTok{vec2}\OperatorTok{(}\FloatTok{0.5}\OperatorTok{,} \FloatTok{0.5}\OperatorTok{);}
\NormalTok{    uv}\OperatorTok{.}\FunctionTok{x} \OperatorTok{*=}\NormalTok{ iResolution}\OperatorTok{.}\FunctionTok{x} \OperatorTok{/}\NormalTok{ iResolution}\OperatorTok{.}\FunctionTok{y}\OperatorTok{;}
    \KeywordTok{return}\NormalTok{ uv }\OperatorTok{*} \FloatTok{4.0}\OperatorTok{;}
\OperatorTok{\}}

\DataTypeTok{void} \FunctionTok{mainImage}\OperatorTok{(}\DataTypeTok{out} \DataTypeTok{vec4}\NormalTok{ fragColor}\OperatorTok{,} \DataTypeTok{in} \DataTypeTok{vec2}\NormalTok{ fragCoord}\OperatorTok{)}
\OperatorTok{\{}
    \DataTypeTok{vec2}\NormalTok{ p }\OperatorTok{=} \FunctionTok{normalize\_coord}\OperatorTok{(}\NormalTok{fragCoord}\OperatorTok{);}
    
    \CommentTok{// Map mouse to (a, b) parameter space}
    \DataTypeTok{float}\NormalTok{ a }\OperatorTok{=} \BuiltInTok{mix}\OperatorTok{({-}}\FloatTok{3.0}\OperatorTok{,} \FloatTok{1.0}\OperatorTok{,}\NormalTok{ iMouse}\OperatorTok{.}\FunctionTok{x} \OperatorTok{/}\NormalTok{ iResolution}\OperatorTok{.}\FunctionTok{x}\OperatorTok{);}
    \DataTypeTok{float}\NormalTok{ b }\OperatorTok{=} \BuiltInTok{mix}\OperatorTok{({-}}\FloatTok{2.0}\OperatorTok{,} \FloatTok{2.0}\OperatorTok{,}\NormalTok{ iMouse}\OperatorTok{.}\FunctionTok{y} \OperatorTok{/}\NormalTok{ iResolution}\OperatorTok{.}\FunctionTok{y}\OperatorTok{);}
    
    \CommentTok{// Discriminant}
    \DataTypeTok{float}\NormalTok{ disc }\OperatorTok{=} \FloatTok{4.0} \OperatorTok{*}\NormalTok{ a }\OperatorTok{*}\NormalTok{ a }\OperatorTok{*}\NormalTok{ a }\OperatorTok{+} \FloatTok{27.0} \OperatorTok{*}\NormalTok{ b }\OperatorTok{*}\NormalTok{ b}\OperatorTok{;}
    
    \CommentTok{// Elliptic curve: y² = x³ + ax + b}
    \DataTypeTok{float}\NormalTok{ F }\OperatorTok{=}\NormalTok{ p}\OperatorTok{.}\FunctionTok{y} \OperatorTok{*}\NormalTok{ p}\OperatorTok{.}\FunctionTok{y} \OperatorTok{{-}}\NormalTok{ p}\OperatorTok{.}\FunctionTok{x} \OperatorTok{*}\NormalTok{ p}\OperatorTok{.}\FunctionTok{x} \OperatorTok{*}\NormalTok{ p}\OperatorTok{.}\FunctionTok{x} \OperatorTok{{-}}\NormalTok{ a }\OperatorTok{*}\NormalTok{ p}\OperatorTok{.}\FunctionTok{x} \OperatorTok{{-}}\NormalTok{ b}\OperatorTok{;}
    \DataTypeTok{vec2}\NormalTok{ grad }\OperatorTok{=} \DataTypeTok{vec2}\OperatorTok{({-}}\FloatTok{3.0} \OperatorTok{*}\NormalTok{ p}\OperatorTok{.}\FunctionTok{x} \OperatorTok{*}\NormalTok{ p}\OperatorTok{.}\FunctionTok{x} \OperatorTok{{-}}\NormalTok{ a}\OperatorTok{,} \FloatTok{2.0} \OperatorTok{*}\NormalTok{ p}\OperatorTok{.}\FunctionTok{y}\OperatorTok{);}
    \DataTypeTok{float}\NormalTok{ dist }\OperatorTok{=} \BuiltInTok{abs}\OperatorTok{(}\NormalTok{F}\OperatorTok{)} \OperatorTok{/} \BuiltInTok{max}\OperatorTok{(}\BuiltInTok{length}\OperatorTok{(}\NormalTok{grad}\OperatorTok{),} \FloatTok{0.01}\OperatorTok{);}
    
    \DataTypeTok{vec3}\NormalTok{ color }\OperatorTok{=} \DataTypeTok{vec3}\OperatorTok{(}\FloatTok{0.1}\OperatorTok{,} \FloatTok{0.1}\OperatorTok{,} \FloatTok{0.2}\OperatorTok{);}
    
    \KeywordTok{if} \OperatorTok{(}\NormalTok{dist }\OperatorTok{\textless{}} \FloatTok{0.05}\OperatorTok{)} \OperatorTok{\{}
        \CommentTok{// Color based on discriminant: red for singular, yellow for smooth}
        \KeywordTok{if} \OperatorTok{(}\BuiltInTok{abs}\OperatorTok{(}\NormalTok{disc}\OperatorTok{)} \OperatorTok{\textless{}} \FloatTok{0.5}\OperatorTok{)} \OperatorTok{\{}
\NormalTok{            color }\OperatorTok{=} \DataTypeTok{vec3}\OperatorTok{(}\FloatTok{1.0}\OperatorTok{,} \FloatTok{0.2}\OperatorTok{,} \FloatTok{0.2}\OperatorTok{);}  \CommentTok{// red for singular}
        \OperatorTok{\}} \KeywordTok{else} \OperatorTok{\{}
\NormalTok{            color }\OperatorTok{=} \DataTypeTok{vec3}\OperatorTok{(}\FloatTok{1.0}\OperatorTok{,} \FloatTok{1.0}\OperatorTok{,} \FloatTok{0.0}\OperatorTok{);}  \CommentTok{// yellow for smooth}
        \OperatorTok{\}}
    \OperatorTok{\}}
    
\NormalTok{    fragColor }\OperatorTok{=} \DataTypeTok{vec4}\OperatorTok{(}\NormalTok{color}\OperatorTok{,} \FloatTok{1.0}\OperatorTok{);}
\OperatorTok{\}}
\end{Highlighting}
\end{Shaded}

\begin{center}\rule{0.5\linewidth}{0.5pt}\end{center}

\subsection{H3. Signed Distance
Functions}\label{h3.-signed-distance-functions}

\begin{Shaded}
\begin{Highlighting}[]
\DataTypeTok{void} \FunctionTok{mainImage}\OperatorTok{(}\DataTypeTok{out} \DataTypeTok{vec4}\NormalTok{ fragColor}\OperatorTok{,} \DataTypeTok{in} \DataTypeTok{vec2}\NormalTok{ fragCoord}\OperatorTok{)}
\OperatorTok{\{}
    \DataTypeTok{vec2}\NormalTok{ uv }\OperatorTok{=}\NormalTok{ fragCoord }\OperatorTok{/}\NormalTok{ iResolution}\OperatorTok{.}\FunctionTok{xy}\OperatorTok{;}
\NormalTok{    uv }\OperatorTok{=}\NormalTok{ uv }\OperatorTok{{-}} \DataTypeTok{vec2}\OperatorTok{(}\FloatTok{0.5}\OperatorTok{,} \FloatTok{0.5}\OperatorTok{);}
\NormalTok{    uv}\OperatorTok{.}\FunctionTok{x} \OperatorTok{*=}\NormalTok{ iResolution}\OperatorTok{.}\FunctionTok{x} \OperatorTok{/}\NormalTok{ iResolution}\OperatorTok{.}\FunctionTok{y}\OperatorTok{;}
    \DataTypeTok{vec2}\NormalTok{ p }\OperatorTok{=}\NormalTok{ uv }\OperatorTok{*} \FloatTok{4.0}\OperatorTok{;}
    
    \CommentTok{// Signed distance to half{-}plane x \textgreater{} 1}
    \CommentTok{// SDF: distance is negative when inside (x \textgreater{} 1), positive outside}
    \DataTypeTok{float}\NormalTok{ sdf\_halfplane }\OperatorTok{=} \FloatTok{1.0} \OperatorTok{{-}}\NormalTok{ p}\OperatorTok{.}\FunctionTok{x}\OperatorTok{;}  \CommentTok{// negative when x \textgreater{} 1}
    
    \CommentTok{// Signed distance to rectangle [{-}1, 1] x [{-}0.5, 0.5]}
    \CommentTok{// SDF for axis{-}aligned box}
    \DataTypeTok{vec2}\NormalTok{ box\_size }\OperatorTok{=} \DataTypeTok{vec2}\OperatorTok{(}\FloatTok{1.0}\OperatorTok{,} \FloatTok{0.5}\OperatorTok{);}
    \DataTypeTok{vec2}\NormalTok{ d }\OperatorTok{=} \BuiltInTok{abs}\OperatorTok{(}\NormalTok{p}\OperatorTok{)} \OperatorTok{{-}}\NormalTok{ box\_size}\OperatorTok{;}
    \DataTypeTok{float}\NormalTok{ sdf\_rect }\OperatorTok{=} \BuiltInTok{length}\OperatorTok{(}\BuiltInTok{max}\OperatorTok{(}\NormalTok{d}\OperatorTok{,} \FloatTok{0.0}\OperatorTok{))} \OperatorTok{+} \BuiltInTok{min}\OperatorTok{(}\BuiltInTok{max}\OperatorTok{(}\NormalTok{d}\OperatorTok{.}\FunctionTok{x}\OperatorTok{,}\NormalTok{ d}\OperatorTok{.}\FunctionTok{y}\OperatorTok{),} \FloatTok{0.0}\OperatorTok{);}
    
    \CommentTok{// Choose which to display (toggle by uncommenting)}
    \DataTypeTok{float}\NormalTok{ sdf }\OperatorTok{=}\NormalTok{ sdf\_rect}\OperatorTok{;}
    \CommentTok{// float sdf = sdf\_halfplane;}
    
    \CommentTok{// Draw with uniform thickness boundary}
    \DataTypeTok{float}\NormalTok{ eps }\OperatorTok{=} \FloatTok{0.05}\OperatorTok{;}
    
    \DataTypeTok{vec3}\NormalTok{ color}\OperatorTok{;}
    \KeywordTok{if} \OperatorTok{(}\NormalTok{sdf }\OperatorTok{\textless{}} \OperatorTok{{-}}\NormalTok{eps}\OperatorTok{)} \OperatorTok{\{}
\NormalTok{        color }\OperatorTok{=} \DataTypeTok{vec3}\OperatorTok{(}\FloatTok{0.3}\OperatorTok{,} \FloatTok{0.3}\OperatorTok{,} \FloatTok{0.6}\OperatorTok{);}  \CommentTok{// inside}
    \OperatorTok{\}} \KeywordTok{else} \KeywordTok{if} \OperatorTok{(}\NormalTok{sdf }\OperatorTok{\textless{}}\NormalTok{ eps}\OperatorTok{)} \OperatorTok{\{}
\NormalTok{        color }\OperatorTok{=} \DataTypeTok{vec3}\OperatorTok{(}\FloatTok{1.0}\OperatorTok{,} \FloatTok{1.0}\OperatorTok{,} \FloatTok{0.0}\OperatorTok{);}  \CommentTok{// boundary}
    \OperatorTok{\}} \KeywordTok{else} \OperatorTok{\{}
\NormalTok{        color }\OperatorTok{=} \DataTypeTok{vec3}\OperatorTok{(}\FloatTok{0.1}\OperatorTok{,} \FloatTok{0.1}\OperatorTok{,} \FloatTok{0.2}\OperatorTok{);}  \CommentTok{// outside}
    \OperatorTok{\}}
    
\NormalTok{    fragColor }\OperatorTok{=} \DataTypeTok{vec4}\OperatorTok{(}\NormalTok{color}\OperatorTok{,} \FloatTok{1.0}\OperatorTok{);}
\OperatorTok{\}}
\end{Highlighting}
\end{Shaded}

\textbf{Teaching note:} The key insight is that for the rectangle SDF: -
\texttt{max(d,\ 0.0)} handles points outside the box -
\texttt{min(max(d.x,\ d.y),\ 0.0)} handles points inside the box - No
gradient correction needed because it's a true distance function!

\begin{center}\rule{0.5\linewidth}{0.5pt}\end{center}

\subsection{H4. Smooth Blending}\label{h4.-smooth-blending}

\begin{Shaded}
\begin{Highlighting}[]
\CommentTok{// Smooth minimum function}
\DataTypeTok{float} \FunctionTok{smin}\OperatorTok{(}\DataTypeTok{float}\NormalTok{ a}\OperatorTok{,} \DataTypeTok{float}\NormalTok{ b}\OperatorTok{,} \DataTypeTok{float}\NormalTok{ k}\OperatorTok{)} \OperatorTok{\{}
    \DataTypeTok{float}\NormalTok{ h }\OperatorTok{=} \BuiltInTok{max}\OperatorTok{(}\NormalTok{k }\OperatorTok{{-}} \BuiltInTok{abs}\OperatorTok{(}\NormalTok{a }\OperatorTok{{-}}\NormalTok{ b}\OperatorTok{),} \FloatTok{0.0}\OperatorTok{)} \OperatorTok{/}\NormalTok{ k}\OperatorTok{;}
    \KeywordTok{return} \BuiltInTok{min}\OperatorTok{(}\NormalTok{a}\OperatorTok{,}\NormalTok{ b}\OperatorTok{)} \OperatorTok{{-}}\NormalTok{ h }\OperatorTok{*}\NormalTok{ h }\OperatorTok{*}\NormalTok{ k }\OperatorTok{*} \FloatTok{0.25}\OperatorTok{;}
\OperatorTok{\}}

\DataTypeTok{void} \FunctionTok{mainImage}\OperatorTok{(}\DataTypeTok{out} \DataTypeTok{vec4}\NormalTok{ fragColor}\OperatorTok{,} \DataTypeTok{in} \DataTypeTok{vec2}\NormalTok{ fragCoord}\OperatorTok{)}
\OperatorTok{\{}
    \DataTypeTok{vec2}\NormalTok{ uv }\OperatorTok{=}\NormalTok{ fragCoord }\OperatorTok{/}\NormalTok{ iResolution}\OperatorTok{.}\FunctionTok{xy}\OperatorTok{;}
\NormalTok{    uv }\OperatorTok{=}\NormalTok{ uv }\OperatorTok{{-}} \DataTypeTok{vec2}\OperatorTok{(}\FloatTok{0.5}\OperatorTok{,} \FloatTok{0.5}\OperatorTok{);}
\NormalTok{    uv}\OperatorTok{.}\FunctionTok{x} \OperatorTok{*=}\NormalTok{ iResolution}\OperatorTok{.}\FunctionTok{x} \OperatorTok{/}\NormalTok{ iResolution}\OperatorTok{.}\FunctionTok{y}\OperatorTok{;}
    \DataTypeTok{vec2}\NormalTok{ p }\OperatorTok{=}\NormalTok{ uv }\OperatorTok{*} \FloatTok{4.0}\OperatorTok{;}
    
    \CommentTok{// Two circles that orbit each other}
    \DataTypeTok{float}\NormalTok{ angle }\OperatorTok{=}\NormalTok{ iTime }\OperatorTok{*} \FloatTok{0.5}\OperatorTok{;}
    \DataTypeTok{vec2}\NormalTok{ c1 }\OperatorTok{=} \FloatTok{0.8} \OperatorTok{*} \DataTypeTok{vec2}\OperatorTok{(}\BuiltInTok{cos}\OperatorTok{(}\NormalTok{angle}\OperatorTok{),} \BuiltInTok{sin}\OperatorTok{(}\NormalTok{angle}\OperatorTok{));}
    \DataTypeTok{vec2}\NormalTok{ c2 }\OperatorTok{=} \FloatTok{0.8} \OperatorTok{*} \DataTypeTok{vec2}\OperatorTok{(}\BuiltInTok{cos}\OperatorTok{(}\NormalTok{angle }\OperatorTok{+} \FloatTok{3.14159}\OperatorTok{),} \BuiltInTok{sin}\OperatorTok{(}\NormalTok{angle }\OperatorTok{+} \FloatTok{3.14159}\OperatorTok{));}
    \DataTypeTok{float}\NormalTok{ r }\OperatorTok{=} \FloatTok{0.8}\OperatorTok{;}
    
    \DataTypeTok{float}\NormalTok{ d1 }\OperatorTok{=} \BuiltInTok{length}\OperatorTok{(}\NormalTok{p }\OperatorTok{{-}}\NormalTok{ c1}\OperatorTok{)} \OperatorTok{{-}}\NormalTok{ r}\OperatorTok{;}
    \DataTypeTok{float}\NormalTok{ d2 }\OperatorTok{=} \BuiltInTok{length}\OperatorTok{(}\NormalTok{p }\OperatorTok{{-}}\NormalTok{ c2}\OperatorTok{)} \OperatorTok{{-}}\NormalTok{ r}\OperatorTok{;}
    
    \CommentTok{// Smooth blend parameter (try 0.1 to 1.0)}
    \DataTypeTok{float}\NormalTok{ k }\OperatorTok{=} \FloatTok{0.5}\OperatorTok{;}
    \DataTypeTok{float}\NormalTok{ d }\OperatorTok{=} \FunctionTok{smin}\OperatorTok{(}\NormalTok{d1}\OperatorTok{,}\NormalTok{ d2}\OperatorTok{,}\NormalTok{ k}\OperatorTok{);}
    
    \DataTypeTok{vec3}\NormalTok{ color}\OperatorTok{;}
    \KeywordTok{if} \OperatorTok{(}\NormalTok{d }\OperatorTok{\textless{}} \FloatTok{0.0}\OperatorTok{)} \OperatorTok{\{}
\NormalTok{        color }\OperatorTok{=} \DataTypeTok{vec3}\OperatorTok{(}\FloatTok{1.0}\OperatorTok{,} \FloatTok{0.8}\OperatorTok{,} \FloatTok{0.2}\OperatorTok{);}
    \OperatorTok{\}} \KeywordTok{else} \OperatorTok{\{}
\NormalTok{        color }\OperatorTok{=} \DataTypeTok{vec3}\OperatorTok{(}\FloatTok{0.1}\OperatorTok{,} \FloatTok{0.1}\OperatorTok{,} \FloatTok{0.2}\OperatorTok{);}
    \OperatorTok{\}}
    
\NormalTok{    fragColor }\OperatorTok{=} \DataTypeTok{vec4}\OperatorTok{(}\NormalTok{color}\OperatorTok{,} \FloatTok{1.0}\OperatorTok{);}
\OperatorTok{\}}
\end{Highlighting}
\end{Shaded}

\textbf{Teaching note:} The \texttt{smin} function smoothly blends two
distance fields. The parameter \texttt{k} controls how ``soft'' the
blend is---larger values create more melting together. Watch the two
blobs merge and separate as they orbit.

\begin{center}\rule{0.5\linewidth}{0.5pt}\end{center}

\subsection{H5. Inversion}\label{h5.-inversion}

\begin{Shaded}
\begin{Highlighting}[]
\DataTypeTok{vec2} \FunctionTok{normalize\_coord}\OperatorTok{(}\DataTypeTok{vec2}\NormalTok{ coord}\OperatorTok{)} \OperatorTok{\{}
    \DataTypeTok{vec2}\NormalTok{ uv }\OperatorTok{=}\NormalTok{ coord }\OperatorTok{/}\NormalTok{ iResolution}\OperatorTok{.}\FunctionTok{xy}\OperatorTok{;}
\NormalTok{    uv }\OperatorTok{=}\NormalTok{ uv }\OperatorTok{{-}} \DataTypeTok{vec2}\OperatorTok{(}\FloatTok{0.5}\OperatorTok{,} \FloatTok{0.5}\OperatorTok{);}
\NormalTok{    uv}\OperatorTok{.}\FunctionTok{x} \OperatorTok{*=}\NormalTok{ iResolution}\OperatorTok{.}\FunctionTok{x} \OperatorTok{/}\NormalTok{ iResolution}\OperatorTok{.}\FunctionTok{y}\OperatorTok{;}
    \KeywordTok{return}\NormalTok{ uv }\OperatorTok{*} \FloatTok{4.0}\OperatorTok{;}
\OperatorTok{\}}

\DataTypeTok{void} \FunctionTok{mainImage}\OperatorTok{(}\DataTypeTok{out} \DataTypeTok{vec4}\NormalTok{ fragColor}\OperatorTok{,} \DataTypeTok{in} \DataTypeTok{vec2}\NormalTok{ fragCoord}\OperatorTok{)}
\OperatorTok{\{}
    \DataTypeTok{vec2}\NormalTok{ p }\OperatorTok{=} \FunctionTok{normalize\_coord}\OperatorTok{(}\NormalTok{fragCoord}\OperatorTok{);}
    
    \CommentTok{// Mouse controls the line position}
    \DataTypeTok{float}\NormalTok{ line\_x }\OperatorTok{=} \BuiltInTok{mix}\OperatorTok{(}\FloatTok{0.5}\OperatorTok{,} \FloatTok{2.0}\OperatorTok{,}\NormalTok{ iMouse}\OperatorTok{.}\FunctionTok{x} \OperatorTok{/}\NormalTok{ iResolution}\OperatorTok{.}\FunctionTok{x}\OperatorTok{);}
    
    \CommentTok{// Apply circle inversion: p {-}\textgreater{} p / |p|²}
    \DataTypeTok{float}\NormalTok{ r2 }\OperatorTok{=} \BuiltInTok{dot}\OperatorTok{(}\NormalTok{p}\OperatorTok{,}\NormalTok{ p}\OperatorTok{);}
    \DataTypeTok{vec2}\NormalTok{ p\_inv }\OperatorTok{=}\NormalTok{ p }\OperatorTok{/} \BuiltInTok{max}\OperatorTok{(}\NormalTok{r2}\OperatorTok{,} \FloatTok{0.01}\OperatorTok{);}  \CommentTok{// avoid division by zero}
    
    \DataTypeTok{vec3}\NormalTok{ color }\OperatorTok{=} \DataTypeTok{vec3}\OperatorTok{(}\FloatTok{0.1}\OperatorTok{,} \FloatTok{0.1}\OperatorTok{,} \FloatTok{0.2}\OperatorTok{);}
    
    \CommentTok{// Draw unit circle (the circle of inversion)}
    \KeywordTok{if} \OperatorTok{(}\BuiltInTok{abs}\OperatorTok{(}\BuiltInTok{length}\OperatorTok{(}\NormalTok{p}\OperatorTok{)} \OperatorTok{{-}} \FloatTok{1.0}\OperatorTok{)} \OperatorTok{\textless{}} \FloatTok{0.02}\OperatorTok{)} \OperatorTok{\{}
\NormalTok{        color }\OperatorTok{=} \DataTypeTok{vec3}\OperatorTok{(}\FloatTok{0.3}\OperatorTok{,} \FloatTok{0.3}\OperatorTok{,} \FloatTok{0.4}\OperatorTok{);}
    \OperatorTok{\}}
    
    \CommentTok{// Draw the original line x = line\_x (in cyan)}
    \KeywordTok{if} \OperatorTok{(}\BuiltInTok{abs}\OperatorTok{(}\NormalTok{p}\OperatorTok{.}\FunctionTok{x} \OperatorTok{{-}}\NormalTok{ line\_x}\OperatorTok{)} \OperatorTok{\textless{}} \FloatTok{0.03}\OperatorTok{)} \OperatorTok{\{}
\NormalTok{        color }\OperatorTok{=} \DataTypeTok{vec3}\OperatorTok{(}\FloatTok{0.0}\OperatorTok{,} \FloatTok{0.8}\OperatorTok{,} \FloatTok{0.8}\OperatorTok{);}
    \OperatorTok{\}}
    
    \CommentTok{// Draw the inverted line (becomes a circle through origin!) in yellow}
    \CommentTok{// The line x = a inverts to a circle with center (1/(2a), 0) and radius 1/(2a)}
    \DataTypeTok{float}\NormalTok{ inv\_radius }\OperatorTok{=} \FloatTok{1.0} \OperatorTok{/} \OperatorTok{(}\FloatTok{2.0} \OperatorTok{*}\NormalTok{ line\_x}\OperatorTok{);}
    \DataTypeTok{vec2}\NormalTok{ inv\_center }\OperatorTok{=} \DataTypeTok{vec2}\OperatorTok{(}\NormalTok{inv\_radius}\OperatorTok{,} \FloatTok{0.0}\OperatorTok{);}
    \KeywordTok{if} \OperatorTok{(}\BuiltInTok{abs}\OperatorTok{(}\BuiltInTok{length}\OperatorTok{(}\NormalTok{p }\OperatorTok{{-}}\NormalTok{ inv\_center}\OperatorTok{)} \OperatorTok{{-}}\NormalTok{ inv\_radius}\OperatorTok{)} \OperatorTok{\textless{}} \FloatTok{0.03}\OperatorTok{)} \OperatorTok{\{}
\NormalTok{        color }\OperatorTok{=} \DataTypeTok{vec3}\OperatorTok{(}\FloatTok{1.0}\OperatorTok{,} \FloatTok{1.0}\OperatorTok{,} \FloatTok{0.0}\OperatorTok{);}
    \OperatorTok{\}}
    
\NormalTok{    fragColor }\OperatorTok{=} \DataTypeTok{vec4}\OperatorTok{(}\NormalTok{color}\OperatorTok{,} \FloatTok{1.0}\OperatorTok{);}
\OperatorTok{\}}
\end{Highlighting}
\end{Shaded}

\textbf{Teaching note:} Key observations: - Lines through origin map to
themselves - Lines not through origin map to circles through origin
(shown here) - Circles through origin map to lines - Circles not through
origin map to circles - The unit circle (gray) is the ``mirror'' of the
inversion

Drag the mouse to move the original line (cyan) and watch its inverted
image (yellow circle) change.

\begin{center}\rule{0.5\linewidth}{0.5pt}\end{center}

\section{Project: Grid Patterns}\label{project-grid-patterns-1}

\textbf{Example solution: Truchet-style pattern}

\begin{Shaded}
\begin{Highlighting}[]
\DataTypeTok{void} \FunctionTok{mainImage}\OperatorTok{(}\DataTypeTok{out} \DataTypeTok{vec4}\NormalTok{ fragColor}\OperatorTok{,} \DataTypeTok{in} \DataTypeTok{vec2}\NormalTok{ fragCoord}\OperatorTok{)}
\OperatorTok{\{}
    \DataTypeTok{vec2}\NormalTok{ uv }\OperatorTok{=}\NormalTok{ fragCoord }\OperatorTok{/}\NormalTok{ iResolution}\OperatorTok{.}\FunctionTok{xy}\OperatorTok{;}
\NormalTok{    uv }\OperatorTok{=}\NormalTok{ uv }\OperatorTok{{-}} \DataTypeTok{vec2}\OperatorTok{(}\FloatTok{0.5}\OperatorTok{,} \FloatTok{0.5}\OperatorTok{);}
\NormalTok{    uv}\OperatorTok{.}\FunctionTok{x} \OperatorTok{*=}\NormalTok{ iResolution}\OperatorTok{.}\FunctionTok{x} \OperatorTok{/}\NormalTok{ iResolution}\OperatorTok{.}\FunctionTok{y}\OperatorTok{;}
    \DataTypeTok{vec2}\NormalTok{ p }\OperatorTok{=}\NormalTok{ uv }\OperatorTok{*} \FloatTok{4.0}\OperatorTok{;}
    
    \DataTypeTok{float}\NormalTok{ aspect }\OperatorTok{=}\NormalTok{ iResolution}\OperatorTok{.}\FunctionTok{x} \OperatorTok{/}\NormalTok{ iResolution}\OperatorTok{.}\FunctionTok{y}\OperatorTok{;}
    \DataTypeTok{float}\NormalTok{ N }\OperatorTok{=} \FloatTok{8.0}\OperatorTok{;}
    \DataTypeTok{float}\NormalTok{ L }\OperatorTok{=} \OperatorTok{(}\FloatTok{4.0} \OperatorTok{*}\NormalTok{ aspect}\OperatorTok{)} \OperatorTok{/}\NormalTok{ N}\OperatorTok{;}
    
    \DataTypeTok{vec2}\NormalTok{ cell\_id }\OperatorTok{=} \BuiltInTok{floor}\OperatorTok{(}\NormalTok{p }\OperatorTok{/}\NormalTok{ L}\OperatorTok{);}
    \DataTypeTok{vec2}\NormalTok{ cell\_p }\OperatorTok{=} \BuiltInTok{mod}\OperatorTok{(}\NormalTok{p}\OperatorTok{,}\NormalTok{ L}\OperatorTok{)} \OperatorTok{{-}} \DataTypeTok{vec2}\OperatorTok{(}\NormalTok{L}\OperatorTok{/}\FloatTok{2.0}\OperatorTok{,}\NormalTok{ L}\OperatorTok{/}\FloatTok{2.0}\OperatorTok{);}
    
    \CommentTok{// Pseudo{-}random per cell (determines orientation)}
    \DataTypeTok{float}\NormalTok{ rand }\OperatorTok{=} \BuiltInTok{fract}\OperatorTok{(}\BuiltInTok{sin}\OperatorTok{(}\BuiltInTok{dot}\OperatorTok{(}\NormalTok{cell\_id}\OperatorTok{,} \DataTypeTok{vec2}\OperatorTok{(}\FloatTok{12.9898}\OperatorTok{,} \FloatTok{78.233}\OperatorTok{)))} \OperatorTok{*} \FloatTok{43758.5453}\OperatorTok{);}
    
    \CommentTok{// Quarter circles in corners}
    \DataTypeTok{float}\NormalTok{ r }\OperatorTok{=}\NormalTok{ L }\OperatorTok{/} \FloatTok{2.0}\OperatorTok{;}
    \DataTypeTok{float}\NormalTok{ thickness }\OperatorTok{=} \FloatTok{0.05}\OperatorTok{;}
    
    \DataTypeTok{float}\NormalTok{ d}\OperatorTok{;}
    \KeywordTok{if} \OperatorTok{(}\NormalTok{rand }\OperatorTok{\textgreater{}} \FloatTok{0.5}\OperatorTok{)} \OperatorTok{\{}
        \CommentTok{// Arcs connecting top{-}left to bottom{-}right}
        \DataTypeTok{float}\NormalTok{ d1 }\OperatorTok{=} \BuiltInTok{abs}\OperatorTok{(}\BuiltInTok{length}\OperatorTok{(}\NormalTok{cell\_p }\OperatorTok{{-}} \DataTypeTok{vec2}\OperatorTok{({-}}\NormalTok{L}\OperatorTok{/}\FloatTok{2.0}\OperatorTok{,}\NormalTok{ L}\OperatorTok{/}\FloatTok{2.0}\OperatorTok{))} \OperatorTok{{-}}\NormalTok{ r}\OperatorTok{);}
        \DataTypeTok{float}\NormalTok{ d2 }\OperatorTok{=} \BuiltInTok{abs}\OperatorTok{(}\BuiltInTok{length}\OperatorTok{(}\NormalTok{cell\_p }\OperatorTok{{-}} \DataTypeTok{vec2}\OperatorTok{(}\NormalTok{L}\OperatorTok{/}\FloatTok{2.0}\OperatorTok{,} \OperatorTok{{-}}\NormalTok{L}\OperatorTok{/}\FloatTok{2.0}\OperatorTok{))} \OperatorTok{{-}}\NormalTok{ r}\OperatorTok{);}
\NormalTok{        d }\OperatorTok{=} \BuiltInTok{min}\OperatorTok{(}\NormalTok{d1}\OperatorTok{,}\NormalTok{ d2}\OperatorTok{);}
    \OperatorTok{\}} \KeywordTok{else} \OperatorTok{\{}
        \CommentTok{// Arcs connecting top{-}right to bottom{-}left}
        \DataTypeTok{float}\NormalTok{ d1 }\OperatorTok{=} \BuiltInTok{abs}\OperatorTok{(}\BuiltInTok{length}\OperatorTok{(}\NormalTok{cell\_p }\OperatorTok{{-}} \DataTypeTok{vec2}\OperatorTok{(}\NormalTok{L}\OperatorTok{/}\FloatTok{2.0}\OperatorTok{,}\NormalTok{ L}\OperatorTok{/}\FloatTok{2.0}\OperatorTok{))} \OperatorTok{{-}}\NormalTok{ r}\OperatorTok{);}
        \DataTypeTok{float}\NormalTok{ d2 }\OperatorTok{=} \BuiltInTok{abs}\OperatorTok{(}\BuiltInTok{length}\OperatorTok{(}\NormalTok{cell\_p }\OperatorTok{{-}} \DataTypeTok{vec2}\OperatorTok{({-}}\NormalTok{L}\OperatorTok{/}\FloatTok{2.0}\OperatorTok{,} \OperatorTok{{-}}\NormalTok{L}\OperatorTok{/}\FloatTok{2.0}\OperatorTok{))} \OperatorTok{{-}}\NormalTok{ r}\OperatorTok{);}
\NormalTok{        d }\OperatorTok{=} \BuiltInTok{min}\OperatorTok{(}\NormalTok{d1}\OperatorTok{,}\NormalTok{ d2}\OperatorTok{);}
    \OperatorTok{\}}
    
    \DataTypeTok{vec3}\NormalTok{ color}\OperatorTok{;}
    \KeywordTok{if} \OperatorTok{(}\NormalTok{d }\OperatorTok{\textless{}}\NormalTok{ thickness}\OperatorTok{)} \OperatorTok{\{}
\NormalTok{        color }\OperatorTok{=} \DataTypeTok{vec3}\OperatorTok{(}\FloatTok{1.0}\OperatorTok{,} \FloatTok{0.9}\OperatorTok{,} \FloatTok{0.7}\OperatorTok{);}
    \OperatorTok{\}} \KeywordTok{else} \OperatorTok{\{}
        \CommentTok{// Subtle gradient background}
        \DataTypeTok{float}\NormalTok{ t }\OperatorTok{=} \FloatTok{0.5} \OperatorTok{+} \FloatTok{0.5} \OperatorTok{*} \BuiltInTok{sin}\OperatorTok{(}\BuiltInTok{length}\OperatorTok{(}\NormalTok{cell\_id}\OperatorTok{)} \OperatorTok{*} \FloatTok{0.5} \OperatorTok{+}\NormalTok{ iTime }\OperatorTok{*} \FloatTok{0.5}\OperatorTok{);}
\NormalTok{        color }\OperatorTok{=} \BuiltInTok{mix}\OperatorTok{(}\DataTypeTok{vec3}\OperatorTok{(}\FloatTok{0.1}\OperatorTok{,} \FloatTok{0.15}\OperatorTok{,} \FloatTok{0.25}\OperatorTok{),} \DataTypeTok{vec3}\OperatorTok{(}\FloatTok{0.2}\OperatorTok{,} \FloatTok{0.1}\OperatorTok{,} \FloatTok{0.2}\OperatorTok{),}\NormalTok{ t}\OperatorTok{);}
    \OperatorTok{\}}
    
\NormalTok{    fragColor }\OperatorTok{=} \DataTypeTok{vec4}\OperatorTok{(}\NormalTok{color}\OperatorTok{,} \FloatTok{1.0}\OperatorTok{);}
\OperatorTok{\}}
\end{Highlighting}
\end{Shaded}

\textbf{Teaching note:} This creates the classic Truchet tile pattern
where quarter-circle arcs connect across cell boundaries to form
continuous meandering paths. The pseudo-random function determines each
cell's orientation.

\begin{center}\rule{0.5\linewidth}{0.5pt}\end{center}

\section{Project: Fourier Epicycles}\label{project-fourier-epicycles-1}

\textbf{Full solution with arms:}

\begin{Shaded}
\begin{Highlighting}[]
\DataTypeTok{vec2} \FunctionTok{normalize\_coord}\OperatorTok{(}\DataTypeTok{vec2}\NormalTok{ coord}\OperatorTok{)} \OperatorTok{\{}
    \DataTypeTok{vec2}\NormalTok{ uv }\OperatorTok{=}\NormalTok{ coord }\OperatorTok{/}\NormalTok{ iResolution}\OperatorTok{.}\FunctionTok{xy}\OperatorTok{;}
\NormalTok{    uv }\OperatorTok{=}\NormalTok{ uv }\OperatorTok{{-}} \DataTypeTok{vec2}\OperatorTok{(}\FloatTok{0.5}\OperatorTok{,} \FloatTok{0.5}\OperatorTok{);}
\NormalTok{    uv}\OperatorTok{.}\FunctionTok{x} \OperatorTok{*=}\NormalTok{ iResolution}\OperatorTok{.}\FunctionTok{x} \OperatorTok{/}\NormalTok{ iResolution}\OperatorTok{.}\FunctionTok{y}\OperatorTok{;}
    \KeywordTok{return}\NormalTok{ uv }\OperatorTok{*} \FloatTok{4.0}\OperatorTok{;}
\OperatorTok{\}}

\DataTypeTok{float} \FunctionTok{sd\_segment}\OperatorTok{(}\DataTypeTok{vec2}\NormalTok{ p}\OperatorTok{,} \DataTypeTok{vec2}\NormalTok{ a}\OperatorTok{,} \DataTypeTok{vec2}\NormalTok{ b}\OperatorTok{)} \OperatorTok{\{}
    \DataTypeTok{vec2}\NormalTok{ pa }\OperatorTok{=}\NormalTok{ p }\OperatorTok{{-}}\NormalTok{ a}\OperatorTok{;}
    \DataTypeTok{vec2}\NormalTok{ ba }\OperatorTok{=}\NormalTok{ b }\OperatorTok{{-}}\NormalTok{ a}\OperatorTok{;}
    \DataTypeTok{float}\NormalTok{ t }\OperatorTok{=} \BuiltInTok{clamp}\OperatorTok{(}\BuiltInTok{dot}\OperatorTok{(}\NormalTok{pa}\OperatorTok{,}\NormalTok{ ba}\OperatorTok{)} \OperatorTok{/} \BuiltInTok{dot}\OperatorTok{(}\NormalTok{ba}\OperatorTok{,}\NormalTok{ ba}\OperatorTok{),} \FloatTok{0.0}\OperatorTok{,} \FloatTok{1.0}\OperatorTok{);}
    \KeywordTok{return} \BuiltInTok{length}\OperatorTok{(}\NormalTok{pa }\OperatorTok{{-}}\NormalTok{ ba }\OperatorTok{*}\NormalTok{ t}\OperatorTok{);}
\OperatorTok{\}}

\DataTypeTok{void} \FunctionTok{mainImage}\OperatorTok{(}\DataTypeTok{out} \DataTypeTok{vec4}\NormalTok{ fragColor}\OperatorTok{,} \DataTypeTok{in} \DataTypeTok{vec2}\NormalTok{ fragCoord}\OperatorTok{)}
\OperatorTok{\{}
    \DataTypeTok{vec2}\NormalTok{ p }\OperatorTok{=} \FunctionTok{normalize\_coord}\OperatorTok{(}\NormalTok{fragCoord}\OperatorTok{);}
    
    \DataTypeTok{float}\NormalTok{ omega }\OperatorTok{=} \FloatTok{1.0}\OperatorTok{;}
    \DataTypeTok{int}\NormalTok{ N }\OperatorTok{=} \DecValTok{7}\OperatorTok{;}
    \DataTypeTok{float}\NormalTok{ scale }\OperatorTok{=} \FloatTok{1.2}\OperatorTok{;}
    
    \DataTypeTok{vec3}\NormalTok{ color }\OperatorTok{=} \DataTypeTok{vec3}\OperatorTok{(}\FloatTok{0.02}\OperatorTok{,} \FloatTok{0.02}\OperatorTok{,} \FloatTok{0.05}\OperatorTok{);}
    
    \DataTypeTok{vec2}\NormalTok{ pos }\OperatorTok{=} \DataTypeTok{vec2}\OperatorTok{(}\FloatTok{0.0}\OperatorTok{,} \FloatTok{0.0}\OperatorTok{);}
    
    \KeywordTok{for} \OperatorTok{(}\DataTypeTok{int}\NormalTok{ i }\OperatorTok{=} \DecValTok{0}\OperatorTok{;}\NormalTok{ i }\OperatorTok{\textless{}} \DecValTok{20}\OperatorTok{;}\NormalTok{ i}\OperatorTok{++)} \OperatorTok{\{}
        \KeywordTok{if} \OperatorTok{(}\NormalTok{i }\OperatorTok{\textgreater{}=}\NormalTok{ N}\OperatorTok{)} \KeywordTok{break}\OperatorTok{;}
        
        \DataTypeTok{int}\NormalTok{ n }\OperatorTok{=} \DecValTok{2} \OperatorTok{*}\NormalTok{ i }\OperatorTok{+} \DecValTok{1}\OperatorTok{;}
        \DataTypeTok{float}\NormalTok{ r }\OperatorTok{=}\NormalTok{ scale }\OperatorTok{/} \DataTypeTok{float}\OperatorTok{(}\NormalTok{n}\OperatorTok{);}
        \DataTypeTok{float}\NormalTok{ freq }\OperatorTok{=} \DataTypeTok{float}\OperatorTok{(}\NormalTok{n}\OperatorTok{)} \OperatorTok{*}\NormalTok{ omega}\OperatorTok{;}
        \DataTypeTok{float}\NormalTok{ fade }\OperatorTok{=} \FloatTok{1.0} \OperatorTok{{-}} \DataTypeTok{float}\OperatorTok{(}\NormalTok{i}\OperatorTok{)} \OperatorTok{/} \DataTypeTok{float}\OperatorTok{(}\NormalTok{N}\OperatorTok{);}
        
        \DataTypeTok{vec2}\NormalTok{ next\_pos }\OperatorTok{=}\NormalTok{ pos }\OperatorTok{+}\NormalTok{ r }\OperatorTok{*} \DataTypeTok{vec2}\OperatorTok{(}\BuiltInTok{cos}\OperatorTok{(}\NormalTok{freq }\OperatorTok{*}\NormalTok{ iTime}\OperatorTok{),} \BuiltInTok{sin}\OperatorTok{(}\NormalTok{freq }\OperatorTok{*}\NormalTok{ iTime}\OperatorTok{));}
        
        \CommentTok{// Circle}
        \DataTypeTok{float}\NormalTok{ d\_circle }\OperatorTok{=} \BuiltInTok{abs}\OperatorTok{(}\BuiltInTok{length}\OperatorTok{(}\NormalTok{p }\OperatorTok{{-}}\NormalTok{ pos}\OperatorTok{)} \OperatorTok{{-}}\NormalTok{ r}\OperatorTok{);}
        \KeywordTok{if} \OperatorTok{(}\NormalTok{d\_circle }\OperatorTok{\textless{}} \FloatTok{0.02}\OperatorTok{)} \OperatorTok{\{}
\NormalTok{            color }\OperatorTok{=} \BuiltInTok{mix}\OperatorTok{(}\NormalTok{color}\OperatorTok{,} \DataTypeTok{vec3}\OperatorTok{(}\FloatTok{0.2}\OperatorTok{,} \FloatTok{0.2}\OperatorTok{,} \FloatTok{0.35}\OperatorTok{),} \FloatTok{0.5} \OperatorTok{*}\NormalTok{ fade}\OperatorTok{);}
        \OperatorTok{\}}
        
        \CommentTok{// Arm}
        \DataTypeTok{float}\NormalTok{ d\_arm }\OperatorTok{=} \FunctionTok{sd\_segment}\OperatorTok{(}\NormalTok{p}\OperatorTok{,}\NormalTok{ pos}\OperatorTok{,}\NormalTok{ next\_pos}\OperatorTok{);}
        \KeywordTok{if} \OperatorTok{(}\NormalTok{d\_arm }\OperatorTok{\textless{}} \FloatTok{0.015}\OperatorTok{)} \OperatorTok{\{}
\NormalTok{            color }\OperatorTok{=} \BuiltInTok{mix}\OperatorTok{(}\NormalTok{color}\OperatorTok{,} \DataTypeTok{vec3}\OperatorTok{(}\FloatTok{0.3}\OperatorTok{,} \FloatTok{0.3}\OperatorTok{,} \FloatTok{0.4}\OperatorTok{),} \FloatTok{0.7} \OperatorTok{*}\NormalTok{ fade}\OperatorTok{);}
        \OperatorTok{\}}
        
\NormalTok{        pos }\OperatorTok{=}\NormalTok{ next\_pos}\OperatorTok{;}
    \OperatorTok{\}}
    
    \CommentTok{// Final point}
    \DataTypeTok{float}\NormalTok{ d\_point }\OperatorTok{=} \BuiltInTok{length}\OperatorTok{(}\NormalTok{p }\OperatorTok{{-}}\NormalTok{ pos}\OperatorTok{);}
    \KeywordTok{if} \OperatorTok{(}\NormalTok{d\_point }\OperatorTok{\textless{}} \FloatTok{0.08}\OperatorTok{)} \OperatorTok{\{}
\NormalTok{        color }\OperatorTok{=} \DataTypeTok{vec3}\OperatorTok{(}\FloatTok{1.0}\OperatorTok{,} \FloatTok{1.0}\OperatorTok{,} \FloatTok{0.2}\OperatorTok{);}
    \OperatorTok{\}}
    
\NormalTok{    fragColor }\OperatorTok{=} \DataTypeTok{vec4}\OperatorTok{(}\NormalTok{color}\OperatorTok{,} \FloatTok{1.0}\OperatorTok{);}
\OperatorTok{\}}
\end{Highlighting}
\end{Shaded}

\textbf{Variation: Mouse-controlled number of terms:}

\begin{Shaded}
\begin{Highlighting}[]
\DataTypeTok{vec2} \FunctionTok{normalize\_coord}\OperatorTok{(}\DataTypeTok{vec2}\NormalTok{ coord}\OperatorTok{)} \OperatorTok{\{}
    \DataTypeTok{vec2}\NormalTok{ uv }\OperatorTok{=}\NormalTok{ coord }\OperatorTok{/}\NormalTok{ iResolution}\OperatorTok{.}\FunctionTok{xy}\OperatorTok{;}
\NormalTok{    uv }\OperatorTok{=}\NormalTok{ uv }\OperatorTok{{-}} \DataTypeTok{vec2}\OperatorTok{(}\FloatTok{0.5}\OperatorTok{,} \FloatTok{0.5}\OperatorTok{);}
\NormalTok{    uv}\OperatorTok{.}\FunctionTok{x} \OperatorTok{*=}\NormalTok{ iResolution}\OperatorTok{.}\FunctionTok{x} \OperatorTok{/}\NormalTok{ iResolution}\OperatorTok{.}\FunctionTok{y}\OperatorTok{;}
    \KeywordTok{return}\NormalTok{ uv }\OperatorTok{*} \FloatTok{4.0}\OperatorTok{;}
\OperatorTok{\}}

\DataTypeTok{float} \FunctionTok{sd\_segment}\OperatorTok{(}\DataTypeTok{vec2}\NormalTok{ p}\OperatorTok{,} \DataTypeTok{vec2}\NormalTok{ a}\OperatorTok{,} \DataTypeTok{vec2}\NormalTok{ b}\OperatorTok{)} \OperatorTok{\{}
    \DataTypeTok{vec2}\NormalTok{ pa }\OperatorTok{=}\NormalTok{ p }\OperatorTok{{-}}\NormalTok{ a}\OperatorTok{;}
    \DataTypeTok{vec2}\NormalTok{ ba }\OperatorTok{=}\NormalTok{ b }\OperatorTok{{-}}\NormalTok{ a}\OperatorTok{;}
    \DataTypeTok{float}\NormalTok{ t }\OperatorTok{=} \BuiltInTok{clamp}\OperatorTok{(}\BuiltInTok{dot}\OperatorTok{(}\NormalTok{pa}\OperatorTok{,}\NormalTok{ ba}\OperatorTok{)} \OperatorTok{/} \BuiltInTok{dot}\OperatorTok{(}\NormalTok{ba}\OperatorTok{,}\NormalTok{ ba}\OperatorTok{),} \FloatTok{0.0}\OperatorTok{,} \FloatTok{1.0}\OperatorTok{);}
    \KeywordTok{return} \BuiltInTok{length}\OperatorTok{(}\NormalTok{pa }\OperatorTok{{-}}\NormalTok{ ba }\OperatorTok{*}\NormalTok{ t}\OperatorTok{);}
\OperatorTok{\}}

\DataTypeTok{void} \FunctionTok{mainImage}\OperatorTok{(}\DataTypeTok{out} \DataTypeTok{vec4}\NormalTok{ fragColor}\OperatorTok{,} \DataTypeTok{in} \DataTypeTok{vec2}\NormalTok{ fragCoord}\OperatorTok{)}
\OperatorTok{\{}
    \DataTypeTok{vec2}\NormalTok{ p }\OperatorTok{=} \FunctionTok{normalize\_coord}\OperatorTok{(}\NormalTok{fragCoord}\OperatorTok{);}
    
    \DataTypeTok{float}\NormalTok{ omega }\OperatorTok{=} \FloatTok{1.0}\OperatorTok{;}
    \CommentTok{// Mouse x controls number of terms (1 to 15)}
    \DataTypeTok{int}\NormalTok{ N }\OperatorTok{=} \DataTypeTok{int}\OperatorTok{(}\BuiltInTok{mix}\OperatorTok{(}\FloatTok{1.0}\OperatorTok{,} \FloatTok{15.0}\OperatorTok{,}\NormalTok{ iMouse}\OperatorTok{.}\FunctionTok{x} \OperatorTok{/}\NormalTok{ iResolution}\OperatorTok{.}\FunctionTok{x}\OperatorTok{));}
    \DataTypeTok{float}\NormalTok{ scale }\OperatorTok{=} \FloatTok{1.2}\OperatorTok{;}
    
    \DataTypeTok{vec3}\NormalTok{ color }\OperatorTok{=} \DataTypeTok{vec3}\OperatorTok{(}\FloatTok{0.02}\OperatorTok{,} \FloatTok{0.02}\OperatorTok{,} \FloatTok{0.05}\OperatorTok{);}
    
    \DataTypeTok{vec2}\NormalTok{ pos }\OperatorTok{=} \DataTypeTok{vec2}\OperatorTok{(}\FloatTok{0.0}\OperatorTok{,} \FloatTok{0.0}\OperatorTok{);}
    
    \KeywordTok{for} \OperatorTok{(}\DataTypeTok{int}\NormalTok{ i }\OperatorTok{=} \DecValTok{0}\OperatorTok{;}\NormalTok{ i }\OperatorTok{\textless{}} \DecValTok{20}\OperatorTok{;}\NormalTok{ i}\OperatorTok{++)} \OperatorTok{\{}
        \KeywordTok{if} \OperatorTok{(}\NormalTok{i }\OperatorTok{\textgreater{}=}\NormalTok{ N}\OperatorTok{)} \KeywordTok{break}\OperatorTok{;}
        
        \DataTypeTok{int}\NormalTok{ n }\OperatorTok{=} \DecValTok{2} \OperatorTok{*}\NormalTok{ i }\OperatorTok{+} \DecValTok{1}\OperatorTok{;}
        \DataTypeTok{float}\NormalTok{ r }\OperatorTok{=}\NormalTok{ scale }\OperatorTok{/} \DataTypeTok{float}\OperatorTok{(}\NormalTok{n}\OperatorTok{);}
        \DataTypeTok{float}\NormalTok{ freq }\OperatorTok{=} \DataTypeTok{float}\OperatorTok{(}\NormalTok{n}\OperatorTok{)} \OperatorTok{*}\NormalTok{ omega}\OperatorTok{;}
        \DataTypeTok{float}\NormalTok{ fade }\OperatorTok{=} \FloatTok{1.0} \OperatorTok{{-}} \DataTypeTok{float}\OperatorTok{(}\NormalTok{i}\OperatorTok{)} \OperatorTok{/} \DataTypeTok{float}\OperatorTok{(}\NormalTok{N}\OperatorTok{);}
        
        \DataTypeTok{vec2}\NormalTok{ next\_pos }\OperatorTok{=}\NormalTok{ pos }\OperatorTok{+}\NormalTok{ r }\OperatorTok{*} \DataTypeTok{vec2}\OperatorTok{(}\BuiltInTok{cos}\OperatorTok{(}\NormalTok{freq }\OperatorTok{*}\NormalTok{ iTime}\OperatorTok{),} \BuiltInTok{sin}\OperatorTok{(}\NormalTok{freq }\OperatorTok{*}\NormalTok{ iTime}\OperatorTok{));}
        
        \CommentTok{// Circle}
        \DataTypeTok{float}\NormalTok{ d\_circle }\OperatorTok{=} \BuiltInTok{abs}\OperatorTok{(}\BuiltInTok{length}\OperatorTok{(}\NormalTok{p }\OperatorTok{{-}}\NormalTok{ pos}\OperatorTok{)} \OperatorTok{{-}}\NormalTok{ r}\OperatorTok{);}
        \KeywordTok{if} \OperatorTok{(}\NormalTok{d\_circle }\OperatorTok{\textless{}} \FloatTok{0.02}\OperatorTok{)} \OperatorTok{\{}
\NormalTok{            color }\OperatorTok{=} \BuiltInTok{mix}\OperatorTok{(}\NormalTok{color}\OperatorTok{,} \DataTypeTok{vec3}\OperatorTok{(}\FloatTok{0.2}\OperatorTok{,} \FloatTok{0.2}\OperatorTok{,} \FloatTok{0.35}\OperatorTok{),} \FloatTok{0.5} \OperatorTok{*}\NormalTok{ fade}\OperatorTok{);}
        \OperatorTok{\}}
        
        \CommentTok{// Arm}
        \DataTypeTok{float}\NormalTok{ d\_arm }\OperatorTok{=} \FunctionTok{sd\_segment}\OperatorTok{(}\NormalTok{p}\OperatorTok{,}\NormalTok{ pos}\OperatorTok{,}\NormalTok{ next\_pos}\OperatorTok{);}
        \KeywordTok{if} \OperatorTok{(}\NormalTok{d\_arm }\OperatorTok{\textless{}} \FloatTok{0.015}\OperatorTok{)} \OperatorTok{\{}
\NormalTok{            color }\OperatorTok{=} \BuiltInTok{mix}\OperatorTok{(}\NormalTok{color}\OperatorTok{,} \DataTypeTok{vec3}\OperatorTok{(}\FloatTok{0.3}\OperatorTok{,} \FloatTok{0.3}\OperatorTok{,} \FloatTok{0.4}\OperatorTok{),} \FloatTok{0.7} \OperatorTok{*}\NormalTok{ fade}\OperatorTok{);}
        \OperatorTok{\}}
        
\NormalTok{        pos }\OperatorTok{=}\NormalTok{ next\_pos}\OperatorTok{;}
    \OperatorTok{\}}
    
    \CommentTok{// Final point}
    \DataTypeTok{float}\NormalTok{ d\_point }\OperatorTok{=} \BuiltInTok{length}\OperatorTok{(}\NormalTok{p }\OperatorTok{{-}}\NormalTok{ pos}\OperatorTok{);}
    \KeywordTok{if} \OperatorTok{(}\NormalTok{d\_point }\OperatorTok{\textless{}} \FloatTok{0.08}\OperatorTok{)} \OperatorTok{\{}
\NormalTok{        color }\OperatorTok{=} \DataTypeTok{vec3}\OperatorTok{(}\FloatTok{1.0}\OperatorTok{,} \FloatTok{1.0}\OperatorTok{,} \FloatTok{0.2}\OperatorTok{);}
    \OperatorTok{\}}
    
\NormalTok{    fragColor }\OperatorTok{=} \DataTypeTok{vec4}\OperatorTok{(}\NormalTok{color}\OperatorTok{,} \FloatTok{1.0}\OperatorTok{);}
\OperatorTok{\}}
\end{Highlighting}
\end{Shaded}

\textbf{Variation: Triangle wave (alternating signs, 1/n²
coefficients):}

\begin{Shaded}
\begin{Highlighting}[]
\CommentTok{// In the loop, replace the square wave coefficients with:}
\DataTypeTok{float} \BuiltInTok{sign} \OperatorTok{=} \OperatorTok{(}\NormalTok{i }\OperatorTok{\%} \DecValTok{2} \OperatorTok{==} \DecValTok{0}\OperatorTok{)} \OperatorTok{?} \FloatTok{1.0} \OperatorTok{:} \OperatorTok{{-}}\FloatTok{1.0}\OperatorTok{;}
\DataTypeTok{float}\NormalTok{ r }\OperatorTok{=}\NormalTok{ scale }\OperatorTok{*} \BuiltInTok{sign} \OperatorTok{/} \DataTypeTok{float}\OperatorTok{(}\NormalTok{n }\OperatorTok{*}\NormalTok{ n}\OperatorTok{);}
\end{Highlighting}
\end{Shaded}

\textbf{Teaching note:} The triangle wave converges faster than the
square wave because of the \(1/n^2\) coefficients. Students should
observe smoother motion with fewer terms.


\backmatter


\end{document}
